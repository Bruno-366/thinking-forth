%% Thinking Forth
%% Copyright (C) 2004 Leo Brodie
%% Initial transcription by Ed Beroset
%% 
%% Chapter: Appendix E, Summary of Style Conventions
\Chapmark{E}
\chapter{Summary~of
Style~Conventions}
The contents of this Appendix are in the public domain. We encourage
publication without restriction, provided that you credit the source.

\section{Spacing and Indentation Guidelines}
\index{I!Indentation}\index{S!Spacing}
\begin{list}
{}{\setlength{\parsep}{0cm}}
\item 1 space between the colon and the name
\item 2 spaces between the name and the comment\footnote{
An often-seen alternative calls for 1 space between the name and comment and 3 between
the comment and the definition. A more liberal technique uses 3 spaces before and after the
comment. Whatever you choose, be consistent.
}
\item 2 spaces, or a carriage return, after the comment and before the definition\footnotemark[1]
\item 3 spaces between the name and definition if no comment is used
\item 3 spaces indentation on each subsequent line (or multiples of 3 for nested indentation)
\item 1 space between words/numbers within a phrase
\item 2 or 3 spaces between phrases
\item 1 space between the last word and the semicolon
\item 1 space between semicolon and {\bf IMMEDIATE} (if invoked)
\end{list}
No blank lines between definitions, except to separate distinct groups of
definitions

\section{Stack-Comment Abbreviations}
\index{S!Stack abbreviation standards|(}
\begin{tabular}{ll}
{}{\setlength{\parsep}{0cm}}n&single-length signed number\\
d&double-length signed number\\
u&single-length unsigned number\\
ud&double-length unsigned number\\
t&triple-length\\
q&quadruple-length\\
c&7-bit character value\\
b&8-bit byte\\
?&boolean flag; or:\\
t=&true\\
f=&false\\
a or adr&address\\
acf&address of code field\\
apf&address of parameter field\\
`&(as prefix) address of\\
s d&(as a pair) source destination\\
lo hi&lower-limit upper-limit (inclusive)\\
\#&count\\
o&offset\\
i&index\\
m&mask\\
x&don't care (data structure notation)\\
\end{tabular}

An ``offset'' is a difference expressed in absolute units, such as bytes.

An ``index'' is a difference expressed in logical units, such as
elements or records.
\index{S!Stack abbreviation standards|)}

\section{Input-Stream Comment Designations}
\index{I!Input-stream comment}
\begin{tabular}{ll}
{}{\setlength{\parsep}{0cm}}c&single character, blank-delimited\\
name&sequence of characters, blank delimited\\
text&sequence of characters, delimited by non-blank\\
\end{tabular}

Follow ``text'' with the actual delimiter required; e.g., text'' or text).

\section{Samples of Good Commenting Style}

Here are two sample screens to illustrate good commenting
style\index{C!Commenting style|(}.

\setcounter{screen}{126}
\begin{Screen}
\ Formatter         Data Structures -- p.2             06/06/83
 6 CONSTANT TMARGIN \ line# where body of text begins)
55 CONSTANT BMARGIN \ line# where body of text ends)

CREATE HEADER 82 ALLOT
  { 1left-ent | 1right-cnt | 80header >
CREATE FOOTER 82 ALLOT
  { 1left-cnt | 1right-ent | 80footer >

VARIABLE ACROSS   \ formatter's current horizontal position
VARIABLE DOWNWARD \ formatter's current vertical position
VARIABLE LEFT     \ current primary left margin
VARIABLE WALL     \ current primary right margin
VARIABLE WALL-WAS \ WALL when curr. line started being formt'd


\end{Screen}


\begin{Screen}
\ Formatter          positioning -- p.1               06/06/83
: SKIP  ( n)  ACROSS +! ;
: NEWLEFT  \ reset left margin
LEFT @  PERMANENT @ +  TEMPORARY @ +  ACROSS ! ;
: \LINE  \ begin new line
DOOR  CR'  1 DOWNWARD +!  NEWLEFT  WALL @  WALL-WAS ! ;
: AT-TOP?  ( -- t=at-top)  TMARGIN  DOWNWARD @ = ;
: >TMARGIN  \ move from crease to TMARGIN
O DOWNWARD !  BEGIN  \LINE  AT-TOP? UNTIL ;







\end{Screen}
\index{C!Commenting style|)}

\section{Naming Conventions}
\index{N!Naming conventions|(}
\begin{longtable}{lll}
{}{\setlength{\parsep}{0cm}}%
{\em Meaning}&{\em Form}&{\em Example}\\[2ex] \endhead
\underline{Arithmetic}\\
integer 1&1name&1+\\
integer 2&2name&2*\\
takes relative input parameters&+name&+DRAW\\
takes scaled input parameters&+name&*DRAW\\[1ex]
\underline{Compilation}\\
start of ``high-level'' code&name:&CASE:\\
end of ``high-level'' code&;name&;CODE\\
put something into dictionary&name,&C,\\
executes at compile time&[name]&[COMPILE]\\
slightly different&name' (prime)&CR'\\
internal form or primitive&(name)&(TYPE)\\
&or \(<\)name\(>\)&\(<\)TYPE\(>\)\\
compiling word run-time part:\\
 systems with no folding&lower-case&if\\
 systems with folding&(NAME)&(IF)\\
defining word&:name&:COLOR\\
block-number where overlay begins&nam{\bf ING}&DISKING\\[1ex]
\underline{Data Structures}\\
table or array&names&EMPLOYEES\\
total number of elements&\#name&\#EMPLOYEES\\
current item number (variable)&name\#&EMPLOYEE\#\\
sets current item&( n) name&13 EMPLOYEE\\
advance to next element&+name&+EMPLOYEE\\
size of offset to item from&name+&DATE +\\
beginning of structure\\
size of (bytes per)&/name&/EMPLOYEE\\
(short for BYTES/name)\\
index pointer&\(>\)name&\(>\)IN\\
convert address of structure to&\(>\)name&\(>\)BODY\\
address of item\\
file index&(name)&(PEOPLE)\\
file pointer&--name&--JOB\\
initialize structure&0name&0RECORD\\[1ex]
\underline{Direction, Conversion}\\
backwards&name\(<\)&SLIDE\(<\)\\
forwards&name \(>\)&CMOVE \(>\)\\
from&\(<\)name&\(<\)TAPE\\
to&\(>\)name&\(>\)TAPE\\
convert to&name\(>\)name&FEET\(>\)METERS\\
downward&$\backslash$name&$\backslash$LINE\\
upward&/name&/LINE\\
open&\{name&\{FILE\\
close&\}name&\}FILE\\[1ex]
\underline{Logic, Control}\\
return boolean value&name?&SHORT?\\
returns reversed boolean&-name?&-SHORT?\\
address of boolean&'name?&'SHORT?\\
operates conditionally&?name&?DUP\\
&&(maybe DUP)\\
enable&+name&+CLOCK\\
or, absence of symbol&name&BLINKING\\
disable&-name&-CLOCK\\
&&-BLINKING\\[1ex]
\underline{Memory}\\
save value of&@name&@CURSOR\\
restore value of&!name&!CURSOR\\
store into&name!&SECONDS!\\
fetch from&name@&INDEX@\\
name of buffer&:name&:INSERT\\
address of name&'name&'S\\
address of pointer to name&'name&'TYPE\\
exchange, especially bytes&\(>\) name\(<\)&\(>\)MOVE\(<\)\\[1ex]
\underline{Numeric Types}\\
byte length&Cname&C@\\
2 cell size, 2's complement&Dname&D+\\
integer encoding\\
mixed 16 and 32-bit operator&Mname&M*\\
3 cell size&Tname&T*\\
4 cell size&Qname&Q*\\
unsigned encoding&Uname&U.\\[1ex]
\underline{Output, Printing}\\
print item&.name&.S\\
print numeric (name denotes type)&name.&D. , U.\\
print right justified&name.R&U.R\\[1ex]
\underline{Quantity}\\
``per''&/name&/SIDE\\[1ex]
\underline{Sequencing}\\
start&\(<\) name&\(<\) \#\\
end&name \(>\)&\# \(>\)\\[1ex]
\underline{Text}\\
string follows delimited by ''&name''&ABORT'' text''\\
text or string operator&``name&``COMPARE\\
(similar to \$ prefix in BASIC\\
superstring array&``name''&``COLORS''\\
\end{longtable}
\index{N!Naming conventions|)}

\subsection{How to Pronounce the Symbols}
\index{S!Symbols, pronunciation of}
\begin{tabular}{ll}
{}{\setlength{\parsep}{0cm}}
!&store\\
@&fetch\\
\#&sharp (or ``number,'' as in \#RECORDS)\\
\$&dollar\\
\%&percent\\
\^{ }&caret\\
\&&ampersand\\
{*}&star\\
(&left paren; paren\\
)&right paren; paren\\
--&dash; not\\
+&plus\\
=&equals\\
\{ \}&faces (traditionally called ``curly brackets'')\\
{[} {]}&square brackets\\
``&quote\\
'&as prefix: tick; as suffix: prime\\
\~{ }&tilde\\
\(|\)&bar\\
\(\backslash\)&backslash. (also ``under,'' ``down,'' and ``skip'')\\
/&slash. (also ``up'')\\
\(<\)&less-than\\
 &left dart\\
\(>\)&greater-than\\
 &right dart\\
?&question (some prefer ``query'')\\
,&comma\\
.&dot\\
\end{tabular}


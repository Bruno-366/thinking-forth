\chapter{The Philosophy of FORTH}

FORTH is a language and an operating system. But that's not all; It's
also the embodiment of a philosophy. The philosophy is not generally
described as something apart from FORTH. It did not precede FORTH,
nor is it described anywhere apart from discussions of FORTH, nor
does it even have a name other than ''FORTH.''

What is this philosophy? How can you apply it to solve your software
problems? Before we can answer these questions, let's take 100 steps
backwards and examine some of the major philosophies advanced by computer
scientists over the years. After tracing the trajectory of these advances,
we'll compare---and contrast---FORTH with these state-of-the-art progamming
principles.


\section{An Armchair History of Software Elegance}

In the prehistoric days of programming, when computers were dinosaurs,
the mere fact that some genius could make a program run correctly
provided great cause for wonderment. As computers became more civilized,
the wonder waned. Management wanted more from programmers and from
their programs. As the cost of hardware steadily dropped, the cost
of software soared. It was no longer good enough for a program to
run correctly. It also had to be developed quickly and maintained
easily. A new demand began to share the spotlight with correctness.
The missing quality was called ''elegance.'' In this section we'll
outline a history of the tools and techniques for writing more elegant
programs.


\subsection{Memorability}

The first computer programs looked something like this:

\begin{verbatim}
00110101
11010011
11011001
\end{verbatim}
Programmers entered these programs by setting rows of switches---``on''
if the digit was {}``1,'' ``off'' if the digit was ``0.'' These
values were the {}``machine instructions'' for the computer, and
each one caused the computer to perform some mundane operation like
``Move the contents or Register B to Register A,'' or {}``Add the
contents of Register C into the contents of Register A.'' This proved
a bit tedious. Tedium being the stepmother of invention, some clever
programmers realized that the computer itself could be used to help.
so they wrote a program that translated easy-to-remember abbreviations
into the hard-to-remember bit patterns. The new language looked something
like this:

\begin{verbatim}
MOV B,A
ADD C,A
JMC REC1
\end{verbatim}
The translator program was called an \emph{assembler}, the new language
\emph{assembly language}. Each instruction {}``assembled'' the appropriate
bit pattern for that instruction, with a one-to-one correspondence
between assembly instruction and machine instruction. But names are
easier for programmers to remember. For this reason the new instructions
were called \emph{mnemonics}.


\subsection{Power}

Assembly-language programming is characterized by a one-for-one correspondence
between each command that the programmer types and each command that
the processor performs. In practice, programmers found themselves
often repeating the same \emph{sequence} of instructions over and
again to accomplish the same thing in different parts of the program.
How nice it would be to have a name which would represent each of
these common sequences.

This need was met by the {}``macro assembler,'' a more complicated
assembler that could recognize not only normal instructions, but also
special names ({}``macros''). For each name, the macro assembler
assembles the five or ten machine instructions represented by the
name, just as though the programmer had written them out in full.


\subsection{Abstraction}

A major advance was the invention of the {}``high-level language.\char`\"{}
Again this was a translator program, but a more powerful one. High-level
languages make it possible for programmers to write code like this:

\begin{verbatim}
X = Y (456/A) - 2
\end{verbatim}
\wepsfigx{img1-004}

which looks a lot like algebra. Thanks to high-level langages, engineers,
not just bizarre bit-jockeys, could start writing programs. BASIC
and FORTRAN are examples of high-level languages. High-level languages
are clearly more {}``powerful'' than assembly languages in the sense
that each instruction might compile dozens of machine instructions.
But more significantly, high-level languages eliminate the linear
correspondence between source code and the resulting machine instructions.
The actual instructions depend on each entire {}``statement'' of
source code taken as a whole. Operators such as + and = have no meaning
by themselves. They are merely part of a complex symbology that depends
upon syntax and the operator's location in the statement. This nonlinear,
syntax-dependent correspondence between source and object code is
widely considered to be an invaluable step in the progress of programming
methodology. But as we'll see, the approach ultimately offers more
restriction than freedom.


\subsection{Manageability}

Most computer programs involve much more than lists of instructions
to work down from start to finish. They also involve testing for various
conditions and then {}``branching'' to the appropriate parts of the
code depending upon the outcome. They also involve {}``looping''
over the same sections of code repeatedly, usually testing for the
moment to branch out of the loop. Both assembler and high-level languages
provide branching and looping capabilities. In assembly languages
you use {}``jump instructions''; in some high-level languages you
use {}``GO TO'' commands. When these capabilities are used in the
most brute-force way, programs tend to look like the jumble you see
in \Fig{fig1-1}. This approach, still widely used in languages like
FORTRAN and BASIC, suffers from being difficult to write and difficult
to change if corrections need to be made. In this {}``bowl-of-spaghetti''
school of programming, it's impossible to test a single part of the
code or to figure out how something is getting executed that isn't
supposed to be getting executed. Difficulties with spaghetti programs
led to the discovery of {}``flow charts.'' These were pen-and-ink
drawings representing the {}``flow'' of execution used by the programmer
as an aid to understanding the code being written. Unfortunately the
programmer had to make the translation from code to flow chart and
back by hand. Many programmers found old-fashioned flow charts less
than useful.


\subsection{Modularity}

A significant advance arose with the invention of {}``Structured
Programming,'' a methodology based on the observation that large problems
are more easily solved if treated as collections of smaller problems
{[}1{]}. Each piece is called a module. Programs consist of modules
within modules.


\wepsfigpp{fig1-1}{Unstructured code, using jumps or {}``GOTOs.''}


Structured programming eliminates spaghetti coding by insisting that
control flow can be diverted only within a module. You can't jump
out from the middle of one module into the middle of another module.
For example, \Fig{fig1-2} shows a structured diagram of a module to
{}``Make Breakfast,'' which consists of four submodules. Within each
submodule you'll find a whole new level of complexity which needn't
be shown at this level.

\wepsfiga{fig1-2}{Design for a structured program}
i branching decision occurs in this module to choose between the {}``cold
cereal'' module and the {}``eggs'' module, but control f1ow stays
within the outer module. Structured programming has three premises:

\begin{enumerate}
\item Every program is described as a linear sequence of self-contained
functions, called \emph{modules}. Each module has exactly one entry
point and one exit point.
\item Each module consists of one or more functions, each of which has exactly
one entry point and one exit point and can itself be described as
a module.
\item A module can contain:

\begin{enumerate}
\item operations or other modules
\item decision structures (IF THEN statements)
\item looping structures
\end{enumerate}
\end{enumerate}
The idea of modules having {}``one-entry, one-exit'' is that you
can unplug them, change their innards, and plug them back in, without
screwing up the connections with the rest of the program. This means
you can test each piece by itself. That's only possible if you know
exactly where you stand when you start the module, and where you stand
when you leave it.

In {}``Make Breakfast'' you'll either fix cereal or make eggs, not
both. And you'll always clean up. (Some programmers I know circumvent
this last module by renting a new apartment every three months.)

%
\wepsfiga{fig1-3}{Structured programming with a non-structured language}
Structured programming was originally conceived as a design approach.
Modules were imaginary entities that existed in the mind of the programmer
or designer, not actual units of source code. When structured programming
design techniques are applied to non-structured languages like BASIC,
the result looks something like \Fig{fig1-3}.


\subsection{Writeability}

Yet another breakthrough encouraged the use of structured programs:
structured programming languages. These languages include control
structures in their command sets, so you can write programs that have
a more modular appearance. Pascal is such a language, invented by
Niklaus Wirth to teach the principles of structured programming to
his students. \Fig{fig1-4} shows how this type of language would allow
{}``Make Breakfast'' to be written.

%
\wepsfiga{fig1-4}{Using a structured language.}
Structured programming languages include control structure operators
such as IF and THEN to ensure a modularity of control flow. As you
can see, indentation is important for readability, since all the instructions
within each module are still written out rather than being referred
to by name (e.g., ''i''). The finished program might take ten pages,
with the ELSE on page five.


\subsection{Designing from the Top}

How does one go about designing these modules? A methodology called
''top-down design'' proclaims that modules should be designed in order
starting with the most general, overall module and working down to
the nitty-gritty modules. Proponents of top-down design have witnessed
shameful wastes of time due to lack of planning, They've learned through
painful experience that trying to correct programs after they've been
written---a practice known as {}``patching''---is like locking the
barn door after the horse has bolted. So they offer as a countermeasure
this official rule of top-down programming:

\begin{quote}
Write no code until you have planned every last detail.
\end{quote}
Because programs are so difficult to change once they've been written,
any design oversight at the preliminary planning stage should be revealed
before the actual code-level modules are written, according to the
top-down design, Otherwise, man-years of effort may be wasted writing
code that cannot be used.


\subsection{Subroutines}

We've been discussing {}``modules'' as abstract entities only. But
all high-level programming languages incorporate techniques that allow
modules of design to be coded as modules of code---discrete units
that can be given names and ''invoked'' by other pieces of code. These
units are called subroutines, procedures, or functions, depending
on the particular high-level language and on how they happen to be
implemented. Suppose we write {}``MAKE-CEREAL'' as a subroutine.
It might look something like this:

\begin{verbatim}
procedure make-cereal
   get clean bowl
   open cereal box
   pour cereal
   open milk
   pour milk
   get spoon
end
\end{verbatim}
We can also write {}``MAKE-EGGS'' and {}``CLEANUP'' as subroutines.
Elsewhere we can define {}``MAKE-BREAKFAST'' as a simple routine
that invokes, or calls, these subroutines:

\begin{Verbatim}[commandchars=\&\{\}]
procedure make-breakfast
   var h: boolean (indicates hurried)
   &textit{test for hurried}
   if h = true then
      &poorbf{call make-cereal}
   else
      &poorbf{call make-eggs}
   end
   &poorbf{call cleanup}
end
\end{Verbatim}


\medskip

The phrase {}``call make-cereal'' causes the subroutine named {}``make-cereal''
to be executed. When the subroutine has finished being executed, control
returns back to the calling program at the point following the call.
Subroutines obey the rules of structured programming.

As you can see, the effect of the subroutine call is as if the sub-routine
code were written out in full within the calling module. But unlike
the code produced by the macro assembler, the subroutine can be compiled
elsewhere in memory and merely referenced. It doesn't necessarily
have to be compiled within the object code of the main program (\Fig{fig1-5}).

Over the years computer scientists have become more forceful in favoring
the use of many small subroutines over long-winded, continuous programs.
Subroutines can be written and tested independently. This makes it
easier to reuse parts of previously written programs, and easier to
assign different parts of a program to different programmers. Smaller
pieces of code are easier to think about and easier to verify for
correctness.

When subroutines are compiled in separate parts of memory and referred
to you can invoke the same subroutine many times throughout a program
without wasting space on repeated object code. Thus the judicious
use of subroutines can also decrease program size. 

Unfortunately, there's a penalty in execution speed when you use a
subroutine. One problem is the overhead in saving registers before
jumping to the subroutine and restoring them afterwards. Even more
time-consuming is the invisible but significant code needed to pass
parameters to and from the subroutine.

Subroutines are also fussy about how you invoke them and particularly
how you pass data to and from them. To test them independently you
need to write a special testing program to call them from.

For these reasons computer scientists recommend their use in moderation.
In practice subroutines are usually fairly large between a half page
to a full page of source code in length.


\subsection{Successive Refinement}

An approach that relies heavily on subroutines is called {}``Successive
Refinement'' {[}2{]}. The idea is that you begin by writing a skeletal
version of your program using natural names for procedures for data
structures. Then you write versions of each of the named procedures.
You continue this process to greater levels of detail until the procedures
can only be written in the computer language itself.

At each step the programmer must make decisions about the algorithms
being used and about the data stuctures they're being used on. Decisions
about the algorithms and associated data structures should be made
in parallel.

If an approach doesn't work out the programmer is encouraged to back
track as far as necessary and start again.

Notice this about successive refinement: You can't actually run any
part of the program until its lowest-level components are written.
Typically this means you can't test the program until after you've
completely designed it. 

Also notice: Successive refinement forces you to work out all details
of control structure on each level before proceeding to the next lower
level.


\subsection{Structured Design}

By the middle of late '70s, the computing industry had tried all the
concepts we've described, and it was still unhappy. The cost of maintaining
software---keeping it functional in the face of change---accounted
for more than half of the total cost of software, in some estimates
as much as ninety percent! 

Everyone agreed that these atrocities could usually be traced back
to incomplete analysis of the program, or poorly thought-out designs.
Not that there was anything wrong with structured programming \emph{per
se}. When projects came in late, incomplete, or incorrect, the designers
took the blame for not anticipating the unforeseen. 

Scholars naturally responded by placing more emphasis on design. {}``Next
time let's think things out better.''

About this time a new philosophy arose, described in an article called
``Structured Design'' {[}3{]}. One of its principles is stated in
this paragraph: 

\begin{quote}
Simplicity is the primary measurement recommended for evaluating alternative
designs relative to reduced debugging and modification time. Simplicity
can be enhanced by dividing the system into separate pieces in such
a way that pieces can be considered, implemented, fixed and changed
with minimal consideration or effect on the other pieces of the system.
\end{quote}
By dividing a problem into simple modules, programs were expected
to be easier to write, easier to change, and easier to understand.

But what is a module, and on what basis does one make the divisions?
{}``Structured Design'' outlines three factors for designing modules.


\subsection{Functional Strength}

One factor is something called {}``functional strength,'' which
is a measure of the uniformity of purpose of all the statements within
a module. If all the statements inside the module collectively can
be thought of as performing a single task, they are functionally bound.

You can generally tell whether the statements in a module are functionally
bound by asking the following questions. First, can you describe its
purpose in one sentence? If not, the module is probably not functionally
bound. Next, ask these four questions about the module:

\begin{enumerate}
\item Does the description have to be a compound sentence?
\item Does it use words involving time such as {}``first'', {}``next'',
{}``then,'' etc.?
\item Does it use a general or nonspecific object following the verb?
\item Does it use words like initialize which imply a lot of different functions
being done at the same time?
\end{enumerate}
If the answer to any of these four questions is {}``yes,'' you're
looking at some less cohesive type of binding than functional binding.
Weaker forms of binding include:

\begin{description}
\item [Coincidental~binding]the statements just happen to appear in the
same module
\item [Logical~binding]the module has several related functions and requires
a flag or parameter to decide which particular function to perform
\item [Temporal~binding]the module contains a group of statements that
happen at the same time, such as initialization but have no other
relationship
\item [Communicational~binding]the module contains a group of statements
that all refer to the same set of data
\item [Sequential~binding]where the output of one statement serves as
input for the next statement
\end{description}
Our {}``MAKE-CEREAL'' module exhibits functional binding, because
it can be thought of as doing one thing, even though it consists of
several subordinate tasks.


\subsection{Coupling}

A second tenet of structured design concerns {}``coupling,'' a measure
of how modules influence the behavior of other modules. Strong coupling
is considered bad form. The worst case is when one module actually
modifies code inside another module. Even passing control flags to
other modules with the intent to control their function is dangerous.

An acceptable form of coupling is {}``data coupling,'' which involves
passing data (not control information) from one module to another.
Even then, systems are easiest to build and maintain when the data
interfaces between modules are as simple as possible.

When data can be accessed by many modules (for instance, global variables),
there's stronger coupling between the modules. If a programmer needs
to change one module, there's a greater danger that the other modules
will exhibit {}``side effects.''

The safest kind of data coupling is the passing of local variables
as parameters from one module to another. The calling module says
to the subordinate module, in effect, {}``I want you to use the data
I've put in these variables named X and Y, and when you re done, I
expect you to have put the answer in the variable named Z. No one
else will use these variables.''

As we said, conventional languages that support subroutines include
elaborate methods of passing arguments from one module to another.


\subsection{Hierarchical lnput-Process-Output Designing}

A third precept of structured design concerns the design process.
Designers are advised to use a top-down approach, but to pay less
attention initially to control structures. {}``Decision designing''
can wait until the later, detailed design of modules. Instead, the
early design should focus on the program's hierarchy (which modules
call which modules) and to the passing of data from one module to
another.

To help designers think along these new lines, a graphic representation
was invented, called the {}``structure chart.'' (A slightly different
form is calledthe {}``HIPO chart,'' which stands for {}``hierarchical
input-process-output.'') Structure charts include two parts, a hierarchy
chart and an input-output chart.

\Fig{fig1-6} shows these two parts. The main program, called DOIT,
consists of three subordinate modules, which in turn invoke the other
modules shown below them. As you can see, the design emphasizes the
transformation of input to output. 

The tiny numbers ofthe hierarchy chart refer to the lines on the in-out
chart. At point 1 (the module READ), the output is the value A. At
point 2 (the module TRANSFORM-TO-B), the input is A, and the output
is B. 

Perhaps the greatest contribution of this approach is recognizing
that decisions about control flow should not dominate the emerging
design. As we' ll see, control flow is a superficial aspect of the
problem. Minor changes in the requirements can profoundly change the
program's control structures, and {}``deep-six'' years of work.
But if programs aredesigned around other concerns, such as the flow
of data, t hen a change in plan won' t have so disastrous an effect.


\subsection{lnformation-Hiding }

In a paper {[}4{]} published back in 1972, Dr. David L. Parnas showed
that the criteria for decomposing modules should not be steps in the
process, but rather pieces of information that might possibly change.
Modules should be used to hide such information.

Let's look at this important idea of ''information- hiding'': Suppose
you are writing a Procedures Manual for your company. Here s a portion:

\begin{quote}
Sales Dept. takes order\\
sends blue copy to Bookkeeping\\
orange copy to Shipping\\
Jay logs the orange copy in the red binder on his desk, and completes
packing slip.
\end{quote}
Everyone agrees that this procedure is correct, and your manual gets
distributed to everyone in the company. Then Jay quits, and Marilyn
takes over. The new duplicate forms have green and yellow sheets,
not blue and orange. The red binder fills up and gets replaced with
a black one. Your entire manual is obsolete. You could have avoided
the obsolescence by using the term {}``Shipping Clerk'' instead
of the name Jay, the terms {}``BookkeepingDept. copy and'' {}``ShippingDept.
copy'' instead of {}``blue'' and {}``orange,'' etc.

This example illustrates that in order to maintain correctness in
the face of a changing environment, arbitrary details should be excluded
from procedures. The details can be recorded elsewhere if necessary.
For instance, every week or so the personnel department might issue
a list of employees and their job titles, so anyone who needed to
know who the shipping clerk was could look it up in this single source.
As the personnel changes, this list would change.

This technique is very important in writing software. Why would a
program ever need to change, once it's running? For any of a million
reasons. You might want to run an old program on new equipment; the
program must be changed just enough to accommodate the new hardware.
The program might not be fast enough, or powerful enough, to suit
the people who are using it. Most software groups find themselves
writing {}``families'' of programs; that is, many versions of related
programs in their particular application field, each a variant on
an earlier program. 

To apply the principle of information-hiding to software, certain
details of the program should be confined to a single location, and
any useful piece of information should be expressed only once. Programs
that ignore this maxim are guilty of redundancy. While hardware redundancy
(backup computers, etc. ) can make a system more secure, redundancy
of information is dangerous.

As any knowledgeable programmer will tell you, a number that a program
uses and that might conceivably change should be made into a {}``constant''
and referred to throughout the program by name, not by value. For
instance, the number of columns representing the width of your computer
paper forms should be expressed as a constant. Even assembly languages
provide {}``EQU''s and labels for associating values such as addresses
and bit-patterns with names.

Any good programmer will also apply the concept of information-hiding
to the development of subroutines, ensuring that each module knows
as little as possible about the insides of other modules. Contemporary
programming languages such as C, Modula 2, and Edison apply this concept
to the architecture of their procedures. But Parnas takes the idea
much further. He suggests that the concept should be extended to algorithms
and data structures. In fact, hiding information---not decision-structure
or calling- hierarchy---should be the primary basis for design!


\section{The Superficiality of Structure}

Parnas proposes two criteria for decomposition:

\begin{enumerate}
\item possible (though currently unplanned) reuse, and
\item possible (though unplanned) change.
\end{enumerate}
This new view of a {}``module'' is different than the traditional
view. This {}``module'' is a collection of routines, usually very
small, which together hide information about some aspect of the problem. 

Two other writers describe the same idea in a different way, using
the term {}``data abstraction'' {[}5{]}. Their example is a push-down
stack. The stack {}``module'' consists of routines to initialize
the stack, push a value onto the stack, pop a value from the stack,
and determine whether the stack is empty. This {}``multiprocedure
module'' hides the information of how the stack is constructed from
the rest of the application. The procedures are considered to be a
single module because they are interdependent. You can't change the
method for pushing a value without also changing the method for popping
a value.

The word \emph{uses} plays an important role in this concept. Parnas
writes in a later paper {[}6{]}:

\begin{quote}
Systems that have achieved a certain {}``elegance''... have done
so by having parts of the system use other parts...

If such a hierarchical ordering exists then each level offers a testable
and usable subset of the system...

The design of the {}``uses'' hierarchy should be one of the major
milestones in a design effort. The division of the system into independently
callable subprograms has to go in parallel with the decisions about
uses, because they influence each other.
\end{quote}
A design in which modules are grouped according to control flow or
sequence will not readily allow design changes. Structure, in the
sense or control-flow hierarchy, is superficial.

A design in which modules are grouped according to things that may
change can readily accommodate change.


\section{Looking Back, and FORTH}

In this section we'll review the fundamental features of FORTH and
relate them to what we've seen about traditional methodologies. Here's
an example of FORTH code;

\begin{verbatim}
: BREAKFAST
   HURRIED?  IF  CEREAL  ELSE  EGGS  THEN CLEAN ;
\end{verbatim}
This is structurally identical to the procedure MAKE-BREAKFAST on
page 8. (If you're new to FORTH, refer to Appendix A for an explanation.)
The words HURRIED?, CEREAL, EGGS, and CLEAN are (most likely) also
defined. as colon definitions. Up to a point, FORTH exhibits all the
traits we've studied: mnemonic value, abstraction, power, structured
control operators, strong functional binding, limited coupling, and
modularity. But regarding modularity, we encounter what may be FORTH's
most significant breakthrough:

\begin{quote}
The smallest atom of a FORTH program is not a module or a subroutine
or a procedure, but a {}``word.''
\end{quote}
Furthermore, there are no subroutines, main programs, utilities, or
executives, each of which must be invoked differently. Everything
in FORTH is a word. Before we explore the significance of a word-based
environment, let's first study two FORTH inventions that make it poss;ble.


\subsection{Implicit Calls}

First, calls are implicit. You don't have to say CALL CEREAL, you
simply say CEREAL. In FORTH, the definition of CEREAL {}``knows''
what kind of word it is and what procedure to use to invoke itself.

Thus variables and constants, system functions, utilities, as well
as any user-defined commands or data structures can all be {}``called''
simply by name.


\subsection{Implicit Data Passing}

Second, data passing is implicit. The mechanism that produces this
effect is FORTH's data stack. FORTH automatically pushes numbers onto
the stack; words that require numbers as input automatically pop them
off the stack; words that produce numbers as output automatically
push them onto the stack. The words PUSH and POP do not exist in high-level
FORTH.

Thus we can write

\begin{verbatim}
: DOIT
    GETC  TRANSFORM-TO-DO  PUT-D ;
\end{verbatim}
confident that GETC will get {}``C'', and leave it on the stack.
TRANSFORM-TO-D will pick up {}``C'' from the stack, transform it,
and leave {}``D'' on the stack. Finally, PUT-D will pick up {}``D''
on the stack and write it. FORTH eliminates the act of passing data
from our code, leaving us to concentrate on the functional steps of
the data's transformation.

Because FORTH uses a stack for passing data, words can nest within
words. Any word can put numbers on the stack and take them off without
upsetting the f1ow of data between words at a higher level (provided,
of course, that the word doesn't consume or leave any unexpected values).
Thus the stack supports structured, modular programming while providing
a simple mechanism for passing local arguments.

FORTH eliminates from our programs the details of \emph{how} words
are invoked and \emph{how} data are passed. What's left? Only the
words that describe our problem.

Having words, we can fully exploit the recommendations of Parnas---to
decompose problems according to things that may change, and have each
{}``module'' consist of many small functions, as many as are needed
to hide information about that module. In FORTH we can write as many
words as we need to do that, no matter how simple each of them may
be.

A line from a typical FORTH application might read:

\begin{verbatim}
20 ROTATE LEFT TURRET
\end{verbatim}
Few other languages would encourage you to concoct a subroutine called
LEFT, merely as a modifier, or a subroutine called TURRET, merely
to name part of the hardware. Since a FORTH word is easier to invoke
than a subroutine (simply by being named, not by being called), a
FORTH program is likely to be decomposed into more words than a conventional
program would be into subroutines.


\section{Component Programming}

Having a larger set of simpler words makes it easy to use a technique
we'll call ``component programming.'' To explain, let's first reexamine
these collections we have vaguely described as ``things that may change.''
In a typical system, just about everything is subject to change: I/O
devices such as terminals and printers, interfaces such as UART chips,
the operating system, any data structure or data representation, any
algorithm, etc.

The question is: ``How can we minimize the impact of any such change?
What is the smallest set of other things that must change along with
such a change?''

The answer is: ``The smallest set of interacting data structures and
algorithms that share knowledge about how they collectively work.''
We'll call this unit a ``component.''

A component is a resource. It may be a piece of hardware such as a
UART or a hardware stack. Or the component may be a software resource
such as a queue, a dictionary, or a software stack.

All components involve data objects and algorithms. It doesn't matter
whether the data object is physical (such as a hardware register),
or abstract (such as a stack location or a field in a data base).
It doesn't matter whether the algorithm is described in machine code
or in problem- oriented words such as CEREAL and EGGS.

\Fig{fig1-7} contrasts the results of structured design with the results
of designing by components. Instead of \emph{modules} called READ--RECORD,
EDIT--RECORD, and WRITE--RECORD, we're con- cerned with components
that describe the structure of records, provide a set of editor commands,
and provide read/write routines to storage.

What have we done? We've inserted a new stage in the development process:
We decomposed by components in our \emph{design}, then we described
the sequence, hierarchy, and input-process-output in our \emph{implementation}.
Yes, it's an extra step, but we now have an extra dimension for decomposition---not
just slicing but \emph{dicing}.

Suppose that, after the program is written, we need to change the
record structure. In the sequential, hierarchical design, this change
would affect all three modules. In the design by components, the change
would be confined to the record-structure component. No code that
uses this component needs to know of the change.

Aside from maintenance, an advantage to this scheme is that programmers
on a team can be assigned components individually, with less interdependence.
The principle of component programming applies to team management
as well as to software design.

We'll call the set of words which describe a component a {}``lexicon.''
(One meaning of lexicon is {}``a set of words pertaining to a particular
field of interest.'') The lexicon is your interface with the component
from the outside (\Fig{fig1-8}).

In this book, the term {}``lexicon'' refers only to those words
of a component that are used by name outside of a component. A component
may also contain definitions written solely to support the externally
visible lexicon. We'll call the supporting definitions {}``internal''
words.

The lexcon provides the logical equivalents to the data objects and
algorithms in the form of names. The lexicon veils the compnent's
data structures and algorithms---the {}``how it works.'' It presents
to the world only a {}``conceptual model'' of the component described
in simple words---the {}``what it does.''

These words then become the language for describing the data structures
and algorithms of components written at a a higher level. The {}``what''
of one component becomes the {}``how'' of a higher component.

Written in FORTH, an entire application consists of nothing but components.
\Fig{fig1-9} show show a robotics application might be decomposed.

You could even say that each lexicon is a special-purpose compiler,
written solely for the purpose of supporting higher-level application
code in the most efficient and reliable way.

By the way, FORTH itself doesn't support components. It doesn't need
to. Components are the product of the program designer's decomposition.
(FORTH does have {}``screens.'' however---small units of mass storage
for saving source code. A component can usually be written in one
or two screens of FORTH.)

It's important to understand that a lexicon can be used by any and
all of the components at higher levels. Each successive component
does \emph{not} bury its supporting components, as is often the case
with layered approaches to design. Instead, each lexicon is free to
use all of the commands beneath it. The robot-movement command relies
on the root language, with its variables, constants, stack operators,
math operators, and so on, as heavily as any other component.

An important result of this approach is that the entire application
employs a single syntax, which makes it easy to learn and maintain.
This is why I use the term {}``lexicon'' and not {}``language.''
Languages have unique syntaxes.

This availability of commands also makes the process of testing and
debugging a whole lot easier. Because FORTH is interactive, the programmer
can type and test the primitive commands, such as

\begin{verbatim}
RIGHT SHOULDER 20 PIVOT
\end{verbatim}
from the {}``outside'' as easily as the more powerful ones like

\begin{verbatim}
LIFT COFFEE-POT
\end{verbatim}
At the same time, the programmer can (if he or she wants) deliberately
seal any commands, including FORTH itself, from being accessed by
the end user, once the application is complete.

Now FORTH's methodology becomes clear. FORTH programming consists
of extending the root language toward the application, providing new
commands that can be used to describe the problem at hand.

Programming languages designed especially for particular applications
such as robotics, inventory control, statistics, etc., are known as
{}``application-oriented languages.'' FORTH is a programming environment
for \emph{creating} application-oriented languages. (That last sentence
may be the most succinct description of FORTH that you'll find.)

In fact, you shouldn't write any serious application in FORTH; as
a language it's simply not powerful enough. What you \emph{should}
do is write your own language in FORTH (lexicons) to model your understanding
of the problem, in which you can elegantly describe its solution.


\section{Hide From Whom?}

Because modern mainstream languages give a slightly different meaning
to the phrase {}``information-hiding,'' we should clarify. From
what, or whom are we hiding information?

The newest traditional languages (such as Modula 2) bend over backwards
to ensure that modules hide internal routines and data structures
from other modules. The goal is to achieve module independence (a
minimum coupling). The fear seems to be that modules strive to attack
each other like alien antibodies. Or else, that evil bands of marauding
modules are out to clobber the precious family data structures.

This is \emph{not} what we're concerned about. The purpose of hiding
information, as we mean it, is simply to minimize the effects of a
possible design-change by localizing things that might change within
each component.

FORTH programmers generally prefer to keep the program under their
own control and not to employ any techniques to physically hide data
structures. (Nevertheless a brilliantly simple technique for adding
Modula-type modules to FORTH has been implemented, in only three lines
of code, by Dewey Val Shorre {[}7{]}.)


\section{Hiding the Construction of Data Structures}

We've noted two inventions of FORTH that make possible the methodology
we've described---implicit calls and implicit data passing. A third
feature allows the data structures within a component to be described
in terms of previously-defined components. This feature is direct
access memory.

Suppose we define a variable called APPLES, like this:

\begin{verbatim}
VARIABLE APPLES
\end{verbatim}
We can store a number into this variable to indicate how many apples
we currently have:

\begin{verbatim}
20 APPLES !
\end{verbatim}
We can display the contents of the variable:

\begin{verbatim}
APPLES ? 20 ok
\end{verbatim}
We can up the count by one:

\begin{verbatim}
1 APPLES +!
\end{verbatim}
(The newcomer can study the mechanics of these phrases in Appendix
A.)

The word APPLES has but one function: to put on the stack the \emph{address}
of the memory location where the tally of apples is kept. The tally
can be thought of as a {}``thing,'' while the words that set the
tally, read the tally, or increment the tally can be considered as
{}``actions.''

FORTH conveniently separates {}``things'' from {}``actions'' by
allowing addresses of data structures to be passed on the stack and
providing the {}``fetch'' and {}``store'' commands.

We've discussed the importance of designing around things that may
change. Suppose we've written a lot of code using this variable APPLES.
And now, at the eleventh hour, we discover that we must keep track
of two different kinds of apples, red and green!

We needn't wring our hands, but rather remember the function of APPLES:
to provide an address. If we need two separate tallies, APPLES can
supply two different addresses depending on which kind of apple we're
currently talking about. So we define a more complicated version of
APPLES as follows:

\begin{verbatim}
VARIABLE COLOR  ( pointer to current tally)
VARIABLE REDS  ( tally of red apples)
VARIABLE GREENS  ( tally of green apples)
: RED  ( set apples-type to RED)  REDS COLOR ! ;
: GREEN  ( set apple-type to GREEN)  GREENS COLOR ! ;
: APPLES  ( -- adr of current apple tally)  COLOR @ ;
\end{verbatim}
Here we've redefined APPLES. Now it fetches the contents of a variable
called COLOR. COLOR is a pointer, either to the variable REDS or to
the variable GREENS. These two variables are the real tallies.

If we first say RED, then we can use APPLES to refer to red apples.
If we say GREEN, we can use it to refer to green apples (\Fig{fig1-10}).

We didn't need to change the syntax of any existing code that uses
APPLES. We can still say

\begin{verbatim}
20 APPLES !
\end{verbatim}
and

\begin{verbatim}
1 APPLES +!
\end{verbatim}
Look again at what we did. We changed the definition of APPLES from
that of a variable to a colon definition, without affecting its usage.
FORTH allows us to hide the details fo how APPLES is defined from
the code that uses it. What appears to be {}``thing'' (a variable)
to the original code is actually defined as an {}``action'' (a colon
definition) within th component.

FORTH encourages the use of abstract data types by allowing data structures
to be defined in terms of lower level components. Only FORTH, which
eliminates the CALLs from procedures, which allows addresses and data
to be implicitly passed via the stack, and which provides direct access
to memory locations with @ and !, can offer this level of information-hiding.

FORTH pays little attention to whether something is a data structure
or an algorithm. This indifference allows us programmers incredible
freedom in creating the parts of speech we need to describe our applications.

I tend to think of any word which returns an address, such as APPLES,
as a {}``noun,'' regardless of how it's defined. A word that performs
an obvious action is a {}``verb.''

Words such as RED and GREEN in our example can only be called {}``adjectives''
since they modify the function of APPLES. The phrase

\begin{verbatim}
RED APPLES ?
\end{verbatim}
is different from

\begin{verbatim}
GREEN APPLES ?
\end{verbatim}
FORTH words can also serve as adverbs and prepositions. There's little
value in trying to determine what part of speech a particular word
is, since FORTH doesn't care anyway. We need only enjoy the ease of
describing an application in natural terms.


\section{But Is It a High-Level Language?}

In our brief technical overview, we noted that traditional high-level
languages broke away from assembly-language by eliminating not only
the \emph{one-for-one} correspondence between commands and machine
operations, but also the \emph{linear} correspondence. Clearly FORTH
lays claim to the first difference; but regarding the second, the
order of words that you use in a definition is the order in which
those commands are compiled. Does this disqualify FORTH from the ranks
of high-level languages? Before we answer, let's explore the advantages
of the FORTH approach.

Here's what Charles Moore, the inventor of FORTH, has to say:

\begin{quote}
You define each word so that the computer knows what it means. The
way it knows is that it executes some code as a consequence of being
invoked. The computer takes an action on every word. It doesn't store
the word away and keep it in mind for later.

In a philosophical sense I think this means that the computer {}``understands''
a word. It understands the word \textbf{DUP}, perhaps more profoundly
than you do, because there's never any question in its mind what \textbf{DUP}
means.

The connection between words that have meaning to you and words that
have meaning to the computer is a profound one. The computer becomes
the vehicle for communication between human being and concept.
\end{quote}
One advantage of the correspondence between source code and machine
execution is the tremendous simplification of the compiler and interpreter.
This simplification improves performance in several ways, as we'll
see in a later section.

From the standpoint of programming methodology, the advantage to the
FORTH approach is that \emph{new} words and \emph{new} syntaxes can
easily be added. FORTH cannot be said to be {}``looking'' for words---it
finds words and executes them. If you add new words FORTH will find
and execute them as well. There's no difference between existing words
and words that you add.

What's more, this {}``extensibility'' applies to all types of words,
not just action-type functions. For instance, FORTH allows you to
add new \emph{compiling} words---like \textbf{IF} and \textbf{THEN}
that provide structured control flow. You can easily add a case statement
or a multiple-exit loop if you need them, or, just as importantly,
take them out if you don't need them.

By contrast, any language that depends on word order to understand
a statement must {}``know'' all legal words and all legal combinations.
Its chances of including all the constructs you'd like are slim. The
language exists as determined by its manufacturer; you can't extend
its knowledge.

Laboratory researchers cite flexibility and extensibility as among
FORTH's most important benefits in their environment. Lexicons can
be developed to hide information about the variety of test equipment
attached to the computer. Once this work has been done by a more experienced
programmer, the researchers are free to use their {}``software toolbox''
of small words to write simple programs for experimentation. As new
equipment appears, new lexicons are added.

Mark Bernstein has described the problem of using an off-the-shelf
special-purpose procedure library in the laboratory {[}8{]}: {}``The
computer, not the user, dominates the experiment.'' But with FORTH,
he writes, {}``the computer actually encourages scientists to modify,
repair, and improve the software, to experiment with and characterize
their equipment. Initiative becomes once more the prerogative of the
researcher.''

For those purists who believe FORTH isn't fit to be called a high-level
language, FORTH makes matters even worse. While strong syntax checking
and data typing are becoming one of the major thrusts of contemporary
programming languages, FORTH does almost no syntax checking at all.
In order to provide the kind of freedom and flexibility we have described,
it cannot tell you that you meant to type RED APPLES instead of APPLES
RED. You have just invented syntax!

Yet FORTH more than makes up for its omission by letting you compile
each definition, one at a time, with turnaround on the order of seconds.
You discover your mistake soon enough when the definition doesn't
work. In addition, you can add appropriate syntax checking in your
definitions if you want to.

An artist's paintbrush doesn't notify the artist of a mistake; the
painter will be the judge of that. The chef's skillet and the composer's
piano remain simple and yielding. Why let a programming language try
to outthink you?

So is FORTH a high-level language? On the question of syntax checking,
it strikes out. On the question of abstraction and power, it seems
to be of \emph{infinite} level---supporting everything from bit manipulation
at an output port to business applications.

You decide. (FORTH doesn't care.)


\section{The Language of Design}

FORTH is a design language. To the student of traditional computer
science, this statement is self-contradictory. {}``One doesn't design
with a language, one implements with a language. Design precedes implementation.''

Experienced FORTH programmers disagree. In FORTH you can write abstract,
design-level code and still be able to test it at any time by taking
advantage of decomposition into lexicons. A component can easily be
rewritten, as development proceeds, underneath any components that
use it. At first the words in a component may print numbers on your
terminal instead of controlling stepper motors. They may print their
own names just to let you know they've executed. They may do nothing
at all.

Using this philosophy you can write a simple by testable version of
your application, then successively change and refine it until you
reach your goal.

Another factor that makes designing in code possible is that FORTH,
like some of the newer languages, eliminates the {}``batch-compile''
development sequence (edit-compile-test-edit-compile-test). Because
the feedback is instantaneous, the medium becomes a partner in the
creative process. The programmer using a batch-compiler language can
seldom achieve the productive state of mind that artists achieve when
the creative current flows unhindered.

For these reasons, FORTH programmers spend less time planning than
their classical counterparts, who feel righteous about planning. To
them, not planning seems reckless and irresponsible. Traditional environments
force programmers to plan because traditional programming languages
do not readily accommodate change.

Unfortunately, human foresight is limited even under the best conditions.
Too much planning becomes counterproductive.

Of course FORTH doesn't eliminate planning. It allows prototyping.
Constructing a prototype is a more refined way to plan, just as breadboarding
is in electronic design.

As we'll see in the next chapter, experimentation proves more reliable
in arriving at the truth than the guesswork of planning.


\section{The Language of Performance}

Although performance is not the main topic of this book, the newcomer
to FORTH should be reassured that its advantages aren't purely philosophical.
Overall, FORTH outdoes all other high-level languages in speed, capability
and compactness.


\subsection{Speed}

Although FORTH is an interpretive language, it executes compiled code.
Therefore it runs about ten times faster than interpretive BASIC.

FORTH is optimized for the execution of words by means of a technique
known as {}``threaded code'' {[}9{]}, {[}10{]}, {[}11{]}. The penalty
for modularizing into very small pieces of code is relatively slight.

It does not run as fast as assembler code because the inner interpreter
(which interprets the list of addresses that comprise each colon definition)
may consume up to 50\% of the run time of primitive words, depending
on the processor.

But in large applications, FORTH comes very close to the speed of
assembler. Here are three reasons:

First and foremost, FORTH is simple. FORTH's use of a data stack greatly
reduces the performance cost of passing arguments from word to word.
In most languages, passing arguments between modules is one of the
main reasons that the use of subroutines inhibits performance.

Second, FORTH allows you to define words either in high-level or in
machine language. Either way, no special calling sequence is needed.
You can write a new definition in high level and, having verified
that it is correct, rewrite it in assembler without changing any of
the code that uses it. In a typical application, perhaps 20\% of the
code will be running 80\% of the time. Only the most often used, time-critical
routines need to be machine coded. The FORTH system itself is largely
implemented in machine-code definitions, so you'll have few application
words that need to be coded in assembler.

Third, FORTH applications tend to be better designed than those written
entirely in assembler. FORTH programmers take advantage of the language's
prototyping capabilities and try out several algorithms before settling
on the one best suited for their needs. Because FORTH encourages change,
it can also be called the language of optimization.

FORTH doesn't guarantee fast applications. It does give the programmer
a creative environment in which to design fast applications.


\subsection{Capability}

FORTH can do anything any other language can do---usually easier.

At the low end, nearly all FORTH systems include assemblers. These
support control-structure operators for writing conditionals and loops
using structured programming techniques. They usually allow you to
write interrupts---you can even write interrupt code in high level
if desired.

Some FORTH systems are multitasked, allowing you to add as many foreground
or background tasks as you want.

FORTH can be written to run on top of any operating system such as
RT-11, CP/M, or MS-DOS---or, for those who prefer it, FORTH can be
written as a self-sufficient operating system including its own terminal
drivers and disk drivers.

With a FORTH cross-compiler or target compiler, you can use FORTH
to recreate new FORTH systems, for the same computer or for different
computers. Since FORTH is written in FORTH, you have the otherwise
unthinkable opportunity to rewrite the operating system according
to the needs of your application. Or you can transport streamlined
versions of applications over to embedded systems. 


\subsection{Size}

There are two considerations here: the size of the root FORTH system,
and the size of compiled FORTH applications.

The FORTH nucleus is very flexible. In an embedded application, the
part of FORTH you need to run your application can fit in as little
as 1K. In a full development environment, a multitasked FORTH system
including interpreter, compiler, assembler, editor, operating system,
and all other support utilities averages 16K. This leaves plenty of
room for applications. (And some FORTHs on the newer processors handle
32-bit addressing, allowing unimaginably large programs.)

Similarly, FORTH compiled applications tend to be very small---usually
smaller than equivalent assembly language programs. The reason, again,
is threaded code. Each reference to a previously defined word, no
matter how powerful, uses only two bytes.

One of the most exciting new territories for FORTH is the production
of FORTH chips such as the Rockwell R65F11 FORTH-based microprocessor
{[}12{]}. The chip includes not only hardware features but also the
run-time portions of the FORTH language and operating system for dedicated
applications. Only FORTH's architecture and compactness make FORTH-based
micros possible.


\section{Summary}

FORTH has often been characterized as offbeat, totally unlike any
other popular language in structure or in philosophy, On the contrary,
FORTH incorporates many principles now boasted by the most contemporary
languages. Structured design, modularity, and information-hiding are
among the buzzwords of the day.

Some newer languages approach even closer to the spirit of FORTH.
The language C, for instance, lets the programmer define new functions
either in C or in assembly language, as does FORTH. And as with FORTH,
most of C is defined in terms of functions.

But FORTH extends the concepts of modularity and information-hiding
further than any other contemporary language. FORTH even hides the
manner in which words are invoked and the way local arguments are
passed.

The resulting code becomes a concentrated interplay of words, the
purest expression of abstract thought. As a result, FORTH programmers
tend to be more productive and to write tighter, more efficient, and
better maintainable code.

FORTH may not be the ultimate language. But I believe the ultimate
language, if such a thing is possible, will more closely resemble
FORTH than any other contemporary language.


\section{References}

\begin{enumerate}
\item O.J. Dahl, E.W. Dijkstra, and C.A.R. Hoare, Structured Progromming,
London, Academic Press, 1972.
\item Niklaus Wirth, {}``Program Development by Stepwise Refinement,\char`\"{}
Com- munications of ACM, 14, No. 4 (1971), 221-27.
\item W.P. Stevens, G.J. Myers, and L.L. Constantine, {}``Structured Design,\char`\"{}
IBM Systems Journal, Vol. 13, No. 2, 1974.
\item David L. Parnas, {}``On the Criteria To Be Used in Decomposing Systems
into Modules,\char`\"{} Communications of the ACM, December 1972.
\item Barbara H. Liskov and Stephen N. Zilles, {}``Specification Techniques
for Data Abstractions,\char`\"{} IEEE Transactions on Software Engineering,
March 1975.
\item David L. Parnas, {}``Designing Software for Ease of Extension and
Contraction,\char`\"{} IEEE Transactions on Software Engineering,
March 1979.
\item Dewey Val Shorre, {}``Adding Modules to FORTH,\char`\"{} 1980 FORML
Proceedings, p. 71.
\end{enumerate}

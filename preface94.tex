\chapter*{Preface to the 1994 Edition}
\addcontentsline{toc}{section}{Preface to the 1994 Edition}
\markboth{Preface to the 1994 Edition}{Preface to the 1994 Edition}
\pagestyle{headings}

\initial I'm honored that the \Forth{} Interest Group is reprinting
\emph{Thinking \Forth{}.} It's gratifying to know that the book may
have value to appreciators and users of \Forth{}.

This edition is a reproduction of the original edition, with only
minor typographical corrections. A lot has happened in the ten years
since the book's original publication, rendering some of the opinions
obsolete, or at best archaic. A ``revised, updated edition'' would
have entrailed a rewrite of many sections, a larger effort than I'm
able to make at the time.

Off all the opinions in the book, the one that I most regret seeing in
print is my criticism of object-oriented programming. Since penning
this book, I've had the pleasure of writing an application in a
version of \Forth{} with support for object-oriented programming,
developed by Digalog Corp. of Ventura, California. I'm no expert, but
it's clear that the methodology has much to offer.

With all this, I believe that many of the ideas in \emph{Thinking
\Forth{}} are as valid today as they were back then. Certainly \person{Charles
Moore}'s comments remain a telling insight on the philosophy that
triggered the development of \Forth{}.

I with to thank \person{Marlin Ouverson} for his excellent job, patiently
struggling against incompatible file formats and OCR errors, to bring
this reprint to life.

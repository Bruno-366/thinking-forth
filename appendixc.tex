%% Thinking Forth
%% Copyright (C) 2004 Leo Brodie
%% Initial transcription by Ed Beroset
%% 
%% Chapter: Appendix C, Other Utilities Described in This Book
\Chapmark{C}
\chapter{Other~Utilities
Described
in~This~Book}
This appendix is here to help you define some of the words referred to in 
this book that may not exist in your system.  Definitions are given in
\Forth{}-83 Standard.
\section{From \Chap{4}}
A definition of {\bf ASCII}\index{A!ASCII} that will work in '83 Standard is:
\begin{verbatim}
    : ASCII  ( -- c)  \  Compile:  c  ( -- )
    \ Interpret:   c   ( -- c)
         BL WORD 1+ C@  STATE @
         IF [COMPILE] LITERAL  THEN ; IMMEDIATE
\end{verbatim}
\section{From \Chap{5}}
The word {\bf $\backslash$}\index{S!Skip commands} can be defined as:
\begin{verbatim}
    : \  ( skip rest of line)
         >IN @  64 / 1+  64 *  >IN ! ; IMMEDIATE
\end{verbatim}
If you decide not to use {\bf EXIT} to terminate a screen, you can define {\bf \S}
as:
\begin{verbatim}
    : \S   1024 >IN ! ;
\end{verbatim}
The word {\bf FH} can be defined simply as:
\begin{verbatim}
    : FH   \   ( offset -- offset-block)   "from here"
        BLK @ + ;
\end{verbatim}
This factoring allows you to use FH in many ways, e.g.:
\begin{verbatim}
    : TEST   [ 1 FH ] LITERAL LOAD ;
\end{verbatim}
or
\begin{verbatim}
    : SEE   [ 2 FH ] LITERAL LIST ;
\end{verbatim}
A slightly more complicated version of FH also lets you edit or load a 
screen with a phrase such as ``14 FH LIST'', relative to the screen that 
you just listed (SCR):
\begin{verbatim}
    : FH   \   ( offset -- offset-block)   "from here"
         BLK @  ?DUP 0= IF  SCR @  THEN  + ;
\end{verbatim}
{\bf BL}\index{B!Blank space(BL)} is a simple constant:
\begin{verbatim}
    32 CONSTANT BL
\end{verbatim}
{\bf TRUE}\index{T!TRUE} and {\bf FALSE}\index{F!FALSE} can be defined as:
\begin{verbatim}
    0 CONSTANT FALSE
    -1 CONSTANT TRUE
\end{verbatim}
(\Forth{}'s control words such as IF and UNTIL interpret zero as ``false'' 
and any non-zero value as ``true.''  Before \Forth{} '83, the convention
was to indicate ``true'' with the value 1.  Starting with \Forth{} '83,
however, ``true'' is indicated with hex FF, which is the signed number --1
(all bits set).

{\bf WITHIN}\index{W!WITHIN} can be defined in high level like this:
\begin{Code}
    : WITHIN  ( n lo hi+1 -- ?)
         >R  1- OVER <  SWAP R>  < AND ;
\end{Code}

\section{From \Chap{8}}

The implementation of {\bf LEAP}\index{L!LEAP} will depend on how your system 
implement {\bf DO LOOP}s.  If {\bf DO} keeps two items on the return stack (the
index and the limit), {\bf LEAP} must drop both of them plus one more
return-stack item to exit:
\begin{verbatim}
    : LEAP   R> R> 2DROP  R> DROP ;
\end{verbatim}
If {\bf DO} keeps \emph{three} items on the return stack, it must be defined: 
\begin{verbatim}
    : LEAP   R> R> 2DROP  R> R> 2DROP ;
\end{verbatim}


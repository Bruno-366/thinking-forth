%% Back page

\noindent
Business, industry, and education are discovering that \Forth{} is an
especially effective language for producing compact, efficient
applications for real-time, real-world tasks. And now there's
\textbf{Thinking \Forth}---an instructive guide that illustrates the
\emph{elegant logic} behind the language and shows you how to apply
specific problem-solving tools to software, regardless of your
programming environment.

It combines the philosophy behind \Forth{} with traditional,
disciplined approaches to software development---to give you a basis
for writing more readable, easier-to-write, and easier-to-maintain
software applications in any language.

Written in the same lucid, humorous style as the author's
\emph{Starting \Forth} and packed with detailed coding samples and
illustrative cartoons, \textbf{Thinking \Forth} reviews fundamental
\Forth{} concepts and takes you from the initial specification of your
software project through the analysis and implementation process,
showing how to simplify your program and still keep it flexible
throughout. Both beginning and experienced programmers will gain a
better understanding and mastery of such topics as
\begin{itemize}
\item \Forth{} style and conventions
\item decomposition
\item factoring
\item handling data
\item simplifying control structures
\item and more.
\end{itemize}

And, to give you an idea of how these concepts can be applied,
\textbf{Thinking \Forth} contains revealing interviews with real-life
users and with \Forth's creator, \person{Charles H. Moore}.

To program intelligently, you must first \emph{think} intelligently,
and that's where \textbf{Thinking \Forth} comes in.

\textbf{\person{Leo Brodie}} is a writer, programmer, consultant,
teacher, and world-renowned authority on \Forth. He was a technical
writer for \Forth{} Inc., and has been an independent consultant
(since 1982) for such clients as IBM, NCR, and Lockheed. He is also
the author of \emph{Starting \Forth} (Prentice-Hall, 1981).

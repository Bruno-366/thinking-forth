%% Thinking Forth
%% Copyright (C) 2004 Leo Brodie
%% Initial transcription by <Josef Gabriel>
%% 
%% Chapter: <6>

\chapter{Factoring}\Chapmark{6}

In this chapter we'll continue our study of the implementations phase,
this time focusing on factoring.

Decomposition and factoring are chips off the same block. Both involve
dividing and organizing. Decomposition occurs during preliminary
design; factoring occurs during detailed design and implementation.

Since every colon definition reflects decisions of factoring, an
understanding of good factoring technique is perhaps the most
important skill for a \Forth{} programmer.

What is factoring? Factoring means organizing code into useful
fragments. To make a fragment useful, you often must separate reusable
parts from non-reusable parts. The reusable parts become new
definitions. The non-reusable parts become arguments or parameters to the
definitions.

Making this separation is usually referred to as ``factoring out.''
The first part of this chapter will discuss various ``factoring-out''
techniques.

Deciding how much should go into, or stay out of, a definition is
another aspect of factoring. The second section will outline the
criteria for useful factoring.

\section{Factoring Techniques}

\begin{tfquot}
If a module seems almost, but not quite, useful form a second place in
the system, try to identify and isolate the useful subfunction. The
remainder of the module might be incorporated in its original caller
(from ``\emph{Structured Design}'' \cite{stevens74-6}).
\end{tfquot}
The ``useful subfunction'' of course becomes the newly factored
definition.What about the part that ``isn't quite useful''? That
depends on what it is.

\subsection{Factoring Out Data}
\index{D!Data, factoring out|(}
The simplest thing to factor out is data thanks to \Forth{}'s data
stack. For instance, to compute two-thirds of 1,000, we write

\begin{Code}
1000 2 3 */
\end{Code}
To define a word that computes two-thirds of \emph{any} number,
we factor out the argument from the definition:

\begin{Code}
: TWO-THIRDS  ( n1 -- n2)  2 3 */ ;
\end{Code}
When the datum comes in the \emph{middle} of the useful phrase, we
have to use stack manipulation. For instance, to center a piece of
text ten characters long on an 80-column screen, we would write:

\begin{Code}
80  10 -   2/ SPACES
\end{Code}
But text isn't always 10 characters long. To make the phrase useful
for any string, you'd factor out the length by writing:

\begin{Code}
: CENTER (length -- ) 80  SWAP -  2/ SPACES ;
\end{Code}
The data stack can also be used to pass addresses. Therefore what's
factored out may be a \emph{pointer} to data rather than the data
themselves. The data can be numbers or even strings, and still be
factored out through use of the stack.

Sometimes the difference appears to be a function, but you can factor
it out simply as a number on the stack. For instance:

\begin{description}
\item[Segment 1:] \forth{WILLY NILLY\ \ PUDDIN' PIE AND}
\item[Segment 2:] \forth{WILLY NILLY\ \ 8 *\ \ PUDDIN' PIE AND}
\end{description}

\noindent How can you factor out the ``\forth{8 *}'' operation?
By including ``\forth{*}'' in the factoring and passing it a one or eight:

\begin{Code}
: NEW  ( n )  WILLY NILLY  *  PUDDIN' PIE AND ;
\end{Code}

\begin{description}
\item[Segment 1:] \forth{1 NEW}
\item[Segment 2:] \forth{8 NEW}
\end{description}

\noindent 
(Of course if \forth{WILLY NILLY} changes the stack, you'll need to
add appropriate stack-manipulation operators.)

If the operation involves addition, you can nullify it by passing a
zero.

\begin{tip}
For simplicity, try to express the difference between similar
fragments as a numeric difference (values or addresses), rather than
as a procedural difference.
\end{tip}
\index{D!Data, factoring out|)}

\subsection{Factoring Out Functions}
On the other hand, the difference sometimes \emph{is} a function. Witness:

%ARN - this gets messed up Code doesn't indent properly
\begin{description}
\item[Segment 1:]~~
\begin{minipage}[t]{.6\hsize}
\begin{Code}[commandchars=\&\{\}]
BLETCH-A  BLETCH-B   &poorbf{BLETCH-C}
         BLETCH-D  BLETCH-E  BLETCH-F
\end{Code}
\end{minipage}
\item[Segment 2:]~~
\begin{minipage}[t]{.6\hsize}
\begin{Code}[commandchars=\&\{\}]
BLETCH-A  BLETCH-B  &poorbf{PERVERSITY}
         BLETCH-D  BLETCH-E  BLETCH-F
\end{Code}
\end{minipage}
\end{description}

\noindent Wrong approach:

\begin{Code}[commandchars=\&\{\}]
: BLETCHES  ( t=do-BLETCH-C | f=do-PERVERSITY -- ) 
   BLETCH-A  BLETCH-B  IF  &poorbf{BLETCH-C}  ELSE  &poorbf{PERVERSITY}
      THEN  BLETCH-D BLETCH-E BLETCH-F ;
\end{Code}

\begin{description}
\item[Segment 1:]~~ \forth{TRUE BLETCHES}
\item[Segment 2:]~~ \forth{FALSE BLETCHES}
\end{description}

\noindent A better approach:

\begin{Code}
: BLETCH-AB   BLETCH-A BLETCH-B ;
: BLETCH-DEF   BLETCH-D BLETCH-E BLETCH-F ;
\end{Code}

\begin{description}
\item[Segment 1:]~~ \forth{BLETCH-AB ~}\forthb{BLETCH-C}\forth{~ BLETCH-DEF}
\item[Segment 2:]~~ \forth{BLETCH-AB ~}\forthb{PERVERSITY}\forth{~ BLETCH-DEF}
\end{description}

\index{C!Control flags|(}
\index{F!Flags|(}
\begin{tip}
Don't pass control flags downward.
\end{tip}
Why not? First, you are asking your running application to make a
pointless decision-one you knew the answer to while
programming-thereby reducing efficiency. Second, the terminology
doesn't match the conceptual model. What are \forth{TRUE BLETCHES}
as opposed to \forth{FALSE BLETCHES}?
\index{C!Control flags|)}
\index{F!Flags|)}

\subsection{Factoring Out Code from Within Control Structures}
\index{C!Code:!factoring out from within control structures|(}
\index{C!Control structures:!factoring out|(}
\index{C!Control structures:!factoring out code from within|(}

Be alert to repetitions on either side of an \forthb{IF }\forthb{ELSE }\forthb{THEN}
statement. For instance:

\index{E!EMIT|(}
\begin{Code}
... ( c)  DUP  BL 127 WITHIN
       IF  EMIT  ELSE
       DROP  ASCII . EMIT   THEN ...
\end{Code}
This fragment normally emits an ASCII character, but if the character
is a control code, it emits a dot. Either way, an \forthb{EMIT} is
performed. Factor \forthb{EMIT} out of the conditional structure, like
this:

\begin{Code}
... ( c)  DUP  BL 127 WITHIN NOT
       IF  DROP  ASCII .  THEN  EMIT  ...
\end{Code}
\index{E!EMIT|)}
The messiest situation occurs when the difference between two
definitions is a function within a structure that makes it impossible
to factor out the half-fragments. In this case, use stack arguments,
variables, or even vectoring. We'll see how vectoring can be used in a
section of \Chap{7} called ``Using DOER/MAKE.''

Here's a reminder about factoring code from out of a \forthb{DO }\forthb{LOOP}:

\begin{tip}
In factoring out the contents of a \forthb{DO }\forthb{LOOP} into a new
definition, rework the code so that \forth{I} (the index) is not
referenced within the new definition, but rather passed as a stack
argument to it.
\end{tip}
\index{C!Code:!factoring out from within control structures|)}
\index{C!Control structures:!factoring out|)}

\subsection{Factoring Out Control Structures Themselves}

Here are two definitions whose differences lies within a \forthb{IF}
\forthb{THEN} construct:

\begin{Code}
: ACTIVE    A B OR  C AND  IF  TUMBLE JUGGLE JUMP THEN ;
: LAZY      A B OR  C AND  IF   SIT  EAT  SLEEP   THEN ;
\end{Code}
The condition and control structure remain the same; only the event
changes. Since you can't factor the \forthb{IF} into one word and the
\forthb{THEN} into another, the simplest thing is to factor the
condition:

\begin{Code}
: CONDITIONS? ( -- ?) A B OR C AND ;
: ACTIVE    CONDITIONS? IF TUMBLE JUGGLE JUMP THEN ;
: LAZY      CONDITIONS? IF    SIT  EAT  SLEEP THEN ;
\end{Code}

\noindent 
Depending on the number of repetitions of the same condition and
control structure, you may even want to factor out both. Watch this:

\begin{Code}
: CONDITIONALLY   A B OR  C AND NOT IF  R> DROP   THEN ;
: ACTIVE   CONDITIONALLY   TUMBLE JUGGLE JUMP ;
: LAZY   CONDITIONALLY  SIT  EAT  SLEEP ;
\end{Code}
The word \forthb{CONDITIONALLY} may---depending on the condition---alter
the control flow so that the remaining words in each definition will be
skipped. This approach has certain disadvantages as well. We'll
discuss this technique---pros and cons---in \Chap{8}.

More benign examples of factoring-out control structures include case
statements,\index{C!Case statements} which eliminate nested
\forthb{IF }\forthb{ELSE }\forthb{THEN}s, and multiple exit loops (the
\forthb{BEGIN }\forthb{WHILE }\forthb{WHILE }\forthb{WHILE}
\forthb{\dots{} }\forthb{REPEAT} construct). We'll also discuss these
topics in \Chap{8}.  \index{C!Control structures:!factoring out code
from within|)}

\subsection{Factoring Out Names}
It's even good to factor out names, when the names seem almost, but
not quite, the same. Examine the following terrible example of code,
which is meant to initialize three variables associated with each of
eight channels:

\begin{Code}
VARIABLE OSTS       VARIABLE 1STS       VARIABLE 2STS 
VARIABLE 3STS       VARIABLE 4STS       VARIABLE 5STS
VARIABLE 6STS       VARIABLE 7STS       VARIABLE 0TNR
VARIABLE 1TNR       VARIABLE 2TNR       VARIABLE 3TNR
VARIABLE 4TNR       VARIABLE 5TNR       VARIABLE 6TNR
VARIABLE 7TNR       VARIABLE OUPS       VARIABLE 1UPS
VARIABLE 2UPS       VARIABLE 3UPS       VARIABLE 4UPS
VARIABLE 5UPS       VARIABLE 6UPS       VARIABLE 7UPS
\end{Code}

\begin{Code} 
: INIT-CHO   0 0STS !  1000 0TNR !  -1 OUPS ! ; 
: INIT-CHO   0 1STS !  1000 1TNR !  -1 1UPS ! ; 
: INIT-CH2   0 2STS !  1000 2TNR !  -1 2UPS ! ; 
: INIT-CH3   0 3STS !  1000 3TNR !  -1 3UPS ! ; 
: INIT-CH4   0 4STS !  1000 4TNR !  -1 4UPS ! ; 
: INIT-CH5   0 5STS !  1000 5TNR !  -1 5UPS ! ; 
: INIT-CH6   0 6STS !  1000 6TNR !  -1 6UPS ! ; 
: INIT-CH7   0 7STS !  1000 7TNR !  -1 7UPS ! ; 
\end{Code}

\begin{Code} 
: INIT-ALL-CHS    INIT-CHO  INIT-CH1  INIT-CH2  INIT-CH3
   INIT-CH4  INIT-CH5  INIT-CH6  INIT-CH7 ;
\end{Code}
First there's a similarity among the names of the variables; then
there's a similarity in the code used in all the \forth{INIT-CH}
words.

Here's an improved rendition. The similar variable names have been
factored into three data structures, and the lengthy recital of
\forth{INIT-CH} words has been factored into a \forthb{DO }\forthb{LOOP}:

\begin{Code}
: ARRAY  ( #cells -- )  CREATE  2* ALLOT
   DOES> ( i -- 'cell)  SWAP  2* + ; 
8 ARRAY STATUS  ( channel# -- adr)
8 ARRAY TENOR   (        "       )
8 ARRAY UPSHOT  (        "       )
: STABLE   8 0 DO  0 I STATUS !  1000 I TENOR ! 
   -1 I UPSHOT !  LOOP ;
\end{Code}
That's all the code we need.

\index{C!Counted strings|(}
Even in the most innocent cases, a little data structure can eliminate
extra names. By convention \Forth{} handles text in ``counted
strings'' (i.e., with the count in the first byte). Any word that
returns the ``address of a string'' actually returns this beginning
address, where the count is. not only does use of this two-element
data structure eliminate the need for separate names for string and
count, it also makes it easier to move a string in memory, because you
can copy the string \emph{and} the count with a single \forthb{CMOVE}.
\index{C!Counted strings|)}

When you start finding the same awkwardness here and there, you can
combine things and make the awkwardness go away.

\subsection{Factoring Out Functions into Defining Words}

\begin{tip}
If a series of definitions contains identical functions, with
variation only in data, use a defining word.
\end{tip}
Examine the structure of this code (without worrying about its
purpose---you'll see the same example later on):

\begin{Code}
: HUE  ( color -- color') 
   'LIGHT? @  OR  0 'LIGHT? ! ;
: BLACK   0 HUE ;
: BLUE   1 HUE ;
: GREEN   2 HUE ;
: CYAN   3 HUE ;
: RED   4 HUE ;
: MAGENTA   5 HUE ;
: BROWN   6 HUE ;
: GRAY   7 HUE ;
\end{Code}

\noindent The above approach is technically correct, but less
memory-efficient than the following approach using defining words:

\begin{Code}
: HUE   ( color -- )  CREATE ,
   DOES>  ( -- color )  @ 'LIGHT? @  OR  0 'LIGHT? ! ;
 0 HUE BLACK         1 HUE BLUE          2 HUE GREEN
 3 HUE CYAN          4 HUE RED           5 HUE MAGENTA
 6 HUE BROWN         7 HUE GRAY
\end{Code}
(Defining words are explained in \emph{Starting \Forth{}}, Chapter Eleven).

By using a defining word, we save memory because each compiled colon
definition needs the address of \forthb{EXIT} to conclude the
definition.(In defining eight words, the use of a defining word saves
14 bytes on a 16-bit \Forth{}.) Also, in a colon definition each
reference to a numeric literal requires the compilation of
\forthb{LIT} (or \forthb{literal}), another 2 bytes per
definition. (If 1 and 2 are predefined constants, this costs another
10 bytes---24 total.)

In terms of readability, the defining word makes it absolutely clear
that all the colors it defines belong to the same family of words.

The greatest strength of defining words, however, arises when a series
of definitions share the same \emph{compile-time} behavior. This topic
is the subject of a later section, ``Compile-Time Factoring.''

\section{Factoring Criteria}
Armed with an understanding of factoring techniques, let's now discuss
several of the criteria for factoring \Forth{} definitions. They
include:

\begin{enumerate}
\item Limiting the size of definitions
\item Linting repetition of code
\item Nameability
\item Information hiding
\item Simplifying the command interface
\end{enumerate}

\index{D!Defining words:!length|(}
\begin{tip}
Keep definitions short.
\end{tip}

\begin{interview}
We asked \person{Moore},\index{M!Moore, Charles|(}
``How long should a \Forth{} definition be?''

\begin{tfquot}
A word should be a line long. That's the target.

When you have a whole lot of words that are all useful in their own
right---perhaps in debugging or exploring, but inevitably there's a
reason for their existence---you feel you've extracted the essence of
the problem and that those words have expressed it.

Short words give you a good feeling.
\end{tfquot}\index{M!Moore, Charles|)}
\end{interview}
An informal examination of one of \person{Moore}'s applications shows
that he averages seven references, including both words and numbers,
per definition. These are remarkably short definitions. (Actually his
code was divided about 50--50 between one-line and two-line
definitions.)

Psychological tests have shown that the human mind can only focus its
conscious attention on seven things, plus or minus two, at a time
\cite{miller56}. Yet all the while, day and night, the vast amounts of
data, making connections and associations and solving problems.

Even if out subconscious mind knows each part of an application inside
out, our narrow-viewed conscious mind can only correlate seven
elements of it at once. Beyond that, our grasp wavers. Short
definitions match our mental capabilities.

Something that many \Forth{} programmers to write overly long
definitions is the knowledge that headers take space in the
dictionary. The coarser the factoring, the fewer the names, and the
less memory that will be wasted.

It's true that more memory will be used, but it's hard to say that
anything that helps you test, debug and interact with your code is a
``waste.'' If your application is large, try using a default width of
three, with the ability to switch to a full-length name to avoid a
specific collision. (``Width'' refers to a limit on the number of
characters stored in the name field of each dictionary header.)

If the application is still too big, switch to a \Forth{} with
multiple dictionaries on a machine with extended memory, or better
yet, a 32-bit \Forth{} on a machine with 32-bit addressing.

A related fear is that over-factoring will decrease performance due to
the overhead of \Forth{}'s inner interpreter. Again, it's true that
there is some penalty for each level of nesting. But ordinarily the
penalty for extra nesting due to proper factoring will not be
noticeable. If you timings are that tight, the real solution is to
translate something into assembler.

\begin{tip}
Factor at the point where you feel unsure about your code (where
complexity approaches the conscious limit).
\end{tip}
Don't let your ego take over with an ``I can lick this!'' attitude.
\Forth{} code should never feel uncomfortably complex. Factor!

\begin{interview}
\person{Moore}:\index{M!Moore, Charles|(}

\begin{tfquot}
Feeling like you might have introduced a bug is one reason for
factoring. Any time you see a doubly-nested \forthb{DO }\forthb{LOOP}, that's a
sign that something's wrong because it will be hard to debug. Almost
always take the inner \forthb{DO }\forthb{LOOP} and make a word.

And having factored out a word for testing, there's no reason for
putting it back. You found it useful in the first place. There's no
guarantee you won't need it again.
\end{tfquot}\index{M!Moore, Charles|)}
\end{interview}
Here's another facet of the same principle:

\begin{tip}
Factor at the point where a comment seems necessary
\end{tip}
Particularly if you feel a need to remind yourself what's on the
stack, this may be a good time to ``make a break.''

\goodbreak
Suppose you have
\begin{Code}
... BALANCE  DUP xxx xxx xxx xxx xxx xxx xxx xxx xxx
     xxx xxx xxx xxx xxx xxx   ( balance) SHOW  ...
\end{Code}
which begins by computing the balance and ends by displaying it. In
the meantime, several lines of code use the balance for purposes of
their own. Since it's difficult to see that the balance is still on
the stack when \forth{SHOW} executes, the programmer has interjected a
stack picture.

This solution is generally a sign of bad factoring. Better to write:
\begin{Code}
: REVISE  ( balance -- )  xxx xxx xxx xxx xxx xxx xxx
     xxx xxx xxx xxx xxx xxx xxx ;
... BALANCE  DUP REVISE  SHOW  ...
\end{Code}
No narrative stack pictures are needed. Furthermore, the programmer
now has a reusable, testable subset of the definition.

\begin{tip}
Limit repetition of code.
\end{tip}
\index{C!Code:!repetition of|(}%
The second reason for factoring, to eliminate repeated fragments of
code, is even more important than reducing the size of definitions.
\index{D!Defining words:!length|)}

\begin{interview}
\person{Moore}:\index{M!Moore, Charles|(}

\begin{tfquot}
When a word is just a piece of something, it's useful for clarity or
debugging, but not nearly as good as a word that is used multiple
times. Any time a word is used only once you want to question its
value.

Many times when a program has gotten too big I will go back through it
looking for phrases that strike my eye as candidates for factoring.
The computer can't do this; there are too many variables.
\end{tfquot}\index{M!Moore, Charles|)}
\end{interview}
In looking over your work, you often find identical phrases or short
passages duplicated several times. In writing an editor I found this
phrase repeated several times:

\begin{Code}
FRAME  CURSOR @ +
\end{Code}
Because it appeared several times I factored it into a new word called
\forth{AT}.

It's up to you to recognize fragments that are coded differently but
functionally equivalent, such as:

\begin{Code}
FRAME  CURSOR @ 1-  +
\end{Code}
The \forth{1-} appears to make this phrase different from the one defined as
\forth{AT}. But in fact, it can be written

\begin{Code}
AT 1-
\end{Code}
On the other hand:

\begin{tip}
When factoring out duplicate code, make sure the factored code serves
a single purpose.
\end{tip}
Don't blindly seize upon duplications that may not be useful. For
instance, in several places in one application I used this phrase:

\begin{Code}
BLK @ BLOCK  >IN @ +  C@
\end{Code}
I turned it into a new word and called it \forth{LETTER}, since it
returned the letter being pointed to by the interpreter.

In a later revision, I unexpectedly had to write:

\begin{Code}
BLK @ BLOCK  >IN @ +  C!
\end{Code}
I could have used the existing \forth{LETTER} were it not for its
\forth{C@} at the end. Rather than duplicated the bulk of the phrase
in the new section, I chose to refactor \forth{LETTER} to a finer
resolution, taking out the \forth{C@}.  The usage was then either
\forth{LETTER C@} or \forth{LETTER C!}. This change required me to
search through the listing changing all instances of \forth{LETTER} to
\forth{LETTER C@}.  But I should have done that in the first place,
separating the computation of the letter's address from the operation
to be performed on the address.
\index{C!Code:!repetition of|)}

Similar to our injunction against repetition of code:

\begin{tip}
Look for repetition of patterns.
\end{tip}
If you find yourself referring back in the program to copy the pattern
of previously-used words, then you may have mixed in a general idea
with a specific application. The part of the pattern you are copying
perhaps can be factored out as an independent definition that can be
used in all the similar cases.

\begin{tip}
Be sure you can name what you factor.
\end{tip}

\medbreak
\begin{interview}
\person{Moore}:\index{M!Moore, Charles|(}
\begin{tfquot}
If you have a concept that you can't assign a single name to , not a
hyphenated name, but a name, it's not a well-formed concept. The
ability to assign a name is a necessary part of decomposition.
Certainly you get more confidence in the idea.
\end{tfquot}\index{M!Moore, Charles|)}
\end{interview}
Compare this view with the criteria for decomposing a module espoused
by structured design in \Chap{1}. According to that method, a module
should exhibit ``functional binding,'' Which can be verified by
describing its function in a single, non-compound, \emph{sentence}.
\Forth{}'s ``atom,'' a \emph{name}, is an order of magnitude more
refined.

\begin{tip}
Factor definitions to hide details that may change.
\end{tip}
We've seen the value of information hiding in earlier chapters,
especially with regard to preliminary design. It's useful to remember
this criterion during the implementation stage as well.

Here's a very short definition that does little except hide information:

\begin{Code}
: >BODY  ( acf -- apf )  2+ ;
\end{Code}
This definition allows you to convert an acf (address of code field) to
an apf (address of parameter field) without depending on the actual
structure of a dictionary definition. If you were to use \forthb{2+} instead of
the word \forthb{>BODY}, you would lose transportability if you ever
converted to a \Forth{} system in which the heads were separated from
the bodies. (This is one of a set of words suggested by Kim Harris,
and included as an Experimental Proposal in the \Forth{}-83 Standard
\cite{harris83}.)

Here's a group of definitions that might be used in writing an editor:

\begin{Code}
: FRAME  ( -- a)  SCR @ BLOCK ;
: CURSOR  ( -- a)  R# ;
: AT  ( -- a)  FRAME  CURSOR @ + ;
\end{Code}
These three definitions can form the basis for all calculations of
addresses necessary for moving text around. Use of these three
definitions completely separates your editing algorithms from a
reliance on \Forth{} blocks.

What good is that? If you should decide, during development, to create
an editing buffer to protect the user from making errors that destroy
a block, you merely have to redefine two of these words, perhaps like
this:

\begin{Code}
CREATE FRAME  1024 ALLOT
VARIABLE CURSOR
\end{Code}
The rest of your code can remain intact. 

\begin{tip}
Factor calculations algorithms out of definitions that display results.
\end{tip}
This is really a question of decomposition.

Here's an example. The word defined below, pronounced
``people-to-paths,'' computes how many paths of communication there are
between a given number of people in a group. (This is a good thing for
managers of programmer teams to know---the number of communication
paths increases drastically with each new addition to the team.)
\begin{Code}
: PEOPLE>PATHS  ( #people -- #paths )  DUP 1-  *  2/ ;
\end{Code}
This definition does the calculation only. Here's the ``user definition''
that invokes \forth{PEOPLE>PATHS} to perform the calculation, and then
displays the result:
\begin{Code}
: PEOPLE  ( #people)
    ." = "  PEOPLE>PATHS  .  ." PATHS " ;
\end{Code}
This produces:
%% I don't know how to underline these.
\begin{Code}[commandchars=\&\{\}]
2 PEOPLE&underline{ = 1 PATHS}
3 PEOPLE&underline{ = 3 PATHS}
5 PEOPLE&underline{ = 10 PATHS}
10 PEOPLE&underline{ = 45 PATHS}
\end{Code}
Even if you think you're going to perform a particular calculation
only once, to display it in a certain way, believe me, you're wrong.
You will have to come back later an factor out the calculation part.
Perhaps you'll need to display the information in a right-justified
column, or perhaps you'll want to record the results in a data
base---you never know. But you'll always have to factor it, so you
might as well do it right the first time. (The few times you might get
away with it aren't worth the trouble.)

The word \forth{.} (dot) is a prime example. Dot is great 99\% of the
time, but occasionally it does too much. Here's what it does, in fact
(in \Forth{}--83):

\begin{Code}
: .   ( n )  DUP ABS 0 <# #S  ROT SIGN  #> TYPE SPACE ;
\end{Code}
But suppose you want to convert a number on the stack into an ASCII
string and store it in a buffer for typing later. Dot converts it, but
also types it. Or suppose you want to format playing cards in the form
\forth{10C} (for ``ten of clubs''). You can't use dot to display the 10
because it prints a final space.

Here's a better factoring found in some \Forth{} systems:

\begin{Code}
: (.)  (n -- a #)  DUP ABS 0  <# #S  ROT SIGN  #> ;
: .  ( n)  (.) TYPE SPACE ;
\end{Code}
We find another example of failing to factor the output function from
the calculation function in our own Roman numeral example in \Chap{4}.
Given our solution, we can't store a Roman numeral i8n a buffer or
even center it in a field. (A better approach would have been to use
\forthb{HOLD} instead of \forthb{EMIT}.)

Information hiding can also be a reason \emph{not} to factor. For
instance, if you factor the phrase

\begin{Code}
SCR @ BLOCK
\end{Code}
into the definition

\begin{Code}
: FRAME   SCR @ BLOCK ;
\end{Code}
remember you are doing so only because you may want to change the
location of the editing frame. Don't blindly replace all occurrences
of the phrase with the new word \forth{FRAME,} because you may change the
definition of \forth{FRAME} and there will certainly be times when you really
want \forthb{SCR }\forthb{@ }\forthb{BLOCK}.

\begin{tip}
If a repeated code fragment is likely to change in some cases but not
others, factor out only those instances that might change. If the
fragment is likely to change in more than one definition.
\end{tip}
Knowing when to hide information requires intuition and experience.
Having made many design changes in your career, you'll learn the hard
way which things will be most likely to change in the future.

You can never predict everything, though. It would be useless to try,
as we'll see in the upcoming section called ``The Iterative Approach
in Implementation.''

\index{C!Commands, reducing number of|(}
\begin{tip}
Simplify the command interface by reducing the number of commands.
\end{tip}
It may seem paradoxical, but good factoring can often yield
\emph{fewer} names. In \Chap{5} we saw how six simple names
(\forth{LEFT}, \forth{RIGHT}, \forth{MOTOR}, \forth{SOLENOID},
\forth{ON}, and \forth{OFF}) could do the work of eight
badly-factored, hyphenated names.

As another example, I found two definitions circulating in one
department in which \Forth{} had recently introduced. Their purpose
was purely instructional, to remind the programmer which vocabulary
was \forth{CURRENT}, and which was \forth{CONTEXT}:

\begin{Code}
: .CONTEXT   CONTEXT @  8 -  NFA  ID.   ;
: .CURRENT   CURRENT @  8 -  NFA  ID.  ;
\end{Code}
\goodbreak
\noindent If you typed

\begin{Code}
.CONTEXT
\end{Code}
the system would respond

\begin{Code}[commandchars=\&\{\}]
.CONTEXT&underline{ FORTH}
\end{Code}
(They worked---at least on the system used there---by backing up to the
name field of the vocabulary definition, and displaying it.)

The obvious repetition of code struck my eye as a sign of bad
factoring. It would have been possible to consolidate the repeated
passage into a third definition:

\begin{Code}
: .VOCABULARY   ( pointer )  @  8 -  NFA  ID. ;
\end{Code}
shortening the original definitions to:

\begin{Code}
: .CONTEXT   CONTEXT .VOCABULARY ;
: .CURRENT   CURRENT .VOCABULARY ;
\end{Code}
But in this approach, the only difference between the two definitions
was the pointer to be displayed. Since part of good factoring is to
make fewer, not more definitions, it seemed logical to have only one
definition, and let it take as an argument either the word
\forth{CONTEXT} or the word \forth{CURRENT}.

\medbreak
Applying the principles of good naming, I suggested:

\begin{Code}
: IS  ( adr)   @  8 -  NFA  ID. ;
\end{Code}
allowing the syntax

\begin{Code}[commandchars=\&\{\}]
CONTEXT IS&underline{ ASSEMBLER ok}
\end{Code}
or

\begin{Code}[commandchars=\&\{\}]
CURRENT IS&underline{ FORTH ok}
\end{Code}
The initial clue was repetition of code, but the final result came
from attempting to simplify the command interface.

Here's another example. The IBM PC has four modes four displaying text
only:

\begin{quote}\sf
40 column monochrome

40 column color

80 column monochrome

80 column color
\end{quote}
The word \forth{MODE} is available in the \Forth{} system I use.
\forth{MODE} takes an argument between 0 and 3 and changes the mode
accordingly. Of course, the phrase \forth{0 MODE} or \forth{1 MODE}
doesn't help me remember which mode is which.

Since I need to switch between these modes in doing presentations, I
need to have a convenient set of words to effect the change. These
words must also set a variable that contains the current number of
columns---40 or 80.

Here's the most straightforward way to fulfill the requirements:

\begin{Code}
: 40-B&W       40 #COLUMNS !  0 MODE ;
: 40-COLOR     40 #COLUMNS !  1 MODE ;
: 80-B&W       80 #COLUMNS !  2 MODE ;
: 80-COLOR     80 #COLUMNS !  3 MODE ;
\end{Code}
By factoring to eliminate the repetition, we come up with this version:

\begin{Code}
: COL-MODE!     ( #columns mode )  MODE  #COLUMNS ! ;
: 40-B&W       40 0 COL-MODE! ;
: 40-COLOR     40 1 COL-MODE! ;
: 80-B&W       80 2 COL-MODE! ;
: 80-COLOR     80 3 COL-MODE! ;
\end{Code}
But by attempting to reduce the number of commands, and also by
following the injunctions against numerically-prefixed and hyphenated
names, we realize that we can use the number of columns as a stack
argument, and \emph{calculate} the mode:

\begin{Code}
: B&W    ( #cols -- )  DUP #COLUMNS !  20 /  2-     MODE ;
: COLOR  ( #cols -- )  DUP #COLUMNS !  20 /  2-  1+ MODE ;
\end{Code}
This gives us this syntax:

\begin{Code}
40 B&W
80 B&W
40 COLOR
80 COLOR
\end{Code}
We've reduced the number of commands from four to two.

Once again, though, we have some duplicate code. If we factor out this
code we get:

\begin{Code}
: COL-MODE!  ( #columns chroma?)
   SWAP DUP #COLUMNS !  20 / 2-  +  MODE ;
: B&W    ( #columns -- )  0 COL-MODE! ;
: COLOR  ( #columns -- )  1 COL-MODE! ;
\end{Code}
Now we've achieved a nicer syntax, and at the same time greatly
reduced the size of the object code. With only two commands, as in
this example, the benefits may be marginal. But with larger sets of
commands the benefits increase geometrical.

Our final example is a set of words to represent colors on a
particular system. Names like \forth{BLUE} and \forth{RED} are nicer
than numbers. One solution might be to define:

\begin{Code}
 0 CONSTANT BLACK                 1 CONSTANT BLUE
 2 CONSTANT GREEN                 3 CONSTANT CYAN
 4 CONSTANT RED                   5 CONSTANT MAGENTA
 6 CONSTANT BROWN                 7 CONSTANT GRAY
 8 CONSTANT DARK-GRAY             9 CONSTANT LIGHT-BLUE
10 CONSTANT LIGHT-GREEN          11 CONSTANT LIGHT-CYAN
12 CONSTANT LIGHT-RED            13 CONSTANT LIGHT-MAGENTA
14 CONSTANT YELLOW               15 CONSTANT WHITE
\end{Code}
These colors can be used with words such as \forth{BACKGROUND},
\forth{FOREGROUND}, and \forth{BORDER}:

\begin{Code}
WHITE BACKGROUND  RED FOREGROUND  BLUE BORDER
\end{Code}
But this solution requires 16 names, and many of them are hyphenated.
Is there a way to simplify this?

We notice that the colors between 8 and 15 are all ``lighter''
versions of the colors between 0 and 7. (In the hardware, the only
difference between these two sets is the setting of the ``intensity
bit.'') If we factor out the ``lightness,'' we might come up with this
solution:

\begin{Code}
VARIABLE 'LIGHT?  ( intensity bit?)
: HUE  ( color)  CREATE ,
   DOES>  ( -- color )  @  'LIGHT? @  OR  0 'LIGHT? ! ;
 0 HUE BLACK         1 HUE BLUE           2 HUE GREEN
 3 HUE CYAN          4 HUE RED            5 HUE MAGENTA
 6 HUE BROWN         7 HUE GRAY
: LIGHT   8 'LIGHT? ! ;
\end{Code}
With this syntax, the word

\begin{Code}
BLUE
\end{Code}
by itself will return a ``1'' on the stack, but the phrase

\begin{Code}
LIGHT BLUE
\end{Code}
will return a ``9.'' (The adjective \forth{LIGHT} sets flag which is
used by the hues, then cleared.)

\goodbreak
If necessary for readability, we still might want to define:

\begin{Code}
8 HUE DARK-GRAY
14 HUE YELLOW
\end{Code}
Again, through this approach we've achieved a more pleasant syntax and
shorter object code.
\index{C!Commands, reducing number of|)}

\begin{tip}
Don't factor for the sake of factoring. Use clich�s.
\index{C!Cliches|(}
\end{tip}
The phrase

\begin{Code}
OVER + SWAP
\end{Code}
may seem commonly in certain applications. (I converts an address and
count into an ending address and starting address appropriate for a
\forthb{DO LOOP}.)\index{D!DO LOOP}

Another commonly seen phrase is

\begin{Code}
1+ SWAP
\end{Code}
(It rearranges a first-number and last-number into the
last-number-plus-one and first-number order required by \forthb{DO}.)

It's a little tempting to seize upon these phrases and turn them into
words, such as (for the first phrase) \forth{RANGE}.

\begin{interview}
\person{Moore}:\index{M!Moore, Charles|(}

\begin{tfquot}
That particular phrase [\forthb{OVER}\forthb{ +}\forthb{ SWAP}] is one
that's right on the margin of being a useful word. Often, though if
you define something as a word, it turns out you use it only once. If
you name such a phrase, you have trouble knowing exactly what
\forth{RANGE} does. You can't see the manipulation in your
mind. \forthb{OVER }\forthb{+ }\forthb{SWAP} has greater mnemonic
value than \forth{RANGE}.
\end{tfquot}
\index{M!Moore, Charles|)}
\end{interview}
%% Could not get that special e in cliches.
I call these phrases ``clich�s.'' They stick together as meaningful
functions. You don't have to remember how the phrase works, just what
it does. And you don't have to remember an extra name.
\index{C!Cliches|)}

\section{Compile-Time Factoring}
\index{C!Compile-time factoring|(}

\noindent
In the last section we looked at many techniques for organizing code
and data to reduce redundancy.

We can also apply limited redundancy during compilation, by letting
\Forth{} do some of out dirty work.

\begin{tip}
For maximum maintainability, limit redundancy even at compile time.
\end{tip}
Suppose in our application we must draw nine boxes as shown in
\Fig{fig6-1}.

\begin{figure*}[hhhh]
\caption{What we're supposed to display}
\labelfig{fig6-1}
\begin{center}
\begin{BVerbatim}[baselinestretch=0.7]
********     ********     ********
********     ********     ********
********     ********     ********
********     ********     ********
********     ********     ********


********     ********     ********
********     ********     ********
********     ********     ********
********     ********     ********
********     ********     ********


********     ********     ********
********     ********     ********
********     ********     ********
********     ********     ********
********     ********     ********
\end{BVerbatim}
\end{center}
\end{figure*}

In our design we need to have constants that represent values such as
the size of each box, the size of the gap between boxes, and the
left-most and top-most coordinates of the first box.

Naturally we can define:

\begin{Code}
8 CONSTANT WIDE
5 CONSTANT HIGH
4 CONSTANT AVE
2 CONSTANT STREET
\end{Code}
(Streets run east and west; avenues run north and south.)

%% I changed "We want" to "we want"
Now, to define the left margin, we might compute it mentally, We want
to center all these boxes on a screen 80 columns wide. To center
something, we subtract its width from 80 and divide by two to
determine the left margin. To figure the total width of all the boxes,
we add
\begin{eqnarray}
8 + 4 + 8 + 4 + 8 = 32\nonumber
\end{eqnarray}
(three widths and two avenues). $(80-31) / 2 = 24$.

\goodbreak
So we could, crudely, define:

\begin{Code}
24 CONSTANT LEFTMARGIN
\end{Code}
and use the same approach for \forth{TOPMARGIN}.

But what if we should later redesign the pattern, so that the width
changed, or perhaps the gap between the boxes? We'd have to recompute
the left margin ourselves.

In the \Forth{} environment, we can use the full power of \Forth{}
even when we're compiling. Why not let \Forth{} do the figuring?

\begin{Code}
WIDE 3 *  AVE 2 *  +  80 SWAP -  2/ CONSTANT LEFTMARGIN
HIGH 3 *  STREET 2 * +  24 SWAP -  2/ CONSTANT TOPMARGIN
\end{Code}

\begin{tip}
If a constant's value depends on the value of an earlier constant, use
\Forth{} to calculate the value of the second.
\end{tip}
None of these computations are performed when the application is
running, so run-time speed is not affected.

Here's another example. \Fig{fig6-2} shows the code for a word that
draws shapes. The word \forth{DRAW} emits a star at every x--y
coordinate listed in the table called \forth{POINTS}. (Note: the word
\forth{XY} positions the cursor to the ( x y ) coordinate on the
stack.)

Notice the line immediately following the list of points:

\begin{Code}
HERE POINTS -  ( /table)  2/  CONSTANT #POINTS
\end{Code}
%% I don't know how to offset the Figure 6-2.
\begin{figure*}[bbbb]
\caption{Another example of limiting compile-time redundancy.}
\labelfig{fig6-2}
\begin{center}
\begin{BVerbatim}
: P  ( x y -- )  C, C, ;
CREATE POINTS
   10 10 P     10 11 P     10 12 P     10 13 P     10 14 P
   11 10 P     12 10 P     13 10 P     14 10 P
   11 12 P     12 12 P     13 12 P     14 12 P
HERE POINTS -  ( /table)  2/  CONSTANT #POINTS
: @POINTS  ( i -- x y)  2* POINTS + DUP 1+ C@  SWAP C@ ;
: DRAW  #POINTS 0 DO  I @POINTS  XY  ASCII * EMIT  LOOP ;
\end{BVerbatim}
\end{center}
\end{figure*}
The phrase ``\forth{HERE POINTS -}'' computes the number of x--y
coordinates in the table: this value becomes the constant
\forth{\#POINTS}, used as the limit in \forth{DRAW}'s \forthb{DO }\forthb{LOOP}.
\index{D!DO LOOP}

This construct lets you add or subtract points from the table without
worrying about the number of points there are. \Forth{} computes this
for you.

\subsection{Compile-Time Factoring through Defining Words}%
\index{W!Words:!defining|see{Defining words}}%
\index{D!Defining words:!compile-time factoring through}
Let's examine a series of approaches to the same problem---defining a
group of related addresses. Here's the first try:

\begin{Code}
HEX 01A0 CONSTANT BASE.PORT.ADDRESS
BASE.PORT.ADDRESS CONSTANT SPEAKER
BASE.PORT.ADDRESS 2+ CONSTANT FLIPPER-A
BASE.PORT.ADDRESS 4 + CONSTANT FLIPPER-B
BASE.PORT.ADDRESS 6 + CONSTANT WIN-LIGHT
DECIMAL
\end{Code}
The idea is right, but the implementation is ugly. The only elements
that change from port to port are the numeric offset and the name of
the port being defined; everything else repeats. This repetition
suggests the use of a defining word.

The following approach, which is more readable, combines all the
repeated code into the ``does'' part  of a defining word:

\begin{Code}
: PORT  ( offset -- )  CREATE ,
   \ does>  ( -- 'port) @ BASE.PORT.ADDRESS + ;
0 PORT SPEAKER
2 PORT FLIPPER-A
4 PORT FLIPPER-B
6 PORT WIN-LIGHT
\end{Code}
In this solution we're performing the offset calculation at
\emph{run}-time, every time we invoke one of these names. It would be
more efficient to perform the calculation at compile time, like this:

\begin{Code}
: PORT  ( offset -- )  BASE.PORT.ADDRESS + CONSTANT ;
   \ does>  ( -- 'port)
0 PORT SPEAKER
2 PORT FLIPPER-A
4 PORT FLIPPER-B
6 PORT WIN-LIGHT
\end{Code}
Here we've created a defining word, \forth{PORT}, that has a unique
\emph{compile}-time behavior, namely adding the offset to
\forth{BASE.PORT.ADDRESS} and defining a \forth{CONSTANT}.

We might even go one step further. Suppose that all port addresses are
two bytes apart. In this case there's no reason we should have to
specify these offsets. The numeric sequence
\begin{quote}
0 2 4 6
\end{quote}
is itself redundant.

In the following version, we begin with the \forth{BASE.PORT.ADDRESS}
on the stack. The defining word \forth{PORT} duplicates this address,
makes a constant out of it, then adds 2 to the address still on the
stack, for the next invocation of \forth{PORT}.

\begin{Code}
: PORT   ( 'port -- 'next-port)  DUP CREATE ,  2+ ;
   \ does>  ( -- 'port)
BASE.PORT.ADDRESS
  PORT SPEAKER
  PORT FLIPPER-A
  PORT FLIPPER-B
  PORT WIN-LIGHT
DROP ( port.address)
\end{Code}
Notice we must supply the initial port address on the stack before
defining the first port, then invoke \forthb{DROP} when we've finished
defining all the ports to get rid of the port address that's still on
the stack.

One final comment. The base-port address is very likely to change, and
therefor should be defined in only one place. This does \emph{not}
mean it has to be defined as a constant. Provided that the base-port
address won't be used outside of this lexicon of port names, it's just
as well to refer to it by number here.

\begin{Code}
HEX 01A0  ( base port adr)
  PORT SPEAKER
  PORT FLIPPER-A
  PORT FLIPPER-B
  PORT WIN-LIGHT
DROP
\end{Code}
\index{C!Compile-time factoring|)}

\section{The Iterative Approach in Implementation}
\index{I!Iterative approach|(}
Earlier in the book we discussed the iterative approach, paying
particular attention to its impact on the design phase. Now that we're
talking about implementation, let's see how the approach is actually
used in writing code.

\begin{tip}
Work on only one aspect of a problem at a time.
\end{tip}
Suppose we're entrusted with the job of coding a word to draw or erase
a box at a given x--y coordinate. (This is the same problem we
introduced in the section called ``Compile-Time Factoring.'')

At first we focus our attention on the problem of drawing a
box---never mind erasing it. We might come up with this:

\begin{Code}
: LAYER   WIDE  0 DO  ASCII * EMIT  LOOP ;
: BOX   ( upper-left-x  upper-left-y -- )
   HIGH  0 DO  2DUP  I +  XY LAYER  LOOP  2DROP ;
\end{Code}
Having tested this to make sure it works correctly, we turn now to the
problem of using the same code to \emph{un}draw a box. The solution is
simple: instead of hard-coding the \forthb{ASCII *} we'd like to change
the emitted character from an asterisk to a blank. This requires the
addition of a variable, and some readable words for setting the
contents of the variable. So:

\begin{Code}
VARIABLE INK
: DRAW   ASCII *  INK ! ;
: UNDRAW   BL  INK ! ;
: LAYER   WIDTH  0 DO  INK @  EMIT  LOOP ;
\end{Code}
The definition of \forth{BOX}, along with the remainder of the application,
remains the same.

This approach allows the syntax

\begin{Code}
( x y ) DRAW BOX
\end{Code}
or

\begin{Code}
( x y ) UNDRAW BOX
\end{Code}
By switching from an explicit value to a variable that contains a
value, we've added a level of indirection. In this case, we've added
indirection ``backwards,'' adding a new level of complexity to the
definition of \forth{LAYER} without substantially lengthening the definition.

By concentrating on one dimension of the problem at a time, you can
solve each dimension more efficiently. If there's an error in your
thinking, the problem will be easier to see if it's not obscured by yet
another untried, untested aspect of your code.

\begin{tip}
Don't change too much at once.
\end{tip}
While you're editing your application---adding a new feature or fixing
something\hy---it's often tempting to go and fix several other things at
the same time. Our advice: Don't.

Make as few changes as you can each time you edit-compile. Be sure to
test the results of each revision before going on. You'd be amazed how
often you can make three innocent modifications, only to recompile and
have nothing work!

Making changes one at a time ensures that when it stops working, you
know why.

\begin{tip}
Don't try to anticipate ways to factor too early.
\end{tip}%
\index{A!Arrays|(}
Some people wonder why most \Forth{} systems don't include the
definition word \forth{ARRAY}. This rule is the reason.
\begin{interview}
\person{Moore}:\index{M!Moore, Charles|(}
\begin{tfquot}
I often have a class of things called arrays. The simplest array
merely adds a subscript to an address and gives you back an
address. You can define an array by saying
\begin{Code}
CREATE X   100 ALLOT
\end{Code}
then saying
\begin{Code}
X +
\end{Code}
Or you can say
\begin{Code}
: X   X + ;
\end{Code}
One of the problems that's most frustrating for me is knowing whether
it's worth creating a defining word for a particular data structure.
Will I have enough instances to justify it?

I rarely know in advance if I'm going to have more than one array. So
I don't define the word \forth{ARRAY}.

After I discover I need two arrays, the question is marginal.

If I need three then it's clear. Unless they're different. And odds
are they will be different. You may want it to fetch it for you. You
may want a byte array, or a bit array. You may want to do bounds
checking, or store its current length so you can add things to the
end.

I grit my teeth and say, ``Should I make the byte array into a cell
array, just to fit the data structure into the word I already have
available?''

The more complex the problem, the less likely it will be that you'll
find a universally applicable data structure. The number of instances
in which a truly complex data structure has found universal use is
very small. One example of a successful complex data structure is the
\Forth{} dictionary. Very firm structure, great versatility. It's used
everywhere in \Forth{}. But that's rare.

If you choose to define the word \forth{ARRAY}, you've done a
decomposition step. You've factored out the concept of an array from
all the words you'll later back in. And you've gone to another level
of abstraction.

Building levels of abstraction is a dynamic process, not one you can
predict.
\end{tfquot}\index{M!Moore, Charles|)}
\end{interview}%
\index{A!Arrays|)}
\begin{tip}
Today, make it work. Tomorrow, optimize it.
\end{tip}
\begin{interview}
Again \person{Moore}.\index{M!Moore, Charles|(} On the day of this
interview, \person{Moore} had been completing work on the design of a
board-level \Forth{} computer, using commercially available ICs. As
part of his toolkit for designing the board, he created a simulator in
\Forth{}, to test the board's logic:

\begin{tfquot}
This morning I realized I've been mixing the descriptions of the chips
with the placement of the chips on the board. This perfectly
convenient for my purposes at the moment, but when I come up with
another board that I want to use the same chips for, I have arranged
things very badly.

I should have factored it with the descriptions here and the uses
there. I would then have had a chip description language. Okay. At the
time I was doing this I was not interested in that level of
optimization.

Even if the thought had occurred to me then, I probably would have
said, ``All right, I'll do that later,'' then gone right ahead with
what I was doing. Optimization wasn't the most important thing to me
at the time.

Of course I try to factor things well. But if there doesn't seem to be
a good way to do something, I say, ``Let's just make it work.''

My motivation isn't laziness, it's knowing that there are other things
coming down the pike that are going to affect this decision in ways I
can't predict. Trying to optimize this now is foolish. Until I get the
whole picture in front of me, I can't know what the optimum is.
\end{tfquot}\index{M!Moore, Charles|)}
\end{interview}
The observations in this section shouldn't contradict what's been said
before about information hiding and about anticipating elements that
may change. A good programmer continually tries to balance the expense
of building-in changeability against the expense of changing things
later if necessary.

These decisions take experience. But as a general rule:

\begin{tip}
Anticipate things-that-may-change by organizing information, not by
adding complexity. Add complexity only as necessary to make the
current iteration work.
\end{tip}
\index{I!Iterative approach|)}

\subsection{Summary}
In this chapter we've discussed various techniques and criteria for
factoring. We also examined how the iterative approach applies to the
implementation phase.

\begin{references}{9}
\bibitem{stevens74-6} \person{W.P. Stevens}, \person{G.J. Myers}, and \person{L.L. Constantine},
\emph{ IBM Systems Journal}, vol. 13, no. 2, 1974, Copyright 1974 by
International Business Machines Corporation.
\bibitem{miller56} \person{G.A. Miller}, ``The Magical Number Seven, Plus or
Minus Two: Some Limits on our Capacity for Processing Information,''
\emph{Psychol. Rev}., vol. 63, pp. 81-97, Mar. 1956.
\bibitem{harris83} \person{Kim R. Harris}, ``Definition Field Address
Conversion Operators,'' \emph{\Forth{}--83 Standard}, \Forth{} Standards
Team.
\end{references}


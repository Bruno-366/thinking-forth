%% Thinking Forth
%% Copyright (C) 2004 Leo Brodie
%% Initial transcription by Ed Beroset
%% 
%% Chapter: Appendix C, Other Utilities Described in This Book
\Chapmark{C}
\chapter{Other~Utilities Described in~This~Book}

\initial This appendix is here to help you define some of the words 
referred to in this book that may not exist in your system.
Definitions are given in \Forth{}-83 Standard.

\section{From \Chap{4}}

A definition of \forthb{ASCII}\index{A!ASCII} that will work in '83
Standard is:
\begin{Code}
: ASCII  ( -- c)  \  Compile:  c  ( -- )
\ Interpret:   c   ( -- c)
     bl word 1+ c@  state @
     IF [compile] Literal  THEN ; immediate
\end{Code}

\section{From \Chap{5}}

The word \forthb{\bs}\index{S!Skip commands} can be defined as:
\begin{Code}
: \  ( skip rest of line)
     >in @  64 / 1+  64 *  >in ! ; immediate
\end{Code}
If you decide not to use \forthb{EXIT} to terminate a screen, you can
define \forthb{\bs S} as:
\begin{Code}
: \S   1024 >in ! ;
\end{Code}
\index{F!FH|(}%
The word \forthb{FH} can be defined simply as:
\begin{Code}
: FH   \   ( offset -- offset-block)   "from here"
    blk @ + ;
\end{Code}
This factoring allows you to use \forth{FH} in many ways, e.g.:
\begin{Code}
: TEST   [ 1 FH ] Literal load ;
\end{Code}
or
\begin{Code}
: see   [ 2 FH ] Literal list ;
\end{Code}
A slightly more complicated version of \forth{FH} also lets you edit or
load a screen with a phrase such as ``\forth{14 FH LIST},'' relative to
the screen that you just listed (\forth{SCR}):
\begin{Code}
: FH   \   ( offset -- offset-block)   "from here"
     blk @  ?dup 0= IF  scr @  THEN  + ;
\end{Code}
\index{F!FH|)}
\forthb{BL}\index{B!Blank space (BL)} is a simple constant:
\begin{Code}
32 Constant bl
\end{Code}
\forthb{TRUE}\index{T!TRUE} and \forthb{FALSE}\index{F!FALSE}
can be defined as:
\begin{Code}
0 Constant false
-1 Constant true
\end{Code}
(\Forth{}'s control words such as \forth{IF} and \forth{UNTIL} interpret
zero as ``false'' and any non-zero value as ``true.''  Before \Forth{}
'83, the convention was to indicate ``true'' with the value $1$.  Starting
with \Forth{} '83, however, ``true'' is indicated with hex \code{FFFF},
which is the signed number $-1$ (all bits set).

\forthb{WITHIN}\index{W!WITHIN} can be defined in high level like this:
\begin{Code}
: within  ( n lo hi+1 -- ?)
     >r  1- over <  swap r>  < and ;
\end{Code}
or
\begin{Code}
: within ( n lo hi+1 -- ?)
   over -  >r - r> u< ;
\end{Code}

\section{From \Chap{8}}

The implementation of \forthb{LEAP}\index{L!LEAP} will depend on how your
system implements \forthb{DO} \forthb{LOOP}s.  If \forthb{DO} keeps two
items on the return stack (the index and the limit), \forthb{LEAP} must
drop both of them plus one more return-stack item to exit:
\begin{Code}
: LEAP   r> r> 2drop  r> drop ;
\end{Code}
If \forthb{DO} keeps \emph{three} items on the return stack, it must be
defined:
\begin{Code}
: LEAP   r> r> 2drop  r> r> 2drop ;
\end{Code}


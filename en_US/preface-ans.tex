\chapter*{Preface to the ANS \Forth{} Edition}\label{preface-ans}
\pagestyle{headings}

\initial The original 1984 \emph{Thinking \Forth} feels a bit dated
today. A lot happend with \Forth{} in the last 20 years, since this
book was first published. One of the most important is the ANS
\Forth{} standard from 1994. Unlike previous \Forth{} standards, it
provided necessary abstraction for machine word size and compilation
methods. Few \Forth{}s today are still indirect threaded code, and
even fewer are 16 bit systems.

What changed, too, is coding style. \Forth{} programs are rarely
written all uppercase these days. Like other languages that started
with uppercase keywords, the result are case insensitive systems---at
least for the ASCII subset of the character set.

Screens are no longer the dominant way to keep sources. Forth
development systems usually are hosted on a modern (large) operating
system, and most people keep their sources in files.

Operating systems now provide services to programs that weren't
possible 20 years ago, and modern \Forth{} systems must be able to use
them. Paradigms like object oriented programming were adopted to
\Forth{}.

All these changes demand a rewrite of this book. Since \person{Leo
Brodie} released the original under a Creative Commons license, this
is now possible. This edition adds all the missing things from the
original:

\begin{itemize}
\item Modify the example sources so that they run with ANS \Forth{}
systems.

\item Update coding style to current practice (lower case and such).

\item Add chapters about \Forth{} and OOP, \Forth{} debugging, and
maintenance.

\item Interview \Forth{} thinkers that didn't have a chance 20 years
ago.
\end{itemize}

\begin{flushright}
\vspace{5em}
\person{Bernd Paysan}
\vspace{2.5em}
\end{flushright}


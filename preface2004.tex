\chapter*{Preface to the 2004 Edition}\label{preface2004}
\pagestyle{headings}

\initial It is an honor to find myself writing a preface again, twenty years
after the original publication of \emph{Thinking \Forth{}.} It is gratifying to
know that the concepts covered here still have validity after so many
waves of technology have crested in popularity. These are simply
concepts of good programming, discovered and rediscovered by countless
working developers over the years, and given a fresh twist by a genius
named \person{Chuck Moore}.

I've never claimed to be an expert on comparative language studies. My
recent career has centered more on requirements and functional design
than on development technologies. But with the honor of writing
another preface to this book comes another opportunity to express my
opinion.

In the course of designing Internet applications I've picked up a
little of Java and C\#. Enough to recognize in them glimmers of the
elegance that Chuck has espoused all along, but burdened with
baggage. I looked into some of the recent books that describe design
patterns, assuming that they would address recurring code designs for
real-world phenomenon. Some of the patterns do, like the controller
pattern, but too many others, such as the factory pattern, address
problems that are created by the programming language itself.

In the 1994 Preface, I apologized that my dismissal of
objected-oriented programming in the 1984 edition was a little
overreaching. What motivated that apology was having worked for a time
with an object-oriented flavor of \Forth{} developed by \person{Gary
Friedlander} for Digalog Corp. I discovered that the principles of
encapsulation could be applied elegantly to \Forth{} ``objects'' that
derived from classes that each had their own implementation of common
methods. The objects ``knew'' information about themselves, which made
code that called them simpler. But these were still just \Forth{}
constructs, and the syntax was still \Forth{}. It wasn't Java written
in \Forth{}. There was no need for garbage collection, etc. I won't
apologize now for my apology then, but please know that I didn't mean
to sell out in favor of full-blown object oriented languages.

Some people have noted parallels between \emph{Thinking \Forth{}} and
Extreme Programming. For example, emphasis on iterative development,
incrementally enhancing code that ``works'', and not over-solving the
problem, and so on.

But in my opinion, Extreme Programming seems to miss an important step
in the software development lifecycle: the design of the conceptual
model. With only one or two developers working on a project, this
phase doesn't need to be formalized because good developers do it
intuitively. But in the projects I've been working on, involving five
or more developers, it's crucial.

I define the conceptual model as the representation of how the
software \emph{appears} to work. The conceptual model is not just a
restatement of the requirements. It is the result of carefully
analyzing the top-level requirements and creatively addressing them in
a design that will make sense to the user. An example is the
``shopping basket'' construct in a commerce application. The
conceptual design forms the basis for a second tier of requirements
and drives use cases describing user/system interactions. This second
tier of requirements then drives the technical design and
implementation, or how the software actually works. The conceptual
model is designed collaboratively by the program manager, developers,
and business owners.

What I've read about Extreme Programming seems to instead assume that
requirements directly drive the implementation. In my career, I've
gravitated to the position of Program Manager, the champion of the
conceptual model. Most software developers I've worked with appreciate
my attention to defining the conceptual model before committing to a
logical and technical design. But the irony in comparing \emph{Thinking
\Forth{}} with Extreme Programming is that some developers of the XP
stripe (and who of course have never heard of this book) don't even
see the value of a spec!

Ah well.

\begin{flushright}
\emph{May wisdom, fun, and the greater good shine forth in all your work.}

\vspace{5em}
\person{Leo Brodie}
\vspace{2.5em}
\end{flushright}

\subsection{Acknowledgments for 2004 Edition}

Normally, this is where the author thanks those who helped with the
book. In this case, it's the other way around. This entire project was
conceived, executed and completed by an inspired group of people with
no prompting or significant assistance from me.

For that reason, I have asked them to describe their contributions in
their own words:

\begin{interview}
\person{John R. Hogerhuis}:

\begin{tfquot}
I contacted \person{Leo Brodie} to discuss the conditions under which
he would be willing to allow republishing \emph{Thinking \Forth}, in
electronic format under an open content license. I acquired a copy of
1984 edition, chopped off the binding, and scanned it in. Then I did a
proof-of-concept and typeset the first chapter of the book in LyX.

It turned out to be a lot more work than I had time to do by myself (I
have more ambition than sense sometimes, as my wife will attest), so I
solicited help from the \Forth{} community on {\tt comp.lang.forth}. The
outpouring of support, based on love for this book, was so tremendous
that I discovered---after dividing up work and setting the ground
rules (stick to 1984 version, use \LaTeX{} for the typesetting
language, and copyrights had to be assigned to \person{Leo
Brodie})---that the more I stayed out of the way and avoided being a
bottleneck the faster the work came together.

I also OCRed all the pages at some point using Transym OCR tool
(others used the same tool in their transcription effort), and did the
cleanup/vectorization pass on the images for \Chap{3}.
\end{tfquot}
\end{interview}
\begin{interview*}
\person{Bernd Paysan}:
\begin{tfquot}
I set up the infrastructure. I got the project approved by
Sourceforge, set up CVS, mailing lists and added developers. The
approval despite the non-commercial license is due to \person{Jacob
Moorman}, who knew this book (``despite limitations on use, I recommend
approval; this is a unique and excellent resource on \Forth{}''). The
actual approval was processed by \person{David Burley}.

As the \LaTeX{} guru, I created most of the style file, and cleaned up
most of the submissions so that they work with the style file.

I cleaned up pictures, restoring halftone and removing raster
as necessary. I translated non-hand-drawn figures to
\LaTeX{}.\footnote{Note from \person{John}: I'll add that
\person{Bernd} really took the ball and ran with it, employing the
``Free Software'' development model to impressive effect.  Of course, an
important part of most Free Software projects is one dedicated super
developer who blazes the trail and gets a large percentage of the work
done.  \person{Bernd} is that guy.}
\end{tfquot}
\end{interview*}
\begin{interview*}
\person{Andrew Nicholson}:
\begin{tfquot}
\begin{itemize}
\item extracted, rotated and converting the scanned images from
  \Chap{1}, \Chap{2}, \Chap{7}, \Chap{8} into PNGs and adding the
  images into the correct places.
  
\item transcribed \Chap{2} from OCR to \LaTeX{}

\item rebuilt the index from 1984

\item revised and cleaned up \Chap{1} and \Chap{5}

\item cleaned up \Chap{6}, \Chap{7}, \Chap{8}
\end{itemize}
\end{tfquot}
\end{interview*}
\begin{interview}
\person{Nils Holm}:
\begin{tfquot}
Transcription/initial typesetting of \Chap{4}, \Chap{7}, and \Chap{8}
\end{tfquot}
\end{interview}
\begin{interview*}
\person{Anton Ertl}:
\begin{tfquot}
I \LaTeX{}ified (typeset) and did some cleanup of \Chap{3}.
\end{tfquot}
\end{interview*}
\begin{interview*}
\person{Joseph Knapka}:
\begin{tfquot}
Transcription of \Chap{3}
\end{tfquot}
\end{interview*}
\begin{interview*}
\person{Josef Gabriel}:
\begin{tfquot}
I transcribed \Chap{6}. I see my contribution as helping to pass on
\Forth{} to other folks.  I hope folks will read \emph{``Thinking
\Forth{}''} and have their code changed.
\end{tfquot}
\end{interview*}
\begin{interview*}
\person{Ed Beroset}:
\begin{tfquot}
Typeset the epilog and appendices, and did some of the \LaTeX{} coding.
\end{tfquot}
\end{interview*}
\begin{interview*}
\person{Albert van der Horst}:
\begin{tfquot}
Transcribed/did initial typesetting for \Chap{5}
\end{tfquot}
\end{interview*}
\begin{interview*}
\person{Steve Fisher}:
\begin{tfquot}
Ran the OCR for \Chap{7} and \Chap{8}
\end{tfquot}
\end{interview*}
To all of the above, I am deeply indebted and honored.

\begin{flushright}
   \person{Leo Brodie} \\
   Seattle, WA \\
   November 2004
\end{flushright}

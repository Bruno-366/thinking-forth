%% Thinking Forth
%% Copyright (C) 2004 Leo Brodie
%% Initial transcription by Andrew Nicholson, based on OCR output
%% provided by John Hogerhuis
%%

%Page 037 in first edition

%\chapternum{TWO}
\chapter{Analysis}\Chapmark{2}

%Page 038 in first edition

\initialb Anyone who tells you there is some definite number of phases to the
software development cycle is a fool.

Nevertheless \dots{}

\section{The Nine Phases of the Programming Cycle}%

As we've seen, \Forth{} integrates aspects of design with aspects of
implementation and maintenance. As a result, the notion of a ``typical
development cycle'' makes as much sense as a ``typical noise.''

But any approach is better than no approach, and indeed, some
approaches have worked out better than others. Here is a development
cycle that represents an ``average'' of the most successful approaches
used in software projects:%
\index{P!Programming cycle:!phases|(}
\begin{description}
\item[Analysis] \hfill\index{A!Analysis}
    \begin{enumerate}
    \item Discover the Requirements and Constraints
    \item Build a Conceptual Model of the Solution
    \item Estimate Cost/Schedule/Performance
    \end{enumerate}
\item[Engineering] \hfill
    \begin{enumerate}
    \setcounter{enumi}{3}
    \item Preliminary Design
    \item Detailed Design
    \item Implementation
    \end{enumerate}
\item[Usage] \hfill
    \begin{enumerate}
    \setcounter{enumi}{6}
    \item Optimization
    \item Validation and Debugging
    \item Maintenance
    \end{enumerate}
\end{description}
In this book we'll treat the first six stages of the cycle,
focusing on analysis, design, and implementation.

In a \Forth{} project the phases occur on several levels. Looking at a
project from the widest perspective, each of these steps could take a
month or more.  One step follows the next, like seasons.

%Page 039 in first edition

But \Forth{} programmers also apply these same phases toward
defining each word. The cycle then repeats on the order of minutes.%
\index{P!Programming cycle:!phases|)}

Developing an application with this rapid repetition of the
programming cycle is known as using the ``Iterative Approach.''

\section{The Iterative Approach}%
\index{I!Iterative approach|(}%
\index{P!Programming cycle:!iterative approach|(}

The iterative approach was explained eloquently by
\person{Kim Harris} \cite{harris81}. \index{H!Harris, Kim|(}
He begins by describing the scientific method:

\begin{tfquot}
\dots{} a never-ending cycle of discovery and refinement. It first
studies a natural system and gathers observations about its behavior.
Then the observations are modeled to produce a theory about the
natural system. Next, analysis tools are applied to the model, which
produces predictions about the real system's behavior. Experiments
are devised to compare actual behavior to the predicted behavior. The
natural system is again studied, and the model is revised.

\wepsfigb{fig2-1}{The iterative approach to the
software development cycle, from ``The \Forth{} Philosophy,''
by \person{Kim Harris}, \emph{Dr. Dobb's Journal.}}%
\index{H!Harris, Kim|)}

The \emph{goal} of the method is to produce a model which accurately
predicts all observable behavior of the natural system.
\end{tfquot}
\person{Harris} then applies the scientific method to the software
development cycle, illustrated in \Fig{fig2-1}:

\begin{enumerate}
\item A problem is analyzed to determine what functions are required
in the solution.
\item Decisions are made about how to achieve those functions with
the available resources.
\item A program is written which attempts to implement the design.
\item The program is tested to determine if the functions were
implemented correctly.
\end{enumerate}
%Page 040 in first edition
Mr. \person{Harris} adds:

\begin{tfquot}
Software development in \Forth{} seeks first to find the simplest
solution to a given problem. This is done by implementing selected
parts of the problem separately and by ignoring as many constraints as
possible. Then one or a few constraints are imposed and the program is
modified.
\end{tfquot}
An excellent testimonial to the development/testing model of design is
evolution. From protozoa to tadpoles to people, each species along the
way has consisted of functional, living beings. The Creator does not
appear to be a top-down designer.

\begin{tip}
Start simple. Get it running. Learn what you're trying to do. Add
complexity gradually, as needed to fit the requirements and
constraints. Don't be afraid to restart from scratch.
\end{tip}%
\index{I!Iterative approach|)}%
\index{P!Programming cycle:!iterative approach|)}

\section{The Value of Planning}%
\index{P!Planning:!value of|(}%
\index{P!Programming cycle:!value of planning|(}

In the nine phases at the start of this chapter we listed five steps
\emph{before} ``implementation.'' Yet in \Chap{1} we saw that an
overindulgence in planning is both difficult and pointless.

Clearly you can't undertake a significant software
project---regardless of the language---without some degree of planning.
Exactly what degree is appropriate?%
\index{J!Johnson, Dave|(}
\begin{interview}
\noindent More than one \Forth{} programmer has expressed high regard for
\person{Dave Johnson}'s meticulous approach to planning. \person{Johnson}
is supervisor at Moore Products Co. in Springhouse, Pennsylvania. The firm
specializes in industrial instrumentation and process control
applications. Dave has been using \Forth{} since 1978.

He describes his approach:
\begin{tfquot}
Compared with many others that use \Forth{}, I suppose we take a more
formal approach. I learned this the hard way, though. My lack of
discipline in the early years has come back to haunt me.

We use two tools to come up with new products: a functional specification
and a design specification. Our department of Sales \& Applications comes
up with the functional specification, through customer contact.

Once we've agreed on what we're going to do, the functional
specification is turned over to our department. At that point we work
through a design, and come up with the design specification.

Up to this point our approach is no different from programming in any
language. But with \Forth{}, we go about designing somewhat
differently.  With \Forth{} you don't have to work 95\% through your
design before you can start coding, but rather 60\% before you can get
into the iterative process.

%Page 041 in first edition

A typical project would be to add a functional enhancement to one of
our products. For example, we have an intelligent terminal with disk
drives, and we need certain protocols for communicating with another
device. The project to design the protocols, come up with displays,
provide the operator interfaces, etc. may take several months. The
functional specification takes a month; the design specification takes
a month; coding takes three months; integration and testing take
another month.

This is the typical cycle. One project took almost two years, but six
or seven months is reasonable.

When we started with \Forth{} five years ago, it wasn't like that. When I
received a functional specification, I just started coding. I used a
cross between top-down and bottom-up, generally defining a structure,
and as I needed it, some of the lower level, and then returning with
more structure.

The reason for that approach was the tremendous pressure to show
something to management. We wound up never writing down what we were
doing. Three years later we would go back and try to modify the code,
without any documentation. \Forth{} became a disadvantage because it
allowed us to go in too early. It was fun to make the lights flash and
disk drives hum. But we didn't go through the nitty-gritty design
work. As I said, our ``free spirits'' have come back to haunt us.

Now for the new programmers, we have an established requirement: a
thorough design spec that defines in detail all the high-level \Forth{}
words---the tasks that your project is going to do. No more reading a
few pages of the functional specification, answering that, reading a
few more, answering that, etc.

No living programmer likes to document. By ensuring the design ahead
of time, we're able to look back several years later and remember what
we did.

I should mention that during the design phase there is some amount of
coding done to test out certain ideas. But this code may not be part
of the finished product. The idea is to map out your design.
\end{tfquot}%
\index{J!Johnson, Dave|)}
\end{interview}
\person{Johnson} advises us to complete the design specification
before starting to code, with the exception of needed preliminary
tests. The next interview backs up this point, and adds some
additional reasons.%
\begin{interview}
\index{T!Teleska, John|(}
\noindent \person{John Teleska} has been an independent software
consultant since 1976, specializing in custom applications for
academic research environments.  He enjoys providing research tools
``right at the edge of what technology is able to do.''
\person{Teleska} works in Rochester, New York:

\begin{tfquot}
I see the software development process as having two phases. The first is
making sure I understand what the problem is. The second is
implementation, including debugging, verification, etc.

My goal in Phase One is an operational specification. I start with a
problem description, and as I proceed it becomes the operational
specification. My understanding of the problem metamorphoses into a
solution. The better the understanding, the more complete the
solution. I look for closure; a sense of having no more questions that
aren't answered in print.

I've found that on each project I've been putting more time into Phase
One, much to the initial dismay of many of my clients. The limiting
factor is how
%Page 042 in first edition
much I can convince the client it's necessary to spend that time up
front.  Customers generally don't know the specifications for the job
they want done. And they don't have the capital---or don't feel they
do---to spend on good specs. Part of my job is to convince them it
will end up costing more time and money not to.

Some of Phase One is spent on feasibility studies. Writing the spec
unearths uncertainties. I try to be as uncertain about uncertainties
as possible. For instance, they may want to collect 200,000 samples a
second to a certain accuracy. I first need to find out if it's even
possible with the equipment they've got. In this case I've got to test
its feasibility by writing a patch of code.

Another reason for the spec is to cover myself. In case the
application performs to the spec but doesn't fully satisfy the
customer, it's the customer's responsibility. If the customer wants
more, we'll have to renegotiate. But I see it as the designer's
responsibility to do whatever is necessary to generate an operational
specification that will do the job to the customer's satisfaction.

I think there are consultants who bow to client pressure and limit the
time they spend on specs, for fear of losing the job. But in these
situations nobody ends up happy.
\end{tfquot}
\index{T!Teleska, John|)}
\end{interview}%
We'll return to the \person{Teleska} interview momentarily.%
\index{P!Planning:!value of|)}%
\index{P!Programming cycle:!value of planning|)}

\section{The Limitations of Planning}%
\index{P!Planning:!limitations of|(}%
\index{P!Programming cycle:!limitations of planning|(}

Experience has taught us to map out where we're going before we begin
coding. But planning has certain limitations. The following interviews
give different perspectives to the value of planning.%
\begin{interview}
\index{T!Teleska, John|(}
\noindent Despite \person{Teleska}'s preference for a well-planned
project, he suggests that the choice between a top-down and bottom-up
approach may depend on the situation:

\begin{tfquot}
On two recent projects involving a lot of technical interface work, I
did the whole thing bottom-up. I milled around in a bunch of
data-sheets and technical descriptions of little crannies of the
operating system I was dealing with. I felt lost most of the time,
wondering why I ever took the job on.  Then finally I reached a
critical mass of some sort and began putting small programs together
that made small things happen. I continued, bottom-up, until I matched
the target application.

My top-down sense was appalled at this procedure. But I've seen me go
through this process successfully too many times to discount it for
any pedagogical reasons. And there is always this difficult phase
which it seems no amount of linear thinking will penetrate.
Programming seems a lot more intuitive than we, in this business, tell
each other it ought to be.

I think if the application elicits this sense of being lost, I proceed
bottom-up. If the application is in familiar territory then I'll
probably use a more traditional by-the-book approach.
\end{tfquot}
\index{T!Teleska, John|)}
\end{interview}

%Page 043 in first edition

\noindent And here's another view:%
\index{S!Starling, Michael|(}
\begin{interview}
\noindent At the time I interviewed him, \person{Michael Starling} of Union
Carbide was putting the final touches on two applications involving
user-configurable laboratory automation and process control automation
systems. For the pilot plant system, \person{Starling} designed both the
hardware and software to known requirements; on the laboratory
automation system he also defined the requirements himself.

His efforts were extremely successful. On one project, the new system
typically costs only 20\% as much as the equivalent system and
requires days, instead of months, to install and configure.

I asked him what techniques of project management he employed.

\begin{tfquot}
On both of these projects much design was needed. I did not follow the
traditional analysis methods, however. I did employ these steps:

First, I clearly defined the boundaries of the problem.

Second, I determined what the smaller functional pieces, the software
subsystems, had to be.

Third, I did each piece, put them together, and the system ran.

Next, I asked the users ``Does this meet your requirements?''
Sometimes it didn't, and in ways that neither the users nor the
specification designers could have anticipated.

For instance, the designers didn't realize that the original
specification wouldn't produce pleasing, human-oriented graphics
displays. After working with the interactive graphics on the first
version, users were applying arbitrary scales and coming up with
oddball displays.

So even after the basic plot algorithm was designed, we realized we
needed auto-scaling. We went back in and analyzed how human beings
plot data and wrote a first level plot function that evaluates the x
and y data and how much will fit on the graph.

After that, we realized that not all the data taken will be of
interest to experimenters. So we added a zoom capability.

This iterative approach resulted in cleaner code and better thought
out code. We established a baseline set of goals and built a minimal
system to the users' known requirements. Then we'd crank in the
programmer's experience to improve it and determine what the users
forgot they needed when they generated the specs.

The users did not invent most of the new ideas. The programmers did,
and they would bounce these ideas off the users. The problem
definition was a two-way street. In some cases they got things they
didn't know they could do on such a small computer, such as applying
digital filters and signal processing to the data.

One of the things about \Forth{} that makes this approach possible is
that primitives are easily testable. It takes some experience with
\Forth{} to learn how to take advantage of this. Guys from traditional
environments want to write ten pages of code at their desk, then sit
down to type it in and expect it to work.

To summarize my approach: I try to find out from the users what they
need, but at the same time recognizing its incompleteness. Then I keep
%Page 044 in first edition
them involved in the design during the implementation, since they have
the expertise in the application.  When they see the result, they feel
good because they know their ideas were involved.

The iterative approach places highest value on producing a good
solution to the real problem. It may not always give you the most
predictable software costs. The route to a solution may depend upon
your priorities. Remember:
\begin{list}{}{}
\item Good
\item Fast
\item Cheap
\end{list}
Pick any two!
\end{tfquot}
\end{interview}

\noindent As \person{Starling} observes, you don't know completely what
you're doing till you've done it once. In my own experience, the best way
to write an application is to write it twice. Throw away the first
version and chalk it up to experience.%
\index{S!Starling, Michael|)}%
\index{K!Kogge, Peter|(}
\begin{interview}
\noindent \person{Peter Kogge} is Senior Technical Staff in the IBM
Federal Systems Division, Oswego, New York:

\begin{tfquot}
One of the key advantages I find in \Forth{} is that it allows me to
very quickly prototype an application without all the bells and
whistles, and often with significant limitations, but enough to wring
out the ``human interface'' by hands-on trial runs.

When I build such a prototype, I do so with the firm constraint that I
will use not a single line of code from the prototype in the final
program. This enforced ``do-over'' almost always results in far
simpler and more elegant final programs, even when those programs are
written in something other than \Forth{}.
\end{tfquot}
\end{interview}%
\index{K!Kogge, Peter|)}

\noindent Our conclusions? In the \Forth{} environment planning is
necessary. But it should be kept short. Testing and prototyping are
the best ways to discover what is really needed.

A word of caution to project managers: If you're supervising any
experienced \Forth{} programmers, you won't have to worry about them
spending too much time on planning. Thus the following tip has two
versions:

\begin{tip}
For newcomers to \Forth{} (with ``traditional'' backgrounds):
Keep the analysis phase to a minimum.

\medskip
For \Forth{} addicts (without a ``traditional'' background):
Hold off on coding as long as you can possibly stand it.
\end{tip}
%Page 045 in first edition
Or, as we observed in \Chap{1}:

\begin{tip}
Plan for change (by designing components that can be changed).
\end{tip}
Or, simply:

\begin{tip}
Prototype.
\end{tip}%
\index{P!Planning:!limitations of|)}%
\index{P!Programming cycle:!limitations of planning|)}

\section{The Analysis Phase}%
\index{A!Analysis|(}%

In the remainder of this chapter we'll discuss the analysis phase.
Analysis is an organized way of understanding and documenting what
the program should do.%
\index{A!Analysis!defined}

With a simple program that you write for yourself in less than an
hour, the analysis phase may take about 250 microseconds. At the other
extreme, some projects will take many man-years to build. On such a
project, the analysis phase is critical to the success of the entire
project.

We've indicated three parts to the analysis phase:

\begin{enumerate}\parsep=0pt\itemsep=0pt
\item Discovering the requirements and constraints
\item Building a conceptual model of the solution
\item Estimating cost, scheduling, and performance
\end{enumerate}

\noindent Let's briefly describe each part:

\subsection{Discovering the Requirements}%
\index{A!Analysis!requirements|(}%
\index{R!Requirements|(}

The first step is to determine what the application should do. The
customer, or whoever wants the system, should supply a ``requirements
specification.'' This is a modest document that lists the minimum
capabilities for the finished product.

The analyst may also probe further by conducting interviews and
sending out questionnaires to the users.%
\index{A!Analysis!requirements|)}%
\index{R!Requirements|)}

\subsection{Discovering the Constraints}%
\index{A!Analysis!constraints|(}%
\index{C!Constraints|(}

The next step is to discover any limiting factors. How important is
speed? How much memory is available? How soon do you need it?

No matter how sophisticated our technology becomes, programmers will
always be bucking limitations. System capacities inexplicably
%Page 046 in first edition
diminish over time. The double-density disk drives that once were the
answer to my storage prayers no longer fill the bill. The
double-sided, double-density drives I'll get next will seem like a
vast frontier---for a while. I've heard guys with 10-megabyte hard
disks complain of feeling cramped.

Whenever there's a shortage of something---and there always will
be---tradeoffs have to be made. It's best to use the analysis phase to
anticipate most limitations and decide which tradeoffs to make.

On the other hand, you should \emph{not} consider other types of
constraints during analysis, but should instead impose them gradually
during implementation, the way one stirs flour into gravy.

The type of constraint to consider during analysis includes those that
might affect the overall approach. The type of defer includes those
that can be handled by making iterative refinements to the planned
software design.

As we heard in our earlier interviews, finding out about \emph{hardware}
constraints\index{H!Hardware constraints} often requires writing
some test code and trying things out.

Finding out about the \emph{customer's} constraints%
\index{C!Customer constraints}
is usually a matter of asking the customer, or of taking written
surveys. ``How fast do you need such-and-such, on a scale of one to
ten?'', etc.%
\index{A!Analysis!constraints|)}%
\index{C!Constraints|)}

\subsection{Building a Conceptual Model of the Solution}%
\index{A!Analysis!conceptual model|(}%
\index{C!Conceptual model|(}

A conceptual model is an imaginary solution to the problem. It is a
view of how the system \emph{appears} to work. It is an answer to all
the requirements and constraints.%
\index{C!Conceptual model!defined}

\wepsfigp{img2-047}{Refining the conceptual model to meet
requirements and constraints.}

If the requirements definition is for ``something to stand on to paint
the ceiling,'' then a description of the conceptual model is ``a
device that is free-standing (so you can paint the center of the
room), with several steps spaced at convenient intervals (so you can
climb up and down), and having a small shelf near the top (to hold
your paint can).''

A conceptual model is not quite a design, however. A design begins to
describe how the system \emph{really} works. In design, the image of a
step ladder would begin to emerge.

\Forth{} blurs the distinction a little, because all definitions are
written in conceptual terms, using the lexicons of lower level
components. In fact, later in this chapter we'll use \Forth{}
``pseudocode'' to describe conceptual model solutions.

Nevertheless, it's useful to make the distinction. A conceptual model
is more flexible than a design. It's easier to fit the requirements
and constraints into the model than into a design.

\begin{tip}
Strive to build a solid conceptual model before beginning the design.
\end{tip}

%Page 047 in first edition
%img2-47 moved forward
%Page 048 in first edition

\noindent Analysis consists of expanding the requirements definition
into a conceptual model. The technique involves two-way communication
with the customer in successive attempts to describe the model.

Like the entire development cycle, the analysis phase is best approach
iteratively. Each new requirement will tend to suggest something in
your mental model. Your job is to juggle all the requirements and
constraints until you can weave a pattern that fits the bill.

\wepsfigb{fig2-2}{An iterative approach to analysis.}

\Fig{fig2-2} illustrates the iterative approach to the analysis phase.
The final step is one of the most important: show the documented model
to the customer. Use whatever means of communication are
necessary---diagrams, tables, or cartoons---to convey your
understanding to the customer and get the needed feedback. Even if you
cycle through this loop a hundred times, it's worth the effort.

In the next three sections we'll explore three techniques for defining
and documenting the conceptual model:

\begin{enumerate}
\item defining the interfaces
\item defining the rules
\item defining the data structures.
\end{enumerate}%
\index{A!Analysis|)}

\section{Defining the Interfaces}%
\index{A!Analysis!interface definition|(}%
\index{C!Conceptual model!interface definition|(}%
\index{I!Interface definition|(}

\begin{tip}
First, and most importantly, the conceptual model should describe the
system's interfaces.\index{I!Interface definition}
\end{tip}%
\begin{interview}
\index{T!Teleska, John|(}
\person{Teleska}:

\begin{tfquot}
The ``spec'' basically deals with WHAT. In its most glorious form, it
describes what the system would look like to the user---you might call it
the user's manual. I find I write more notes on the human
interaction---what it
%Page 049 in first edition
will look like on the outside---than on the part that gets the job
done. For instance, I'll include a whole error-action listing to show
what happens when a particular error occurs. Oddly, this is the part
that takes the most time to implement anyway.

I'm currently working on a solid-state industrial washing-machine
timer. In this case, the user interface is not that complex. What is
complex is the interface to the washing machine, for which I must
depend on the customer and the documentation they can provide.

The significant interface is whatever is the arms and legs of the
product. I don't make the distinction between hardware and software at
this early stage. They can be interchanged in the implementation.

The process of designing hardware and the process of designing
software are analogous. The way I design hardware is to treat it as a
black box. The front panel is input and output. You can do the same
with software.

I use any techniques, diagrams, etc., to show the customer what the
inputs and outputs look like, using his description of what the
product has to do.  But in parallel, in my own mind, I'm imagining how
it will be implemented.  I'm evaluating whether I can do this
efficiently. So to me it's not a black box, it's a gray box. The
designer must be able to see inside the black boxes.

When I design a system that's got different modules, I try to make the
coupling as rational and as little as possible. But there's always
give and take, since you're compromising the ideal.

\index{D!Data-flow diagrams|(}
For the document itself, I use DFDs {[}data-flow diagrams, which we'll
discuss later{]}, and any other kind of representation that I can show
to my client. I show them as many diagrams as I can to clarify my
understanding.  I don't generally use these once it comes to
implementation. The prose must be complete, even without reference to
the diagrams.
\end{tfquot}%
\index{T!Teleska, John|)}
\end{interview}%
\index{E!Error handling|(}
\begin{tip}
Decide on error- and exception-handling early as part of defining the
interface.
\end{tip}

\noindent It's true that when coding for oneself, a programmer can
often concentrate first on making the code run correctly under
\emph{normal} conditions, then worry about error-handling later. When
working for someone else, however, error-handling should be worked out
ahead of time. This is an area often overlooked by the beginning
programmer.

The reason it's so important to decide on error-handling at this stage
is the wide divergence in how errors can be treated. An error might be:

\begin{itemize}
\item ignored
\item made to set a flag indicating that an error occurred, while
processing continues
\item made to halt the application immediately
\item designed to initiate procedures to correct the problem and keep
the program running.
\end{itemize}
%
%Page 050 in first edition
%
There's room for a serious communications gap if the degree of
complexity required in the error-handling is not nailed down early.
Obviously, the choice bears tremendous impact on the design and
implementation of the application.%
\index{E!Error handling|)}

\begin{tip}
Develop the conceptual model by imagining the data traveling through and
being acted upon by the parts of the model.
\end{tip}%
\index{S!Structured analysis|(}
A discipline called \emph{structured analysis} \cite{weinberg80}
offers some techniques for describing interfaces in ways that your
clients will easily understand.  One of these techniques is called the
``data-flow diagram'' (DFD), which \person{Teleska} mentioned.

\wepsfiga{fig2-3}{A data-flow diagram.}

A data-flow diagram, such as the one depicted in \Fig{fig2-3},
emphasizes what happens to items of data as they travel through the
system.  The circles represent ``transforms,'' functions that act upon
information.  The arrows represent the inputs and outputs of the
transforms.

The diagram depicts a frozen moment of the system in action. It
ignores initialization, looping structures, and other details of
programming that relate to time.

Three benefits are claimed for using DFDs:

First, they speak in simple, direct terms to the customer. If your
%Page 051 in first edition
customer agrees with the contents of your data-flow diagram, you know
you understand the problem.

Second, they let you think in terms of the logical ``whats,'' without
getting caught up in the procedural ``hows,'' which is consistent with
the philosophy of hiding information as we discussed in the last chapter.

Third, they focus your attention on the interfaces to the system and
between modules.

\Forth{} programmers, however, rarely use DFDs except for the customer's
benefit. \Forth{} encourages you to think in terms of the conceptual
model, and \Forth{}'s implicit use of a data stack makes the passing of
data among modules so simple it can usually be taken for granted.
This is because \Forth{}, used properly, approaches a functional language.%
\index{D!Data-flow diagrams|)}%
\index{S!Structured analysis|)}%
\index{P!Pseudocode|(}

For anyone with a few days' familiarity with \Forth{}, simple definitions
convey at least as much meaning as the diagrams:
\begin{Code}
: REQUEST  ( quantity part# -- )
   ON-HAND?  IF  TRANSFER  ELSE  REORDER  THEN ;
: REORDER   AUTHORIZATION?  IF  P.O.  THEN ;
: P.O.   BOOKKEEPING COPY   RECEIVING COPY
   VENDOR MAIL-COPY ;
\end{Code}
This is \Forth{} pseudocode. No effort has been made to determine what
values are actually passed on the stack, because that is an
implementation detail. The stack comment for \forth{REQUEST} is used
only to indicate the two items of data needed to initiate the process.

(If I were designing this application, I'd suggest that the user
interface be a word called \forth{NEED}, which has this syntax:

\begin{Code}
NEED 50 AXLES
\end{Code}

\noindent \forth{NEED} converts the quantity into a numeric value on
the stack, translates the string \forth{AXLES} into a part number,
also on the stack, then calls \forth{REQUEST}. Such a command should
be defined only at the outer-most level.)%
\begin{interview}
\index{J!Johnson, Dave|(}
\person{Johnson} of Moore Products Co. has a few words on \Forth{}
pseudocode:
\begin{tfquot}
IBM uses a rigorously documented PDL (program design language). We use
a PDL here as well, although we call it FDL, for \Forth{} design
language. It's probably worthwhile having all those standards, but
once you're familiar with \Forth{}, \Forth{} itself can be a design
language. You just have to leave out the so-called ``noise'' words:
\forth{C@}, \forth{DUP}, \forth{OVER}, etc., and show only the basic
flow. Most \Forth{} people probably do that informally. We do it
purposefully.
\end{tfquot}%
\index{J!Johnson, Dave|)}%
\end{interview}%
\index{P!Pseudocode|)}
%Page 052 in first edition
\begin{interview*}
During one of our interviews I asked \person{Moore}\index{M!Moore,
Charles|(} if he used diagrams of any sort to plan out the conceptual
model, or did he code straight into \Forth{}? His reply:

\begin{tfquot}
The conceptual model \emph{is} \Forth{}. Over the years I've learned
to think that way.
\end{tfquot}
Can everyone learn to think that way?

\begin{tfquot}
I've got an unfair advantage. I codified my programming style and other
people have adopted it. I was surprised that this happened. And I feel at a
lovely advantage because it is my style that others are learning to emulate.
Can they learn to think like I think? I imagine so. It's just a matter of
practice, and I've had more practice.
\end{tfquot}\index{M!Moore, Charles|)}
\end{interview*}%
\index{A!Analysis!interface definition|)}%
\index{C!Conceptual model!interface definition|)}%
\index{I!Interface definition|)}

\section{Defining the Rules}%
\index{A!Analysis!rule definition|(}%
\index{C!Conceptual model!rule definition|(}%
\index{R!Rule definition|(}%
\label{phone1}

Most of your efforts at defining a problem will center on describing
the interface.%
\index{I!Interface definition}
Some applications will also require that you define the set of
application rules.

All programming involves rules. Usually these rules are so simple it
hardly matters how you express them: ``If someone pushes the button,
ring the bell.''

Some applications, however, involve rules so complicated that they
can't be expressed in a few sentences of English. A few formal techniques
can come in handy to help you understand and document these more
complicated rules.

Here's an example. Our requirements call for a system to compute the
charges on long-distance phone calls. Here's the customer's
explanation of its rate structure. (I made this up; I have no idea how
the phone company actually computes their rates except that they
overcharge.)

\begin{tfquot}
All charges are computed by the minute, according to distance in
hundreds of miles, plus a flat charge. The flat charge for direct dial
calls during weekdays between 8 A.M. and 5 P.M. is .30 for the first
minute, and .20 for each additional minute; in addition, each minute
is charged .12 per 100 miles. The flat charge for direct calls during
weekdays between 5 P.M. and 11 P.M. is .22 for the first minute, and
.15 for each additional minute; the distance rate per minute is .10
per 100 miles. The flat charge for direct calls late during weekdays
between 11 P.M. or anytime on Saturday, Sundays, or holidays is .12
for the first minute, and .09 for each additional minute; the distance
rate per minute is .06 per 100 miles. If the call requires assistance
from the operator, the flat charge increases by .90, regardless of the hour.
\end{tfquot}
This description is written in plain old English, and it's quite a
mouthful.  It's hard to follow and, like an attic cluttered with
accumulated belongings, it may even hide a few bugs.

%Page 053 in first edition
In building a conceptual model for this system, we must describe the
rate structure in an unambiguous, useful way. The first step towards
cleaning up the clutter involves factoring out irrelevant pieces of
information---that is, applying the rules of limited redundancy. We
can improve this statement a lot by splitting it into two statements.
First there's the time-of-day rule:%
\index{A!Analysis!conceptual model|)}

\begin{tfquot}
Calls during weekdays between 8 A.M. and 5 P.M. are charged at ``full'' rate.
Calls during weekdays between 5 P.M. and 11 P.M. are charged at ``lower''
rate. Calls placed during weekdays between 11 P.M. or anytime on Saturday,
Sundays, or holidays are charged at the ``lowest'' rate.
\end{tfquot}
Then there's the rate structure itself, which should be described in
terms of ``first-minute rate,'' ``additional minute rate,'' ``distance
rate,'' and ``operator-assistance rate.''

\begin{tip}
Factor the fruit. (Don't confuse apples with oranges.)
\end{tip}
These prose statements are still difficult to read, however. System
analysts use several techniques to simplify these statements:
structured English, decision trees, and decision tables. Let's study
each of these techniques and evaluate their usefulness in the \Forth{}
environment.

\subsection{Structured English}%
\index{A!Analysis!Structured English|(}%
\index{S!Structured English|(}

Structured English is a sort of structured pseudocode in which our rate
statement would read something like this:

%Page 053/054 in first edition
\begin{Code}[baselinestretch=0.95]
IF full rate
   IF direct-dial
      IF first-minute
	 .30 + .12/100miles
      ELSE ( add'l- minute)
	 .20 + .12/100miles
      ENDIF
   ELSE ( operator )
      IF first-minute
	 1.20 + .12/100miles
      ELSE ( add'l- minute)
	 .20 + .12/100miles
      ENDIF
   ENDIF
ELSE  ( not-full-rate)
   IF lower-rate
      IF direct-dial
	 IF first-minute
	    .22 + .10/100miles
	 ELSE ( add'l- minute)
	    .15 + .10/100miles
	 END IF
      ELSE ( operator)
	 IF first-minute
	    1.12 + .10/100miles
	 ELSE ( add'l- minute)
	    .15 + .10/100miles
	 ENDIF
      ENDIF
   ELSE ( lowest-rate)
      IF direct-dial
	 IF first-minute
	    .12 + .06/100miles
	 ELSE ( add'l- minute)
	    .09 + .O6/100miles
	 ENDIF
      ELSE ( operator)
	 IF first-minute
	    1.02 + .O6/100miles
	 ELSE ( add'l- minute)
	    .09 + .06/100miles
	 ENDIF
      ENDIF
   ENDIF
ENDIF
\end{Code}
This is just plain awkward. It's hard to read, harder to maintain, and
hardest to write. And for all that, it's worthless at implementation
time. I don't even want to talk about it anymore.%
\index{A!Analysis!Structured English|)}%
\index{S!Structured English|)}

\subsection{The Decision Tree}%
\index{A!Analysis!decision tree|(}%
\index{D!Decision tree|(}

\wepsfigt{fig2-4}{Example of a decision tree.}

\Fig{fig2-4} illustrates the telephone rate rules by means of a
decision tree.  The decision tree is the easiest method of any to
``follow down'' to determine the result of certain conditions. For
this reason, it may be the best representation to show the customer.

Unfortunately, the decision tree is difficult to ``follow up,'' to
determine which conditions produce certain results. This difficulty
inhibits seeing ways to simplify the problem. The tree obscures the
fact that additional minutes cost the same, whether the operator
assists or not. You can't see the facts for the tree.%
\index{A!Analysis!decision tree|)}%
\index{D!Decision tree|)}

%Page 055 in first edition

% Figure 2-4 was here

\subsection{The Decision Table}%
\index{A!Analysis!decision table|(}%
\index{D!Decision table|(}

The decision table, described next, provides the most usable graphic
representation of compound rules for the programmer, and possibly for
the customer as well. \Fig{fig2-5} shows our rate structure rules in
decision-table form.

\wepsfigt{fig2-5}{The decision table.}

In \Fig{fig2-5} there are three dimensions: the rate discount, whether
an operator intervenes, and initial minute vs. additional minute.

Drawing problems with more than two dimensions gets a little tricky.
As you can see, these additional dimensions can be depicted on
%Page 056 in first edition
paper as subdimensions within an outer dimension. All of the
subdimension's conditions appear within every condition of the outer
dimension.  In software, any number of dimensions can be easily
handled, as we'll see.

All the techniques we've described force you to analyze which
conditions apply to which dimensions. In factoring these dimensions,
two rules apply:

First, all the elements of each dimension must be mutually exclusive.
You don't put ``first minute'' in the same dimension as ``direct
dial,'' because they are not mutually exclusive.

Second, all possibilities must be accounted for within each dimension.
If there were another rate for calls made between 2 A.M. to 2:05 A.M.,
the table would have to be enlarged.

But our decision tables have other advantages all to themselves.  The
decision table not only reads well to the client but actually benefits
the implementor in several ways:

\begin{description}
\item[Transferability to actual code.] This is particularly true in
\Forth{}, where decision tables are easy to implement in a form very
similar to the drawing.

\item[Ability to trace the logic upwards.] Find a condition and see what
factors produced it.

\item[Clearer graphic representation.] Decision tables serve as a better
tool for understanding, both for the implementor and the analyst.
\end{description}

Unlike decision trees, these decision tables group the \emph{results}
together in a graphically meaningful way. Visualization of ideas helps in
understanding problems, particularly those problems that are too
complex to perceive in a linear way.

For instance, \Fig{fig2-5} clearly shows that the charge for
additional minutes does not depend on whether an operator assisted or not.
With this new understanding we can draw a simplified table, as shown
in \Fig{fig2-6}.

\wepsfigt{fig2-6}{A simplified decision table.}

%Page 057 in first edition

It's easy to get so enamored of one's analytic tools that one forgets
about the problem. The analyst must do more than carry out all
possibilities of a problem to the nth degree, as I have seen authors
of books on structured analysis recommend. That approach only
increases the amount of available detail. The problem solver must also
try to simplify the problem.

\begin{tip}
You don't understand a problem until you can simplify it.
\end{tip}

\noindent If the goal of analysis is not only understanding, but
simplification, then perhaps we've got more work to do.

Our revised decision table (\Fig{fig2-6}) shows that the per-mile
charge depends only on whether the rate is full, lower, or lowest. In
other words, it's subject to only one of the three dimensions shown in
the table.  What happens if we split this table into two tables, as in
\Fig{fig2-7}?

\wepsfiga{fig2-7}{The sectional decision table.}

Now we're getting the answer through a combination of table look-up
and calculation. The formula for the per-minute charge can be
expressed as a pseudo\Forth{} definition:

\begin{Code}
: PER-MINUTE-CHARGE ( -- per-minute-charge)
        CONNECT-CHARGE  MILEAGE-CHARGE  + ;
\end{Code}
The ``\forth{+}'' now appears once in the definition,
not nine times in the table.

Taking the principle of calculation one step further, we note (or
remember from the original problem statement) that operator assistance
merely adds a one-time charge of .90 to the total charge. In this
sense, the operator charge is not a function of any of the three
dimensions. It's more
%Page 058 in first edition
appropriately expressed as a ``logical calculation''; that is, a
function that combines logic with arithmetic:

\begin{Code}
: ?ASSISTANCE
   ( direct-dial-charge -- total-charge)
   OPERATOR? IF .90 + THEN ;
\end{Code}
(But remember, this charge applies only to the first minute.)

\wepsfigt{fig2-8}{The decision table without operator involvement depicted.}

This leaves us with the simplified table shown in \Fig{fig2-8}, and an
increased reliance on expressing calculations. Now we're getting
somewhere.

Let's go back to our definition of \forth{PER-MINUTE-CHARGE}:
\begin{Code}
: PER-MINUTE-CHARGE ( -- per-minute-charge)
   CONNECT-CHARGE  MILEAGE-CHARGE  + ;
\end{Code}
Let's get more specific about the rules for the connection charge and for
the mileage charge.

The connection charge depends on whether the minute is the first or
an additional minute. Since there are two kinds of per-minute charges,
perhaps it will be easiest to rewrite \forth{PER-MINUTE-CHARGE} as two
different words.

Let's assume we will build a component that will fetch the appropriate
rates from the table. The word \forth{1MINUTE} will get the rate for
the first minute; \forth{+MINUTES} will get the rate for each
additional minute.  Both of these words will depend on the time of day
to determine whether to use the full, lower, or lowest rates.

Now we can define the pair of words to replace \forth{PER-MINUTE-CHARGE}:

%Page 059 in first edition

\begin{Code}
: FIRST  ( -- charge)
  1MINUTE  ?ASSISTANCE   MILEAGE-CHARGE + ;
: PER-ADDITIONAL  ( -- charge)
   +MINUTES  MILEAGE-CHARGE + ;
\end{Code}
What is the rule for the mileage charge? Very simple. It is the rate
(per hundred miles) times the number of miles (in hundreds). Let's
assume we can define the word \forth{MILEAGE-RATE}, which will fetch
the mileage rate from the table:

\begin{Code}
: MILEAGE-CHARGE  ( -- charge)
   #MILES @  MILEAGE-RATE * ;
\end{Code}
Finally, if we know the total number of minutes for a call, we can now
calculate the total direct-dial charge:

\begin{Code}
: TOTAL   ( -- total-charge)
   FIRST                        ( first minute rate)
   ( #minutes) 1-               ( additional minutes)
      PER-ADDITIONAL *          ( times the rate)
   +  ;                         ( added together)
\end{Code}
We've expressed the rules to this particular problem through a
combination of simple tables and logical calculations.

(Some final notes on this example: We've written something very close
to a running \Forth{} application. But it is only pseudocode. We've
avoided stack manipulations by assuming that values will somehow be on
the stack where the comments indicate. Also, we've used hyphenated
names because they might be more readable for the customer. Short
names are preferred in real code---see \Chap{5}.)\label{phone2}

We'll unveil the finished code for this example in \Chap{8}.%
\index{A!Analysis!decision table|)}%
\index{A!Analysis!rule definition|)}%
\index{C!Conceptual model!rule definition|)}%
\index{D!Decision table|)}%
\index{R!Rule definition|)}

\section{Defining the Data Structures}%
\index{A!Analysis!conceptual model|(}%
\index{A!Analysis!data structure definition|(}%
\index{C!Conceptual model|)}%
\index{C!Conceptual model!data structure definition|(}%
\index{D!Data structures:!defining|(}

After defining the interfaces, and sometimes defining the rules,
occasionally you'll need to define certain data structures as well.
We're not referring here to the implementation of the data structures,
but rather to a description of their conceptual model.

If you're automating a library index, for instance, a crucial portion
of your analysis will concern developing the logical data structure.
You'll have to decide what information will be kept for each book:
title, author, subject, etc. These ``attributes'' will comprise an
``entity'' (set of related records) called \forth{BOOKS}. Then you'll
have to determine what other data structures will be required to let
the users search the \forth{BOOKS} efficiently.

%Page 060 in first edition

\wepsfigp{img2-060}{Given two adequate solutions,
the correct one is the simpler.}

%Page 061 in first edition

You may need another entity consisting of authors' names in
alphabetical order, along with ``attribute pointers'' to the books
each author has written.

Certain constraints will also affect the conceptual model of the data
structure. In the library index example, you need to know not only
\emph{what} information the users need, but also how long they're
willing to \emph{wait} to get it.

For instance, users can request listings of topics by year of
publication---say everything on ladies' lingerie between 1900 and
1910.  If they expect to get this information in the snap of a girdle,
you'll have to index on years and on topics. If they can wait a day,
you might just let the computer search through all the books in the
library.%
\index{A!Analysis!conceptual model|)}%
\index{A!Analysis!data structure definition|)}%
\index{C!Conceptual model!data structure definition|)}%
\index{D!Data structures:!defining|)}

\section{Achieving Simplicity}%
\index{A!Analysis!simplicity|(}%
\index{S!Simplicity|(}

\begin{tip}
Keep it simple.
\end{tip}

\noindent While you are taking these crucial first steps toward
understanding the problem, keep in mind the old saying:

\begin{tfquot}
Given two solutions to a problem, the correct one is the simpler.
\end{tfquot}
This is especially true in software design. The simpler solution is often
more difficult to discover, but once found, it is:

\begin{itemize}
\item easier to understand
\item easier to implement
\item easier to verify and debug
\item easier to maintain
\item more compact
\item more efficient
\item more fun
\end{itemize}

\begin{interview}
One of the most compelling advocates of simplicity is
\person{Moore}\index{M!Moore, Charles|(}:

\begin{tfquot}
You need a feeling for the size of the problem. How much code should
it take to implement the thing? One block? Three? I think this is a
very useful design tool. You want to gut-feel whether it's a trivial
problem or a major problem, how much time and effort you should spend
on it.

When you're done, look back and say, ``Did I come up with a solution
that is reasonable?'' If your solution fills six screens, it may seem
you've used a sledgehammer to kill a mosquito. Your mental image is
out of proportion to the significance of the problem.

I've seen nuclear physics programs with hundreds of thousands of lines
of FORTRAN. Whatever that code does, it doesn't warrant hundreds of
%Page 062 in first edition
thousands of lines of code. Probably its writers have overgeneralized
the problem. They've solved a large problem of which their real needs
are a subset. They have violated the principle that the solution
should match the problem.
\end{tfquot}\index{M!Moore, Charles|)}
\end{interview}
\begin{tip}
Generality usually involves complexity. Don't generalize your solution any
more than will be required; instead, keep it changeable.
\end{tip}
\begin{interview}
\person{Moore} continues:\index{M!Moore, Charles|(}
\begin{tfquot}

Given a problem, you can code a solution to it. Having done that, and found
certain unpleasantnesses to it, you can go back and change the problem,
and end up with a simpler solution.

There's a class of device optimization---minimizing the number of gates in a
circuit-where you take advantage of the ``don't care'' situation. These
occur either because a case won't arise in practice or because you really
don't care. But the spec is often written by people who have no appreciation
for programming. The designer may have carefully specified all the cases,
but hasn't told you, the programmer, which cases are really important.

If you are free to go back and argue with him and take advantage of the
``don't cares,'' you can come up with a simpler solution.

Take an engineering application, such as a 75-ton metal powder press,
stamping out things. They want to install a computer to control the
valves in place of the hydraulic control previously used. What kind of
spec will you get from the engineer? Most likely the sensors were
placed for convenience from an electromechanical standpoint. Now they
could be put somewhere else, but the engineer has forgotten. If you
demand explanations, you can come closer to the real world and further
from their model of the world.

Another example is the PID (proportional integration and
differentiation) algorithm for servos. You have one term that
integrates, another term that differentiates, and a third term that
smooths. You combine those with 30\% integration, 10\%
differentiation, or whatever. But it's only a digital filter. It used
to be convenient in analog days to break out certain terms of the
digital filter and say, ``This is the integrator and this is the
differentiator. I'll make this with a capacitor and I'll make that
with an inductor.''

Again the spec writers will model the analog solution which was
modeling the electromechanical solution, and they're several models
away from reality. In fact, you can replace it all with two or three
coefficients in a digital filter for a much cleaner, simpler and more
efficient solution.
\end{tfquot}\index{M!Moore, Charles|)}
\end{interview}

\begin{tip}
Go back to what the problem was before the customer tried to solve it.
Exploit the ``don't cares.''
\end{tip}

%Page 063 in first edition

\wepsfigp{img2-063}{An overgeneralized solution.}

%Page 064 in first edition

\begin{interview}
\person{Moore} continues:\index{M!Moore, Charles|(}

\begin{tfquot}
Sometimes the possibilities for simplification aren't immediately
obvious.

There's this problem of zooming in a digitized graphics display, such
as CAD systems. You have a picture on the screen and you want to zoom
in on a portion to see the details.

I used to implement it so that you move the cursor to the position of
interest, then press a button, and it zooms until you have a window of
the desired size. That was the way I've always done it. Until I
realized that that was stupid. I never needed to zoom with such fine
resolution.

So instead of moving the cursor a pixel at a time, I jump the cursor
by units of, say, ten. And instead of increasing the size of box, I
jump the size of the box. You don't have a choice of sizes. You zoom
by a factor of four. The in-between sizes are not interesting. You can
do it as many times as you like.

By quantizing things fairly brutally, you make it easier to work with,
more responsive, and simpler.
\end{tfquot}\index{M!Moore, Charles|)}
\end{interview}

\begin{tip}
To simplify, quantize.\index{Q!Quantization}
\end{tip}

\begin{interview}
\person{Moore} concludes:\index{M!Moore, Charles|(}%
\begin{tfquot}
It takes arrogance to go back and say ``You didn't really mean this,''
or ``Would you mind if I took off this page and replaced it with this
expression?'' They get annoyed. They want you to do what they told you
to do.

\index{S!Stuart, LaFarr|(}
\person{LaFarr Stuart} took this attitude when he redesigned \Forth{}
\cite{stuart80}. He didn't like the input buffer, so he implemented
\Forth{} without it, and discovered he didn't really need an input buffer.%
\index{S!Stuart, LaFarr|)}

If you can improve the problem, it's a great situation to get into.
It's much more fun redesigning the world than implementing it.
\end{tfquot}\index{M!Moore, Charles|)}
\end{interview}

\noindent Effective programmers learn to be tactful and to couch their
approaches in non-threatening ways: ``What would be the consequences
of replacing that with this?'' etc.

Yet another way to simplify a problem is this:

\begin{tip}
To simplify, keep the user out of trouble.
\end{tip}
\noindent Suppose you're designing part of a word processor that
displays a directory of stored documents on the screen, one per line.
You plan that the
%Page 065 in first edition
user can move the cursor next to the name of any document, then type a
one-letter command indicating the chosen action: ``p'' for print,
``e'' for edit, etc.

Initially it seems all right to let the user move the cursor anywhere
on the screen. This means that those places where text already appears
must be protected from being overwritten. This implies a concept of
``protected fields'' and special handling. A simpler approach confines
the cursor to certain fields, possibly using reverse video to let the
user see the size of the allowable field.

Another example occurs when an application prompts the user for a
numeric value. You often see such applications that don't check input
until you press ``return,'' at which time the system responds with an
error message such as ``invalid number.'' It's just as easy---probably
easier---to check each key as it's typed and simply not allow
non-numeric characters to appear.

\begin{tip}
To simplify, take advantage of what's available.
\end{tip}%
\index{L!LaManna, Michael|(}
\begin{interview}
\person{Michael LaManna}, a \Forth{} programmer in Long Island, New York,
comments:

\begin{tfquot}
I always try to design the application on the most powerful processor
I can get my hands on. If I have a choice between doing development on
a 68000-based system and a 6809-based system, I'd go for the
68000-based system. The processor itself is so powerful it takes care
of a lot of details I might otherwise have to solve myself.

If I have to go back later and rewrite parts of the application for a
simpler processor, that's okay. At least I won't have wasted my time.
\end{tfquot}
\end{interview}%
\index{L!LaManna, Michael|)}

\noindent A word of caution: If you're using an existing component to
simplify your prototype, don't let the component affect your design.
You don't want the design to depend on the internals of the component.%
\index{A!Analysis!simplicity|)}%
\index{S!Simplicity|)}

\section{Budgeting and Scheduling}%
\index{B!Budgeting|(}%
\index{A!Analysis!budgeting|(}%
\index{A!Analysis!scheduling|(}%
\index{S!Scheduling|(}

Another important aspect of the analysis phase is figuring the price
tag.  Again, this process is much more difficult than it would seem.
If you don't know the problem till you solve it, how can you possibly
know how long it will take to solve it?

Careful planning is essential, because things always take longer than
you expect. I have a theory about this, based on the laws of
probability:

%Page 066 in first edition

\wepsfigp{img2-066}{Conventional wisdom reveres complexity.}

%Page 067 in first edition
\begin{tip}
The mean time for making a ``two-hour'' addition to an application is
approximately 12 hours.
\end{tip}

\noindent Imagine the following scenario: You're in the middle of
writing a large application when suddenly it strikes you to add some
relatively simple feature. You think it'll take about two hours, so
without further planning, you just do it. Consider: That's two hours
coding time. The design time you don't count because you perceived the
need---and the design---in a flash of brilliance while working on the
application. So you estimate two hours.

But consider the following possibilities:

\begin{enumerate}
\item Your implementation has a bug. After two hours it doesn't work.
So you spend another two hours recoding. (Total 4.)
\item OR, before you implemented it, you realized your initial design
wouldn't work. You spend two hours redesigning. \emph{These} two hours
count. Plus another two hours coding it. (Total 4.)
\item OR, you implement the first design before you realize the design
wouldn't work. So you redesign (two more hours) and reimplement (two
more). (Total 6.)
\item OR, you implement the first design, code it, find a bug, rewrite
the code, find a design flaw, redesign, recode, find a bug in the new
code, recode again. (Total 10.)
\suspend{enumerate}
You see how the thing snowballs?
\resume{enumerate}
\item Now you have to document your new feature. Add two hours to the
above. (Total 12.)
\item After you've spent anywhere from 2 to 12 hours installing and
debugging your new feature, you suddenly find that element Y of your
application bombs out. Worst yet, you have no idea why. You spend two
hours reading memory dumps trying to divine the reason. Once you do,
you spend as many as 12 additional hours redesigning element Y. (Total
26.) Then you have to document the syntax change you made to element
Y. (Total 27.)
\end{enumerate}

\noindent That's a total of over three man-days. If all these mishaps
befell you at once, you'd call for the men with the little white
coats. It rarely gets that bad, of course, but the odds are decidedly
\emph{against} any project being as easy as you think it will be.

How can you improve your chances of judging time requirements
correctly? Many fine books have been written on this topic, notably
\emph{The Mythical Man-Month} by \person{Frederick P. Brooks}, Jr.
\cite{brooks75}.%
\index{B!Brooks,Fredrick P., Jr.}%
\index{M!Mythical@\emph{Mythical Man-Month, The} Brooks}
I have little to add to this body of knowledge except for some
personal observations.

\begin{enumerate}

\item Don't guess on a total. Break the problem up into the smallest
possible pieces, then estimate the time for each piece. The sum of the
pieces is always greater than what you'd have guessed the total would
be. (The whole appears to be less than the sum of the parts.)

%Page 068 in first edition
\item In itemizing the pieces, separate those you understand well
enough to hazard a guess from those you don't. For the second
category, give the customer a range.

\item A bit of psychology: always give your client some options.
Clients \emph{like} options. If you say, ``This will cost you \$6,000,''
the client will probably respond ``I'd really like to spend \$4,000.''
This puts you in the position of either accepting or going without a job.

But if you say, ``You have a choice: for \$4,000 I'll make it
\emph{walk} through the hoop; for \$6,000 I'll make it \emph{jump}
through the hoop. For \$8,000 I'll make it \emph{dance} through the
hoop waving flags, tossing confetti and singing ``Roll Out the Barrel.''

Most customers opt for jumping through the hoop.
\end{enumerate}

\begin{tip}
Everything takes longer than you think, including thinking.
\end{tip}%
\index{A!Analysis!budgeting|)}%
\index{A!Analysis!scheduling|)}%
\index{B!Budgeting|)}%
\index{S!Scheduling|)}

\section{Reviewing the Conceptual Model}%
\index{A!Analysis!conceptual model|(}%
\index{C!Conceptual model|(}

The final box on our iterative analytic wheel is labeled ``Show Model
to Customer.'' With the tools we've outlined in this chapter, this job
should be easy to do.

In documenting the requirements specification, remember that specs are
like snowmen. They may be frozen now, but they shift, slip, and melt
away when the heat is on. Whether you choose data-flow diagrams or
straight \Forth{} pseudocode, prepare yourself for the great thaw by
remembering to apply the concepts of limited redundancy.

Show the documented conceptual model to the customer. When the
customer is finally satisfied, you're ready for the next big step: the
design!%
\index{A!Analysis!conceptual model|)}%
\index{C!Conceptual model|)}

\begin{references}{9}
\bibitem{harris81} \person{Kim Harris}, ``The \Forth{} Philosophy,''
  \emph{Dr. Dobb's Journal,} Vol. 6, Iss. 9, No. 59 (Sept. 81),
  pp. 6-11.
\bibitem{weinberg80} \person{Victor Weinberg}, \emph{Structured Analysis,}
  Englewood Cliffs, N.J.: Prentice-Hall, Inc., 1980.
\bibitem{stuart80} \person{LaFarr Stuart}, ``LaFORTH'',
  \emph{1980 FORML Proceedings,} p. 78.
\bibitem{brooks75} \person{Frederick P. Brooks}, Jr., \emph{The Mythical
  Man-Month,} Reading, Massachusetts, Addison-Wesley, 1975.
\end{references}

% end of chapter two

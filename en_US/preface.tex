\chapter*{Preface}
\pagestyle{headings}

\initial Programming computers can be crazy-making. Other professions give
you the luxury of seeing tangible proof of your efforts. A watchmaker
can watch the cogs and wheels; a seamstress can watch the seams come
together with each stitch. But programmers design, build, and repair
the stuff of imagination, ghostly mechanisms that escape the senses.
Our work takes place not in RAM, not in an editor, but within our
own minds.

Building models in the mind is both the challenge and the joy of programming.
How should we prepare for it? Arm ourselves with better debuggers,
decompilers, and disassemblers? They help, but our most essential
tools and techniques are mental. We need a consistent and practical
methodology for \emph{thinking about} software problems. That is what
I have tried to capture in this book. \emph{Thinking \Forth{}} is meant
for anyone interested in writing software to solve problems. It focuses
on design and implementation; deciding what you want to accomplish,
designing the components of the system, and finally building the program.

The book stresses the importance of writing programs that not only
work, but that are also readable, logical, and that express the best
solution in the simplest terms.

Although most of the principles described here can be applied to any
language, I've presented them in the context of \Forth{}. \Forth{} is
a language, an operating system, a set of tools, and a philosophy. It
is an ideal means for thinking because it corresponds to the way our
minds work. \emph{Thinking \Forth{}} is thinking simple, thinking
elegant, thinking flexible. It is \emph{not} restrictive, \emph{not}
complicated, \emph{not} over-general.  You don't have to know \Forth{}
to benefit from this book. \emph{Thinking \Forth{}} synthesizes the
\Forth{} approach with many principles taught by modern computer
science. The marriage of \Forth{}'s simplicity with the traditional
disciplines of analysis and style will give you a new and better way
to look at software problems and will be helpful in all areas of
computer application.

If you want to learn more about \Forth{}, another book of mine,
\emph{Starting \Forth{}}, covers the language aspects of
\Forth{}. Otherwise, \App{A} of this book introduces \Forth{} fundamentals.

A few words about the layout of the book: After devoting the first
chapter to fundamental concepts, I've patterned the book after the
software development cycle; from initial specification up through
implementation. The appendixes in back include an overview of \Forth{}
for those new to the language, code for several of the utilities described,
answers to problems, and a summary of style conventions.

Many of the ideas in this book are unscientific. They are based on
subjective experience and observations of our own humanity. For this
reason, I've included interviews with a variety of \Forth{}
professionals, not all of whom completely agree with one another, or
with me. All these opinions are subject to change without notice. The
book also offers suggestions called ``tips.'' They are meant to be
taken only as they apply to your situation. \Forth{} thinking accepts
no inviolable rules.  To ensure the widest possible conformity to
available \Forth{} systems, all coded examples in this book are
consistent with the \Forth{}-83 Standard.

One individual who greatly influenced this book is the man who
invented \Forth{}, \person{Charles Moore}. In addition to spending
several days interviewing him for this book, I've been privileged to
watch him at work. He is a master craftsman, moving with speed and
deftness, as though he were physically altering the conceptual models
inside the machine---building, tinkering, playing. He accomplishes
this with a minimum of tools (the result of an ongoing battle against
insidious complexity) and few restrictions other than those imposed by
his own techniques. I hope this book captures some of his
wisdom. Enjoy!


\subsection{Acknowledgments}

Many thanks to all the good people who gave their time and ideas to
this book, including: \person{Charles Moore}, Dr.\@ \person{Mark
Bernstein}, \person{Dave Johnson}, \person{John Teleska},
Dr.\@ \person{Michael Starling}, Dr.\@ \person{Peter Kogge}, \person{Tom
Dowling}, \person{Donald Burgess}, \person{Cary Campbell},
Dr.\@ \person{Raymond Dessy}, \person{Michael Ham}, and \person{Kim
Harris}. Another of the interviewees, \person{Michael LaManna}, passed
away while this book was in production. He is deeply missed by those
of us who loved him.


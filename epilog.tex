%% Thinking Forth
%% Copyright (C) 2004 Leo Brodie
%% Initial transcription by Ed Beroset
%% 
%% Chapter: Epilog
% \epilog{}  - it's not an Appendix, but not a numbered chapter, either
\part{Epilog}
\chapter{
FORTH's Effect
on Thinking}
\begin{tfquot}
FORTH is like the Tao: it is a Way, and is realized when followed.  Its
fragility is its strength; its simplicity is its direction {\em (Michael Ham,
winning entry in Mountain View Press's contest to describe FORTH in twenty-five 
words or less).}
\end{tfquot}

To help extract something of the FORTH philosophy, I conducted a poll
among several FORTH users in which I asked, ``How has FORTH affected
your thinking?  Have you found yourself applying `FORTH-like' 
principles in other areas?''

Here are some of the replies:

Mark Berstein is president of Eastgate Systems Inc. in Cambridge, 
Massachusetts, and holds a doctorate from the department of 
chemistry at Harvard University.

\begin{tfquot}
I first met FORTH while working in laser chemistry.  I was trying to build a
rather complicated controller for a new laser spectrometer.  The original
plans called for a big green box full of electronics, The Interface.  Nobody 
had built this particular kind of instrument before---that's why we were 
doing it---and the list of things we wanted the computer to handle changed
every couple of weeks.

After a few months, I had hundreds of pages of assembly-language 
routines, three big circuit boards filled with ICs, and a 70-odd pin System
Bus.  Day by day, everything got more fragile and harder to fix.  The wiring 
on the circuit boards frayed, the connectors got loose, the assembler code
grew ever more tangled.

FORTH was an obvious solution to the software problem, since it provided
a decent environment in which to build and maintain a complex and
rapidly-changing program.  But the essence of good FORTH programming is the 
art of factoring procedures into useful, free-standing words.  The idea of the
FORTH word had unexpected implications for laboratory hardware design.

Instead of building a big, monolithic, all-purpose Interface, I found myself
building piles of simple little boxes which worked a lot like FORTH words:
they had a fixed set of standard inputs and standard outputs, they performed
just one function, they were designed to connect up to each other 
without much effort, and they were simple enough that you could tell what 
a box did just by looking at its label.

\ldots The idea of ``human scale'' is, I think, today's seminal concept in 
software design.  This isn't specifically a FORTH development; the great joy of
UNIX, in its youth at least, was that you could read it (since it was written
in C), understand it (since it was small), and modify it (since it was simple).
FORTH shares these virtues, although it's designed to tackle a different 
sort of problem.

Because FORTH is small, and because FORTH gives its users control over
their machines, FORTH lets humans control their applications.  It's just
silly to expect scientists to sit in front of a lab computer playing
``twenty-questions'' with packaged software.  FORTH, used properly, lets a scientist
instruct the computer instead of letting the computer instruct the scientist.

In the same sense that in baseball, a batter is supposed to feel the bat as an 
extension of himself, FORTH is human-scaled, and helps convince you that
the computer's achievements, and its failures, are also your own.
\end{tfquot}
Raymond E. Dessy is Professor of Chemistry at Virginia Polytechnic Institute
and State University, Blacksburg, Virginia.
\begin{tfquot}
As I attempted to understand the nature and structure of the language C, I 
found myself drawing upon the knowledge I had of the organization and
approach of FORTH.  This permitted me to understand convoluted, or
high-fog-coefficient sections describing C.

I have found the FORTH approach is an ideal platform upon which to build
an understanding and an educational framework for other languages and
operating system concepts.
\end{tfquot}
Jerry Boutelle is owner of Nautilus Systems in Santa Cruz, California, 
which markets the Nautilus Cross-compiler.
\begin{tfquot}
FORTH has changed my thinking in many ways.  Since learning FORTH 
I've coded in other languages, including assembler, BASIC and FORTRAN.
I've found that I used the same kind of decomposition we do in
FORTH, in the sense of creating words and grouping them together.  For
example, in handling strings I would define subroutines analogous to
CMOVE, --TRAILING, FILL, etc.

More fundamentally, FORTH has reaffirmed my faith in simplicity.  Most 
people go out and attack problems with complicated tools.  But simpler
tools are available and more useful.

I try to simplify all the aspects of my life.  There's a quote I like
from {\em Tao Te Ching}
by the Chinese philosopher Lao Tzu: ``To attain knowledge, add 
things every day; to obtain wisdom, remove things every day.''
\end{tfquot}

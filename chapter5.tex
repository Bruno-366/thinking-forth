%% Thinking Forth
%% Copyright (C) 2004 Leo Brodie
%% Initial transcription by Albert van der Horst
%%
%% Chapter: Implementation: Elements of FORTH Style

\chapter{Implementation:
Elements~of FORTH~Style}\Chapmark{5}

Badly written FORTH has been accused of looking like ``code that went
through a trash compactor.'' It's true, FORTH affords more freedom in
the way we write applications.  But that freedom also gives us a chance to
write exquisitely readable and easily maintainable code, provided we consciously
employ the elements of good FORTH style.

In this chapter we'll delve into FORTH coding convention
including:
%!!<ENUMERATION>
\begin{tfquot}
listing organization

screen layout, spacing and indentation

commenting

choosing names.
\end{tfquot}
%!!</ENUMERATION>
I wish I could recommend a list of hard-and-fast conventions for
everyone.  Unfortunately, such a list may be inappropriate in many situations.
This chapter merges many widely-adopted conventions with personal
preferences, commented with alternate ideas and the reasons for
the preferences.  In other words:
\begin{Code}
: TIP VALUE JUDGEMENT ;
\end{Code}
I'd especially like to thank Kim Harris, who proposed many of the conventions
described in this chapter, for his continuing efforts at unifying
divergent views on good FORTH style.

\section{Listing Organization}

A well-organized book has clearly defined chapters, with clearly defined
sections, and a table of contents to help you see the organization at a
glance.  A well-organized book is easy to read.  A badly organized book
makes comprehension more difficult, and makes finding information
later on nearly impossible.

The necessity for good organization applies to an application listing
as well.  Good organization has three aspects:
%Page 136 in first edition.
%!!<ILLUSTRATION>
\wepsfigp{fig5-1}{I still don't see how these programming conventions enhance readability.}
%!!</ILLUSTRATION>

%!!<ENUMERATION>
\begin{enumerate}
\item Decomposition
\item Composition
\item Disk partitioning
\end{enumerate}
%!!</ENUMERATION>
\subsection{Decomposition}

As we've already seen, the organization of a listing should follow the
decomposition of the application into lexicons.  Generally these lexicons
should be sequenced in ``uses'' order.  Lexicons being used should precede
the lexicons which use them.

On a larger scale, elements in a listing should be organized by
degree of complexity, with the most complex variations appearing
towards the end.  It's best to arrange things so that you can leave off the
lattermost screens (i.e., not load them) and still have a self-sufficient,
running application, working properly except for the lack of the more advanced
features.

We discussed the art of decomposition extensively in \Chap{3}.

\subsection{Composition}

Composition is the putting together of pieces to create a whole.  Good
composition requires as much artistry as good decomposition.

One of FORTH's present conventions is that source code resides in
``screens,'' which are 1K units of mass storage.  (The term ``screen'' refers
to a block used specifically for source code.) It's possible in FORTH to
chain every screen of code to the next, linking the entire listing together
linearly like a lengthy parchment scroll.  This is not a useful approach.
Instead:

\begin{tip}
Structure your application listing like a book: hierarchically.
\end{tip}

An application may consist of:
%!!<ENUMERATION>
\begin{description}
\item[Screens:] the smallest unit of FORTH source
\item[Lexicons:] one to three screens, enough to implement a component
\item[Chapters:] a series of related lexicons, and
\item[Load screens:] analogous to a table of contents, a screen that loads the
chapters in the proper sequence.
\end{description}
%!!</ENUMERATION>
%Page 138 in first edition.
%!!<Figure>
\begin{figure*}[tttt]
\caption{Example of an application-load screen.}
\labelfig{fig5-1}
\setcounter{screen}{1}
\begin{Screen}
\ QTF+ Load Screen   07/09/83
: RELEASE#   ." 2.01" ;
  9 LOAD \ compiler tools, language primitives
 12 LOAD \ video primitives
 21 LOAD \ editor
 39 LOAD \ line display
 48 LOAD \ formatter
 69 LOAD \ boxes
 81 LOAD \ deferring
 90 LOAD \ framing
 96 LOAD \ labels, figures, tables
102 LOAD \ table of contents generator




\end{Screen}
\end{figure*}
%!!</Figure>
%% Note: AH: I have added line 15, count that as a typo.

\subsection{Application-load Screen}

\Fig{fig5-1} is an example of an application-load screen.  Since it resides in
Screen 1, you can load this entire application by entering
\begin{Code}
1 LOAD
\end{Code}
The individual load commands within this screen load the chapters of the
application.  For instance, Screen 12 is the load screen for the video
primitives chapter.

As a reference tool, the application-load screen tells you where to
find all of the chapters.  For instance, if you want to look at the routines
that do framing, you can see that the section starts at Screen 90.

Each chapter-load screen in turn, loads all of the screens comprising
the chapter.  We'll study some formats for chapter-load screens shortly.

The primary benefit of this hierarchical scheme is that you can load
any section, or any screen by itself, without having to load the entire application.
Modularity of the source code is one of the reasons for
FORTH's quick turnaround time for editing, loading, and testing
(necessary for the iterative approach).  Like pages of a book, each screen
can be accessed individually and quickly.  It's a ``random access'' approach
to source-code maintenance.

You can also replace any passage of code with anew, trial version by
simply changing the screen numbers in the load screen.  You don't have to
move large passages of source code around within a file.

In small applications, there may not be such things as chapters.  The
application-load screen will directly load all the lexicons.  In larger applications,
however, the extra level of hierarchy can improve maintainability.
%Page 139 in first edition.

A screen should either be a load-screen or a code-screen, not a mixture.
Avoid embedding a LOAD or THRU command in the middle of a
screen containing definitions just because you ``need something'' or
because you ``ran out of room.''

\subsection{Skip Commands}
\index{S!Skip commands}
Two commands make it easy to control what gets loaded in each screen
and what gets ignored.  They are:
%!!<ENUMERATION>
\begin{description}
\item[\(\backslash\)]
\item[\(\backslash\){}S] also called EXIT
\end{description}
%!!</ENUMERATION>
$\backslash${} is pronounced ``skip-line.'' It causes the FORTH interpreter to ignore
everything to the right of it on the same line.  (Since $\backslash${} is a FORTH word, it
must be followed by a space.) It does not require a delimiter.

In \Fig{fig5-1}, you see $\backslash${} used in two ways: to begin the screen-comment
line (Line 0), and to begin comments on individual lines which
have no more code to the right of the comment.

During testing, $\backslash${} also serves to temporarily ``paren out'' lines that
already contain a right parenthesis in a name or comment.  For instance,
these two ``skip-line''s keep the definition of NUTATE from being compiled
without causing problems in encountering either right parenthesis:
\begin{Code}
\ : NUTATE (x y z )
\ SWAP ROT (NUTATE);
\end{Code}
$\backslash${}S is pronounced ``skip-screen.'' It causes the FORTH interpreter to stop
interpreting the screen entirely, as though there were nothing else in the
screen beyond $\backslash${}S.

In many FORTH systems, this function is the same as EXIT, which
is the run-time routine for semicolon.  In these systems the use of EXIT is
acceptable.  Some FORTH systems, however, require for internal reasons
a different routine for the ``skip-screen'' function.

Definitions for $\backslash${} and $\backslash${}S can be found in \Chap{C}.

\subsection{Chapter-load Screens}

\Fig{fig5-2} illustrates a typical chapter-load screen.  The screens loaded by
this screen are referred to relatively, not absolutely as they were in the
application-load screen.

This is because the chapter-load screen is the first screen of the contiguous
range of screens in the chapter.  You can move an entire chapter
forward or backward within the listing; the relative pointers in the
chapter-load screen are position-independent.  All you have to change is
%Page 140 in first edition.
%!!</Figure>
\begin{figure*}
\caption{Example of a chapter-load screen.}
\labelfig{fig5-2}

\setcounter{screen}{100}
\begin{Screen}
\ GRAPHICS                 Chapter load                 07/11/83

1 FH LOAD             \ dot-drawing primitive
2 FH 3 FH THRU        \ line-drawing primitives
4 FH 7 FH THRU        \ scaling, rotation
8 FH LOAD             \ box
9 FH 11 FH THRU       \ circle



CORNER  \ initialize relative position to low-left corner





\end{Screen}
\end{figure*}
%!!</Figure>
the single number in the application-load screen that points to the beginning
of the chapter.

\begin{tip}
Use absolute screen numbers in the application-load screen.  Use relative
screen numbers in the chapter- or section-load screens.
\end{tip}
There are two ways to implement relative loading.  The most common is
to define:
%!!<FORTHCODE>
\begin{Code}
: +LOAD  ( offset -- )  BLK @ +  LOAD;
\end{Code}
and
\begin{Code}
: +THRU  ( la-offset hi-offset -- )
     1+ SWAP DO  I +LOAD  LOOP;
\end{Code}

My own way, which I submit as a more useful factoring, requires a single
word, FH (see \Chap{C} for its definition).

The phrase
\begin{Code}
1 FH LOAD
\end{Code}
is read ``1 from here LOAD,'' and is equivalent to 1 + LOAD.

Similarly,
\begin{Code}
2 FH   5 FH THRU
\end{Code}
is read ``2 from here, 5 from here THRU.''

Some programmers begin each chapter with a dummy word; e.g.,
\begin{Code}
: VIDEO-IO ;
\end{Code}
%Page 141 in first edition.
and list its name in the comment on the line where the chapter is loaded in
the application-load screen.  This permits selectively FORGETting any
chapter and reloading from that point on without having to look at the
chapter itself.

Within a chapter the first group of screens will usually define those
variables, constants, and other data structures needed globally within
the chapter.  Following that will come the lexicons, loaded in ``uses''
order.  The final lines of the chapter-load screen normally invoke any
needed initialization commands.

Some of the more style-conscious FORTHwrights begin each
chapter with a ``preamble'' that discusses in general terms the theory of
operation for the components described in the chapter.  \Fig{fig5-3} is a
sample preamble screen which demonstrates the format required at
Moore Products Co.

%!!<Figure>
\begin{figure*}[tttt]
\caption{Moore Products Co.'s format for chapter preambles.}
\labelfig{fig5-3}
\begin{Screen}
CHAPTER 5  -  ORIGIN/DESTINATION - MULTILOOP BIT ROUTINES

DOCUMENTS  - CONSOLE STRUCTURE CONFIGURATION
        DESIGN SPECIFICATION
        SECTIONS - 3.2.7.5.4.1.2.8
                   3.2.7.5.4.1.2.10

ABSTRACT - File control types E M T Q and R can all
           originate from a Regional Satellite or a
           Data Survey Satellite.  These routines allow
           the operator to determine whether the control
           originated from a Regional Satellite or not.

\end{Screen}

\begin{Screen}
CHAPTER NOTES - Whether or not a point originates from
                a Regional Satellite is determined by
                the Regional bit in BITS, as follows:

                  1 = Regional Satellite
                  2 = Data Survey Satellite

                 For the location of the Regional bit
                 in BITS, see the Design Specification
                 Section - 3.2.7.5.4.1.2.10

HISTORY  -

\end{Screen}
\end{figure*}
%!!</Figure>
%Page 142 in first edition.

Charles Moore\index{M!Moore, Charles|(} (no relation to Moore Products
Co.) places less importance on the well-organized hierarchical listing
than I do.  Moore:

\begin{tfquot}
I structure \emph{applications} hierarchically, but not necessarily
listings.  My listings are organized in a fairly sloppy way, not at
all hierarchically in the sense of primitives first.

I use LOCATE {[}also known as VIEW; see the Handy Hint in Starting
FORTH, Chapter Nine{]}.  As a result, the listing is much less carefully
organized because I have LOCATE to find things for me.  I never look
at listings.
\end{tfquot}\index{M!Moore, Charles|)}


\subsection{--\/--> vs.\ THRU}

On the subject of relative loading, one popular way to load a series of adjacent
screens is with the word --\/--> (pronounced ``next block'').  This word
causes the interpreter to immediately cease interpreting the current
screen and begin interpreting the next (higher-numbered) screen.

If your system provides --\/-->, you must choose between using the
THRU command in your chapter-load screen to load each series of
screens, or linking each series together with the arrows and LOADing
only the first in the series.  (You can't do both; you'd end up loading most
of the screens more than once.)

The nice thing about the arrows is this: suppose you change a screen
in the middle of a series, then reload the screen.  The rest of the series will
automatically get loaded.  You don't have to know what the last screen is.

That's also the nasty thing about the arrows: There's no way to stop
the loading process once it starts.  You may compile a lot more screens
than you need to test this one screen.

To get analytical about it, there are three things you might want to
do after making the change just described:
%!!<ENUMERATION>
\begin{enumerate}
\item load the one screen only, to test the change,
\item load the entire section in which the screen appears,
or
\item load the entire remainder of the application.
\end{enumerate}
%!!</ENUMERATION>
The use of THRU seems to give you the greatest control.

Some people consider the arrow to be useful for letting definitions
cross screen boundaries.  In fact --\/--> is the only way to compile a high-level
(colon) definition that occupies more than one screen, because --\/--> is
``immediate.'' But it's NEVER good style to let a colon definition cross
screen boundaries.  (They should never be that long!)

On the other hand, an extremely complicated and time-critical piece
of assembler coding might occupy several sequential screens.  In this
case, though, normal LOADing will do just as well, since the assembler
does not use compilation mode, and therefore does not require
immediacy.

Finally, the arrow wastes an extra line of each source screen.  We
don't recommend it.
%Page 143 in first edition.
\section{An Alternative to Screens: Source in Named Files}

Some FORTH practitioners advocate storing source code in variable-length,
named text files, deliberately emulating the approach used by
traditional compilers and editors.  This approach may become more and
more common, but its usefulness is still controversial.

Sure, it's nice not to have to worry about running out of room in a
screen, but the hassle of writing in a restricted area is compensated for by
retaining control of discrete chunks of code.  In developing an application,
you spend a lot more time loading and reloading screens than you do rearranging
their contents.

``Infinite-length'' files allow sloppy, disorganized thinking and bad
factoring.  Definitions become longer without the discipline imposed by
the 1K block boundaries.  The tendency becomes to write a 20K file, or
worse: a 20K definition.

Perhaps a nice compromise would be a file-based system that allows
nested loading, and encourages the use of very small named files.  Most
likely, though, the more experienced FORTH programmers would not
use named files longer than 5K to 10K.  So what's the benefit?

Some might answer that rhetorical question: ``It's easier to remember
names than numbers.'' If that's so, then predefine those block
numbers as constants, e.g.:
\begin{Code}
90 CONSTANT FRAMING
\end{Code}
Then to load the ``framing'' section, enter
\begin{Code}
FRAMING LOAD
\end{Code}
Or, to list the section's load block, enter
\begin{Code}
FRAMING LIST
\end{Code}
(It's a convention that names of sections end in ``ING.'')

Of course, to minimize the hassle of the screen-based approach you
need good tools, including editor commands that move lines of source
from one screen to another, and words that slide a series of screens forward
or back within the listing.

\subsection{Disk Partitioning}
The final aspect of the well-organized listing involves standardizing an
arrangement for what goes where on the disk.  These standards must be
set by each shop, or department, or individual programmer, depending on
the nature of the work.
%Page 144 in first edition.
%!!<Figure>
\begin{figure*}
\caption{Example of a disk-partitioning scheme within one department.}
\labelfig{fig5-4}

\begin{description}
\item[Screen 0] is the title screen, showing the name of the
    application, the current release number, and primary
    author.
\item[Screen 1] is the application-load block.
\item[Screen 2] is reserved for possible continuation from
\item[Screen 4 and 5] contain system messages.
\item[Screens 9 thru 29] incorporate general utilities needed
    in, but not restricted to, this application.
\item[Screen 30] begins the application screens.
\end{description}
\end{figure*}
%!!</Figure>

\Fig{fig5-4} shows a typical department's partitioning scheme.

In many FORTH shops it's considered desirable to begin sections of
code on screen numbers that are evenly divisible by three.  Major divisions
on a disk should be made on boundaries evenly divisible by thirty.

The reason? By convention, FORTH screens are printed three to a
page, with the top screen always evenly divisible by three.  Such a page is
called a ``triad''; most FORTH systems include the word TRIAD to
produce it, given as an argument the number of any of the three screens
in the triad.  For instance, if you type
\begin{Code}
77 TRIAD
\end{Code}
you'll get a page that includes 75, 76, and 77.

The main benefit of this convention is that if you change a single
screen, you can slip the new triad right into your binder containing the
current listing, replacing exactly one page with no overlapping screens.

Similarly, the word INDEX lists the first line of each screen, 60 per
page, on boundaries evenly divisible by 60.

\begin{tip}
Begin sections or lexicons on screen numbers evenly divisible by three.
Begin applications or chapters on screen numbers evenly divisible by
thirty.
\end{tip}

\subsection{Electives}
Vendors of FORTH systems have a problem.  If they want to include
every command that the customer might expect-words to control
graphics, printers, and other niceties-they often find that the system
has swollen to more than half the memory capacity of the computer, leaving
less room for serious programmers to compile their applications.
%Page 145 in first edition.

The solution is for the vendor to provide the bare bones as a precompiled
nucleus, with the extra goodies provided in source form.  This approach
allows the programmer to pick and choose the special routines
actually needed.

These user-loadable routines are called ``electives.'' Double-length
arithmetic, date and time support, CASE statements and the DOER/
MAKE construct (described later) are some of the features that FORTH
systems should offer as electives.

\section{Screen Layout}

In this section we'll discuss the layout of each source screen.

\begin{tip}
Reserve Line 0 as a ``comment line.''
\end{tip}

The comment line serves both as a heading for the screen, and also as a
line in the disk INDEX.  It should describe the purpose of the screen (not
list the words defined therein).

The comment line minimally contains the name of the screen.  In
larger applications, you may also include both the chapter name and
screen name.  If the screen is one of a series of screens implementing a
lexicon, you should include a ``page number'' as well.

The upper right hand corner is reserved for the ``stamp.'' The stamp
includes the date of latest revision and, when authorship is important,
the programmer's initials (three characters to the left of the date); e.g.:
\begin{Code}
( Chapter name          Screen Name -- pg #       JPJ 06/10/83)
\end{Code}
Some FORTH editors will enter the stamp for you at the press of a key.
A common form for representing dates is
\begin{Code}
mm-dd-yy
\end{Code}
that is, February 6, 1984 would be expressed
\begin{Code}
02-06-84
\end{Code}
An increasingly popular alternative uses
\begin{Code}
ddMmmyy
\end{Code}
where ``Mmm'' is a three-letter abbreviation of the month.  For instance:
\begin{Code}
22Oct84
\end{Code}
%Page 146 in first edition.

This form requires fewer characters than
\begin{Code}
10-22-84
\end{Code}
and eliminates possible confusion between dates and months.

If your system has $\backslash${} (``skip-line''-see \Chap{C}), you can write
the comment line like this:
\begin{Code}
\ Chapter name          Screen Name -- pg.#         JPJ 06/10/83
\end{Code}
As with all comments, use lower-case or a mixture of lower- and uppercase
text in the comment line.

One way to make the index of an application reveal more about the
organization of the screens is to indent the comment line by three spaces
in screens that continue a lexicon.  \Fig{fig5-5} shows a portion of a list produced
by INDEX in which the comment lines for the continuing screens
are indented.

\begin{tip}
Begin all definitions at the left edge of the screen, and define only one word
per lines.
\end{tip}
\emph{Bad:}
%!!<FORTHCODE>
\begin{Code}
: ARRIVING ." HELLO" ;  : DEPARTING ." GOODBYE" ;
\end{Code}
\emph{Good:}
\begin{Code}
: ARRIVING  ." HELLO" ;
: DEPARTING ." GOODBYE" ;
\end{Code}
%!!</FORTHCODE>

This rule makes it easier to find a definition in the listing.  (When definitions
continue for more than one line, the subsequent lines should always
be indented.)
%!!<Figure>
\begin{figure*}
\caption{The output of INDEX showing indented comment lines.}
\labelfig{fig5-5}
\begin{Code}
 90 \ Graphics         Chapter load     JPJ 06/10/83
 91    \ Dot-drawing primitives         JPJ 06/10/83
 92 \ Line-drawing primitives           JPJ 06/11/83
 93    \ Line-drawing primitives        JPJ 06/10/83
 94    \ Line-drawing primitives        JPJ 09/02/83
 95 \ Scaling, rotation                 JPJ 06/10/83
 96    \ Scaling, rotation              JPJ 02/19/84
 97    \ Scaling, rotation              JPJ 02/19/84
 98    \ Scaling, rotation              JPJ 02/19/84
 99 \ Boxes                             JPJ 06/10/83
100 \ Circles                           JPJ 06/10/83
101    \ Circles                        JPJ 06/10/83
102    \ Circles                        JPJ 06/10/83
\end{Code}
\end{figure*}
%!!</Figure>
%Page 147 in first edition.

VARIABLES and CONSTANTS should also be defined one per
line.  (See ``Samples of Good Commenting Style'' in \Chap{E}.) This
leaves room for an explanatory comment on the same line.  The exception
is a large ``family'' of words (defined by a common defining-word) which
do not need unique comments:
\begin{Code}
0 HUE BLACK     1 HUE BLUE      2 HUE GREEN
3 HUE CYAN      4 HUE RED       5 HUE MAGENTA
\end{Code}
\begin{tip}
Leave lots of room at the bottom of the screen for later additions.
\end{tip}

On your first pass, fill each screen no more than half with code.  The
iterative approach demands that you sketch out the components of your
application first, then iteratively flesh them out until all the requirements
are satisfied.  Usually this means adding new commands, or
adding special-case handling, to existing screens.  (Not always, though.  A
new iteration may see a simplification of the code.  Or anew complexity
may really belong in another component and should be factored out, into
another screen.)

Leaving plenty of room at the outset makes later additions more
pleasant.  One writer recommends that on the initial pass, the screen
should contain about 20-40 percent code and 80-60 percent whitespace \cite{stevenson81}.

Don't skip a line between each definition.  You may, however, skip a
line between groups of definitions.'
\begin{tip}
All screens must leave BASE\index{B!BASE} set to DECIMAL.
\end{tip}
Even if you have three screens in a row in which the code is written in
HEX (three screens of assembler code, for instance), each screen must set
BASE to HEX at the top, and restore base to DECIMAL at the bottom.
This rule ensures that each screen could be loaded separately, for purposes
of testing, without mucking up the state of affairs.  Also, in reading
the listing you know that values are in decimal unless the screen explicitly
says HEX.

Some shops take this rule even further.  Rather than brashly resetting
base to DECIMAL at the end, they reset base to whatever it was
at the beginning.  This extra bit of insurance can be accomplished in this
fashion:
\begin{Code}
BASE @          HEX     \ save original BASE on stack
OA2 CONSTANT BELLS
OA4 CONSTANT WHISTLES
... etc. ...
R> BASE !               \ restore it
\end{Code}
%Page 148 in first edition.

Sometimes an argument is passed on the stack from screen to screen,
such as the value returned by BEGIN or IF in a multiscreen assembler
definition, or the base address passed from one defining word to
another-see ``Compile-Time Factoring'' in \Chap{6}.  In these cases,
it's best to save the value of BASE on
the return stack like this:
\begin{Code}
BASE @ >R HEX
... etc. ...
R> BASE !
\end{Code}

Some folks make it a policy to use this approach on any screen that
changes BASE\index{B!BASE}, so they don't have to worry about it.

Moore\index{M!Moore, Charles} prefers to define LOAD to
invoke DECIMAL after loading.
This approach simplifies the screen's contents because you don't have to
worry about resetting.

\subsection{Spacing and Indentation}

\begin{tip}
Spacing and indentation are essential for readability.
\end{tip}
The examples in this book use widely accepted conventions of spacing
and indenting style.  Whitespace, appropriately used, lends readability.
There's no penalty for leaving space in source screens except disk
memory, which is cheap.

For those who like their conventions in black and white, \ref{tab-5-1} is
a list of guidelines.  (But remember, FORTH's interpreter couldn't care
less about spacing or indentation.)
%!!<TABLE>
\begin{table}
\caption{Indentation and spacing guidelines}
\label{tab-5-1}

1 space between the colon and the name\\
2 spaces between the name and the comment\footnotemark[1]\\
2 spaces, or a carriage return, after the comment and before the definition\footnotemark[1]\\
3 spaces between the name and definition if no comment is used\\
3 spaces indentation on each subsequent line (or multiples of 3 for nested
indentation)\\
1 space between words/numbers within a phrase\\
2 or 3 spaces between phrases\\
1 space between the last word and the semicolon\\
1 space between semicolon and IMMEDIATE (if invoked)\\

No blank lines between definitions, except to separate distinct groups of
definitions

%% Note: AH: Again type writer style footnotes.
\end{table}
%!!</TABLE>
%Page 149 in first edition.

The last position of each line should be blank except for:
%!!<ENUMERATION>
\begin{enumerate}
\item[(a)] quoted strings that continue onto the next line, or
\item[(b)] the end of a comment.
\end{enumerate}
%!!</ENUMERATION>
A comment that begins with $\backslash$ may continue right to the end of the line.
Also, a comment that begins with ( may have its delimiting right parenthesis
in the last column.

\footnotetext[1]{An often-seen alternative calls for 1 space between
the name and comment and 3 between the comment and the definition.  A
more liberal technique uses 3 spaces before and after the comment.
Whatever you choose, be consistent.}

Here are some common errors of spacing and indentation:

\emph{Bad}(name not separated from the body of the definition):
\begin{Code}
: PUSH HEAVE HO ;
\end{Code}
\emph{Good:}
\begin{Code}
: PUSH   HEAVE HO ;
\end{Code}
\emph{Bad} (subsequent lines not indented three spaces):
\begin{Code}
: RIDDANCE (thing-never-to-darken-again -- )
DARKEN   NEVER AGAIN;
\end{Code}
\emph{Good:}
\begin{Code}
: RIDDANCE (thing-never-to-darken-again -- )
   DARKEN NEVER AGAIN;
\end{Code}
\emph{Bad} (lack of phrasing):
\begin{Code}
: GETTYSBURG   4 SCORE 7 YEARS + AGO ;
\end{Code}
\emph{Good:}
\begin{Code}
: GETTYSBURG   4 SCORE   7 YEARS +   AGO ;
\end{Code}
Phrasing is a subjective art; I've yet to see a useful set of formal rules.
Simply strive for readability.

\section{Comment Conventions}
Appropriate commenting is essential.  There are five types of comments:
stack-effect comments, data-structure comments, input-stream comments,
purpose comments and narrative comments.

A \emph{stack-effect comment} shows the arguments that the definition
consumes from the stack, and the arguments it returns to the stack, if
any.
%Page 150 in first edition.

A \emph{data-structure comment} indicates the position and meaning of elements
in a data structure.  For instance, a text buffer might contain a
count in the first byte, and 63 free bytes for text.

An \emph{input-stream comment} indicates what strings the word expects
to see in the input stream.  For example, the FORTH word FORGET
scans for the name of a dictionary entry in the input stream.

A \emph{purpose comment} describes, in as few words possible, what the
definition does.  How the definition works is not the concern of the purpose
comment.

A \emph{narrative comment} appears amidst a definition to explain what is
going on, usually line-by-line.  Narrative comments are used only in the
``vertical format,'' which we'll describe in a later section.

Comments are usually typed in lower-case letters to distinguish
them from source code.  (Most FORTH words are spelled with upper-case
letters, but lower-case spellings are sometimes used in special cases.)

In the following sections we'll summarize the standardized formats
for these types of comments and give examples for each type.

\subsection{Stack Notation}
\begin{tip}
Every colon or code definition that consumes and/or returns any arguments
on the stack must include a stack-effect comment.
\end{tip}


``Stack notation'' refers to conventions for representing what's on the
stack.  Forms of stack notation include ``stack pictures,'' ``stack effects,''
and ``stack-effect comments.''

\subsection{Stack Picture}

A stack picture depicts items understood to be on the stack at a given
time.  Items are listed from left to right, with the leftmost item representing
the bottom of the stack and the rightmost item representing the top.

For instance, the stack picture
\begin{Code}
nl n2
\end{Code}
indicates two numbers on the stack, with n2 on the top (the most accessible
position).
This is the same order that you would use to type these values in;
i.e., if n1 is 100 and n2 is 5000, then you would type
\begin{Code}
100 5000
\end{Code}
to place these values correctly on the stack.

%Page 151 in first edition.
A stack picture can include either abbreviations, such as ``n1,'' or
fully spelled-out words.  Usually abbreviations are used.  Some standard
\index{A!Abbreviations!in stack diagrams}
abbreviations appear in \ref{tab-5-2}.  Whether abbreviations or fully
spelled-out words are used, each stack item should be separated by a
space.

If a stack item is described with a phrase (such as ``address-of-latest-link''),
the words in the phrase should be joined by hyphens.  For
example, the stack picture:
\begin{Code}
address current-count max-count
\end{Code}
shows three elements on the stack.

\subsection{Stack Effect}

A ``stack effect'' shows two stack pictures: one picture of any items that
may be consumed by a definition, and another picture of any items
\emph{returned} by the definition.  The ``before'' picture comes first, followed by
two hyphens, then the ``after'' picture.

For instance, the stack effect for FORTH's addition operator, + is
\begin{Code}
n n -- sum
\end{Code}
where + consumes two numbers and returns their sum.

Remember that the stack effect describes only the \emph{net result} of the
operation on the stack.  Other values that happen to reside on the stack
beneath the arguments of interest don't need to be shown.  Nor do values
that may appear or dissappear while the operation is executing.
If the word returns any input arguments unchanged, they should be
repeated in the output picture; e.g.,
\begin{Code}
3rd 2nd top-input -- 3rd 2nd top-output
\end{Code}

Conversely, if the word changes any arguments, the stack comment must
use a different descriptor:
\begin{Code}
nl -- n2
n  -- n'
\end{Code}
A stack effect might appear in a formatted glossary.

\subsection{Stack Effect Comment}

A ``stack-effect comment'' is a stack effect that appears in source code
surrounded by parentheses.  Here's the stack-effect comment for the word
COUNT:
\begin{Code}
( address-of-counted-string -- address-of-text count)
\end{Code}
%Page 152 in first edition.
or:
\begin{Code}
( 'counted-string -- 'text count)
\end{Code}
(The ``count'' is on top of the stack after the word has executed.)

If a definition has no effect on the stack (that is, no effect the user is
aware of, despite what gyrations occur within the definition), it needs no
stack-effect comment:
\begin{Code}
: BAKE   COOKIES OVEN ! ;
\end{Code}
On the other hand, you may want to use an empty stack comment-i.e.,
\begin{Code}
: BAKE  ( -- )  COOKIES OVEN ! ;
\end{Code}
to emphasize that the word has no effect on the stack.
If a definition consumes arguments but returns none, the double-hyphen
is optional.  For instance,
\begin{Code}
( address count -- )
\end{Code}
can be shortened to
\begin{Code}
( address count)
\end{Code}
The assumption behind this convention is this: There are many more
colon definitions that consume arguments and return nothing than
definitions that consume nothing and return arguments.

\subsection{Stack Abbreviation Standards}

Abbreviations used in stack notation should be consistent.  \ref{tab-5-2} lists
most of the commonly used abbreviations.  (This table reappears in \Chap{E}.)
The terms ``single-length,'' ``double-length,'' etc. refer to
the size of a ``cell'' in the particular FORTH system.  (If the system uses a
16-bit cell, ``n'' represents a 16-bit number; if the system uses a 32-bit
cell, ``n'' represents a 32-bit number.)

\subsection{Notation of Flags}

\ref{tab-5-2} shows three ways to represent a boolean flag.  To illustrate, here
are three versions of the same stack comment for the word -TEXT:
\index{A!Abbreviations!in stack diagrams}
\begin{Code}
( at u a2 -- ?)
( at u a2 -- t=no-match)
( at u a2 -- f=match)
\end{Code}
%Page 153 in first edition.
%!!<TABLE>
\begin{table}[tttt]
\caption{Stack-comment abbreviations.}
\label{tab-5-2}
\vspace{1ex}
\blackline{1ex}
\begin{tabular}{@{\hspace{2.5em}}ll}
n             &  single-length signed number \\
d             &  double-length signed number \\
u             &  single-length unsigned number \\
ud            &  double-length unsigned number \\
t             &  triple-length \\
q             &  quadruple-length \\
c             &  7-bit character value \\
b             &  8-bit byte \\
?             &  boolean flag; or; \\
\quad t=         &  true \\
\quad f=         &  false \\
a or adr      &  address \\
acf           &  address of code field \\
apf           &  address of parameter field \\
,             &  (as prefix) address of \\
s d           &  (as a pair) source destination \\
lo hi         &  lower-limit upper-limit (inclusive, \\
\#            &   count \\
0             &  offset \\
i             &  index \\
m             &  mask \\
x             &  don't care (data structure notation) \\
\end{tabular}
\bigskip

An ``offset'' is a difference expressed in absolute units, such as bytes.

An ``index'' is a difference expressed in logical units, such as elements or records.
\vspace{0ex}
\blackline{0ex}
\end{table}
%!!</TABLE>
The equal sign after the symbols ``t'' and ``f'' equates the flag outcome
with its meaning.  The result-side of the second version would be read
``true means no match.''


\subsection{Notation of Variable Possibilities}
Some definitions yield a different stack effect under different circumstances.

If the number of items on the stack remains the same under all conditions,
but the items themselves change, you can use the vertical bar ( | )
to mean ``or.'' The following stack-effect comment describes a word that
returns either the address of a file or, if the requested file is not found,
zero:
\begin{Code}
( -- address|O=undefined-file)
\end{Code}
If the number of items in a stack picture can vary---in either the ``before''
or ``after'' picture---you must write out both versions of the entire stack
%Page 154 in first edition.
picture, along with the double-hyphen, separated by the ``or'' symbol.  For
instance:
\begin{Code}
-FIND ( -- apf len t=found | -- f=not-found )
\end{Code}
This comment indicates that if the word is found, three arguments are
returned (with the flag on top); otherwise only a false flag is returned.

Note the importance of the second ``--''.  Its omission would indicate
that the definition always returned three arguments, the top one being a
flag.

If you prefer, you can write the entire stack effect twice, either on
the same line, separated by three spaces:
\begin{Code}
  UP \ if zero: ( n -- n)    if non-zero:( n -- n n)
\end{Code}
or listed vertically:
\begin{Code}
-FIND  \     found: ( -- apf len t )
       \ not-found: ( -- f )
\end{Code}
\subsection{Data-Structure Comments}

A ``data-structure comment'' depicts the elements in a data structure.
For example, here's the definition of an insert buffer called |INSERT :
%!!<FORTHCODE>
\begin{Code}
CREATE |INSERT  64 ALLOT  \  { 1# | 63text }
\end{Code}
%!!<FORTHCODE>
The ``faces'' (curly-brackets) begin and end the structure comment; the
bars separate the various elements in the structure; the numbers represent
bytes per element.  In the comment above, the first byte contains the
count, and the remaining 63 bytes contain the text.

A ``bit comment'' uses the same format as a data-structure comment
to depict the meaning of bits in a byte or cell.  For instance, the bit
comment
\begin{Code}
{ Ibusy? | lacknowledge? | 2x | 6input-device | 6output-device}
\end{Code}
describes the format of a 16-bit status register of a communications channel.
The first two bits are flags, the second two bits are unused, and the
final pair of six-bit fields indicate the input and output devices which this
channel is connected to.

If more than one data structure employs the same pattern of elements,
write out the comment only once (possibly in the preamble), and
%Page 155 in first edition.
give a name to the pattern for reference in subsequent screens.  For instance,
if the preamble gives the above bit-pattern the name ``status,''
then ``status'' can be used in stack comments to indicate values with that
pattern:
\begin{Code}
: STATUS?  ( -- status) ... ;
\end{Code}
If a 2VARIABLE contains one double-length value, the comment should
be a stack picture that indicates the contents:
\begin{Code}
2VARIABLE PRICE  \ price in cents
\end{Code}
If a 2VARIABLE contains two single-length data elements, it's given a
stack picture showing what would be on the stack after a 2@.  Thus:
\begin{Code}
2VARIABLE MEASUREMENTS  ( height weight )
\end{Code}
This is different from the comment that would be used if
MEASUREMENTS were defined by CREATE.
\begin{Code}
CREATE MEASUREMENTS 4 ALLOT  \ { 2weight | 2height }
\end{Code}
(While both statements produce the same result in the dictionary, the use
of 2VARIABLE implies that the values will normally be ``2-fetched'' and
``2-stored'' together-thus we use a \emph{stack} comment.  The high-order part,
appearing on top of the stack, is listed to the right.  The use of CREATE
implies that the values will normally be fetched and stored
separately-thus we use a data structure comment.  The item in the 0th
position is listed to the left.)

\subsection{Input-stream Comments}

The input-stream comment indicates what words and/or strings are
presumed to be in the input stream.  \ref{tab-5-3} lists the designations used
for input stream arguments.
%!!<TABLE>
\begin{table*}[hhhh]
\caption{Input-stream comment designations.}
\label{tab-5-3}
\vspace{1ex}\blackline{1ex}
\centerline{\begin{tabular}{ll}
c     &  single character, blank-delimited \\
name  &  sequence of characters, blank delimited \\
text  &  sequence of characters, delimited by non-blank \\
\end{tabular}}

\medskip
Follow ``text'' with the actual delimiter required; e.g.: text" or text)

\vspace{1ex}\blackline{0ex}
\end{table*}

%!!</TABLE>
%% Note: AH : typewriter style footnote with *
%% Note: AH : text" and text) are literal examples, and should be slanted or such.

The input-stream comment appears \emph{before} the stack comment, and
is not encapsulated between its own pair of parentheses, but simply surrounded
%Page 156 in first edition.
by three spaces on each side.  For instance, here's one way to
comment the definition of ' (tick) showing first the input-stream comment,
then the stack comment:
\begin{Code}
: ,   \ name   ( -- a)
\end{Code}
If you prefer to use ( , the comment would look like this:
\begin{Code}
: ,   ( name   ( -- a)
\end{Code}

Incidentally, there are three distinct ways to receive string input.  To
avoid confusion, here are the terms:
%!!<ENUMERATION>
\begin{description}
\item[Scanning-for] means looking ahead in the input stream, either for a word or
number as in the case of tick, or for a delimiter as in the case of .'' and ( .
\item[Expecting] means waiting for.  EXPECT and KEY, and definitions that invoke
them, are ones that ``expect'' input.
\item[Presuming] indicates that in normal usage something will follow.  The word:
``scans-for'' the name to be defined, and ``presumes'' that a definition will
follow.
\end{description}
%!!</ENUMERATION>
The input-stream comment is only appropriate for input being scanned-for.

\subsection{Purpose Comments}

\begin{tip}
Every definition should bear a purpose comment unless:
a. its purpose is clear from its name or its stack-effect comment, or
b. if it consists of three or fewer words.
\end{tip}

The purpose comment should be kept to a minimum-never more than a
full line.  For example:
\begin{Code}
: COLD \ restore system to start condition
   ... ;
\end{Code}
use the imperative mood: ``set Foreground color,'' not ``sets Foreground
color.''

On the other hand, a word's purpose can often be described in terms
of its stack-effect comment.  You rarely need both a stack comment and a
purpose comment.  For instance:
\begin{Code}
: SPACES ( #)
\end{Code}
%Page 157 in first edition.
or
\begin{Code}
: SPACES  ( #spaces-to-type -- )  ... ;
\end{Code}

This definition takes as its incoming argument a number that represents
the number of spaces to type.
\begin{Code}
: ELEMENT  ( element# -- 'element)  2*  TABLE + ;
\end{Code}
This definition converts an index, which it consumes, into an address
within a table of 2-byte elements corresponding to the indexed element.
\begin{Code}
: PAD ( -- 'scratch-pad)  HERE 80 + ;
\end{Code}
This definition returns an address of a scratch region of memory.

Occasionally, readability is best served by including both types of
comment.  In this case, the purpose comment should appear last.  For
instance:
\begin{Code}
: BLOCK  ( n -- a)  \  ensure block n in buffer at a
\end{Code}

\begin{tip}
Indicate the type of comment by ordering: input-stream comments first,
stack-effect comments second, purpose comments last.
\end{tip}
For example:
\begin{Code}
: GET  \  name  ( -- a)   get first match
\end{Code}
If you prefer to use (, then write:
\begin{Code}
: GET  (  name  ( -- a) ( get first match
\end{Code}

If necessary, you can put the purpose comment on a second line:
\begin{Code}
: WORD   \   name   (c -- a)
   \ scan for string delimt'd by "c"; 1eave at a
   ... ;
\end{Code}
\subsection{Comments for Defining Words}
The definition of a defining word involves two behaviors:
%!!<ENUMERATION>
that of the defining word as it defines its ``child'' (compile-time behavior),
and
that of the child itself (run-time behavior).
%!!</ENUMERATION>
%Page 158 in first edition.

These two behaviors must be commented separately.

\begin{tip}
Comment a defining word's compile-time behavior in the usual way; comment
its run-time behavior separately, following the word DOES> (or
;CODE).
\end{tip}

For instance,
\begin{Code}
: CONSTANT   ( n )
   DOES> ( -- n)  @ ;
\end{Code}

The stack-effect comment for the run-time (child's) behavior represents
the net stack effect for the child word.  Therefore it does not include the
address returned by DOES>, even though this address is on the stack
when the run-time code begins.

Bad (run-time comment includes apf):
\begin{Code}
: ARRAY    \  name ( #cells)
   CREATE 2* ALLOT
    DOES>   ( i apf -- 'cell)   SWAP  2* + ;
\end{Code}
Good:
\begin{Code}
: ARRAY \  name ( #cells)
   CREATE 2* ALLOT
    DOES>   ( i -- 'cell)   SWAP  2* + ;
\end{Code}
%% Note: AH: CELLS instead of 2* is indicated
Words defined by this word ARRAY will exhibit the stack effect:
\begin{Code}
( i -- 'cell)
\end{Code}
If the defining word does not specify the run-time behavior, there still
exists a run-time behavior, and it may be commented:
\begin{Code}
: VARIABLE   ( name ( -- ) CREATE 2 ALLOT ;
   \ does>   ( -- adr )
\end{Code}
\subsection{Comments for Compiling Words}
As with defining words, most compiling words involve two behaviors:
%!!<ENUMERATION>
\begin{enumerate}
\item That of the compiling word as the definition in which it appears is compiled
\item That of the run-time routine which will execute when we invoke the word
being defined.  Again we must comment each behavior separately.
\end{enumerate}
%!!</ENUMERATION>
%Page 159 in first edition.

\begin{tip}
Comment a compiling word's run-time behavior in the usual way; comment
its compile-time behavior separately, beginning with the label ``Compile:''.
\end{tip}
For instance:
\begin{Code}
: IF   ( ? -- ) ...
\ Compile:   ( -- address-of-unresolved-branch)
   ... ; IMMEDIATE
\end{Code}
In the case of compiling words, the first comment describes the run-time
behavior, which is usually the \emph{syntax for using} the word.  The second comment
describes what the word \emph{actually does} in compiling (which is of less
importance to the user).
Other examples:
%!!<FORTHCODE>
: ABORT"  ( ? -- )
\ Compile:    text"   ( -- )

Occasionally a compiling word may exhibit a different behavior when it is
invoked outside a colon definition.  Such words (to be fastidious about it)
require three comments.  For instance:
\begin{Code}
: ASCII  (-- c)
\ Compile:   c   (--)
\ Interpret:   c   (-- c)
     ... ; IMMEDIATE
\end{Code}
\Chap{E} includes two screens showing good commenting style.

\section{Vertical Format vs.\ Horizontal Format}
The purpose of commenting is to allow a reader of your code to easily
determine what's going on.  But how much commenting is necessary? To
determine the level of commenting appropriate for your circumstances,
you must ask yourself two questions:
%!!<ENUMERATION>
Who will be reading my code?
How readable are my definitions?
%!!</ENUMERATION>
There are two basic styles of commenting to choose from.  The first style,
often called the ``vertical format,'' includes a step-by-step description of
the process, in the manner of a well-commented assembly language
listing.  These line-by-line comments are called ``narrative comments.''

%Page 160 in first edition.
\begin{Code}
\ CRC Checksum                             07/15/83
: ACCUMULATE    ( oldcrc char -- newcrc)
    256 *              \ shift char to hi-order byte
    XOR                \ & xor into previous crc
    8 0 DO             \ Then for eight repetitions,
       DUP 0< IF     \ if hi-order bit is "1"
          16386 XOR    \ xor it with mask and
          DUP +        \ shift it left one place
          1+           \ set lo-order bit to "1"
             ELSE      \ otherwise, i.e. hi-order bit is "0"
          DUP +        \ shift it left one place
             THEN
       LOOP;           \ complete the loop
\end{Code}
The other approach does not intersperse narrative comments between
code phrases.  This is called the ``horizontal format.''
\begin{Code}
\ CRC Checksum                             07/15/83
: ACCUMULATE    ( oldcrc char -- newcrc)
    256 *  XOR  8 0 DO  DUP 0< IF
       16386 XOR  DUP +  1+  ELSE  DUP +  THEN  LOOP;
\end{Code}

The vertical format is preferred when a large team of programmers are
coding and maintaining the application.  Typically, such a team will include
several junior-level programmers responsible for minor corrections.
In such an environment, diligent commenting can save a lot of time and
upset.  As Johnson of Moore Products Co. says: ``When maintaining code
you are usually interested in just one small section, and the more information
written there the better your chances for a speedy fix.''

Here are several pertinent rules required of the FORTH programmers
at Moore Products Co. (I'm paraphrasing):
%!!<ENUMERATION>
\begin{enumerate}
\item A vertical format will be used.  Comments will appear to the right of the
source code, but may continue to engulf the next line totally if needed.
\item There should be more comment characters than source characters.  (The
company encourages long descriptive names, greater than ten characters,
and allows the names to be counted as comment characters.)
\item Any conditional structure or application word should appear on a separate
line.  ``Noise words'' can be grouped together.  Indentation is used to show
nested conditionals.
\end{enumerate}
%!!</ENUMERATION>
There are some difficulties with this format, however.  For one thing, line-by-line
commenting is time-consuming, even with a good screen editor.
Productivity can be stifled, especially when stopping to write the comments
breaks your chain of thought.

Also, you must also carefully ensure that the comments are up-to-date.
Very often code is corrected, the revision is tested, the change
%Page 161 in first edition.
works-and the programmer forgets to change the comments.  The more
comments there are, the more likely they are to be wrong.  If they're
wrong, they're worse than useless.

This problem can be alleviated if the project supervisor carefully reviews
code and ensures the accuracy of comments.

Finally, line-by-line commenting can allow a false sense of security.
Don't assume that because each \emph{line} has a comment, the \emph{application} is
well-commented.  Line-by-line commenting doesn't address the significant
aspects of a definition's operation.  What, for instance, is the thinking
behind the checksum algorithm used? Who knows, from the narrative
comments?

To properly describe, in prose, the implications of a given procedure
usually requires many paragraphs, not a single phrase.  Such descriptions
properly belong in auxiliary documentation or in the chapter preamble.

Despite these cautions, many companies find the vertical format
necessary.  Certainly a team that is newly exposed to FORTH should
adopt it, as should any very large team.

What about the horizontal format? Perhaps it's an issue of art vs.
practicality, but I feel compelled to defend the horizontal format as equally
valid and in some ways superior.

If FORTH code is really well-written, there should be nothing ambiguous
about it.  This means that:
%!!<ENUMERATION>
\begin{itemize}
\item supporting lexicons have a well-designed syntax
\item stack inputs and outputs are commented
\item the purpose is commented (if it's not clear from the name or stack comment)
\item definitions are not too long
\item not too many arguments are passed to a single definition via the stack (see
``The Stylish Stack'' in \Chap{7}).
\end{itemize}
%!!</ENUMERATION>
FORTH is simply not like other languages, in which line-by-line
commenting is one of the few things you can do to make programs more
readable.

Skillfully written FORTH code is like poetry, containing precise
meaning that both programmer and machine can easily read.  Your goal
should be to write code that does not need commenting, even if you
choose to comment it.  Design your application so that the code, not the
comments, conveys the meaning.

If you succeed, then you can eliminate the clutter of excessive
commenting, achieving a purity of expression without redundant
explanations.

\begin{tip}
The most-accurate, least-expensive documentation is self-documenting
code.
\end{tip}
%Page 162 in first edition.
%!!<ILLUSTRATION>
\wepsfigp{fig5-2}{Wiggins, proud of his commenting technique.}
%!!</ILLUSTRATION>

Unfortunately, even the best programmers, given the pressure of a
deadline, may write working code that is not easily readable without comments.
If you are writing for yourself, or for a small group with whom
you can verbally communicate, the horizontal format is ideal.  Otherwise,
consider the vertical format.

\section{Choosing Names: The Art}

%!!<SMALLPRINT>
\begin{tfquot}
Besides a mathematical inclination, an exceptionally good mastery of one's
native tongue is the most vital asset of a competent programmer (Prof.
Edsger W. Dijkstra \cite{dijkstra82}).
\end{tfquot}
%!!</SMALLPRINT>
We've talked about the significance of using names to symbolize ideas
and objects in the application.  The choosing of
names\index{W!Words:!choosing names|(} turns out to be an
important part of the design process.

Newcomers tend to overlook the important of names.  ``After all,''
they think, ``the computer doesn't care what names I choose.''

But good names are essential for readability.  Moreover, the mental
exercise of summoning a one-word description bears a synergistic effect
on your perceptions of what the entity should or should not do.
Here are some rules for choosing good names:

\begin{tip}
Choose names according to ``what,'' not ``how.''
\end{tip}
A definition should hide the complexities of implementation from other
definitions which invoke it.  The name, too, should hide the details of the
procedure, and instead should describe the outward appearance or net
effect.

For instance, the FORTH word ALLOT simply increments the dictionary
pointer (called DP or H in most systems).  But the name ALLOT
is better than D P +! because the user is thinking of reserving space, not
incrementing a pointer.

The '83 Standard adopted the name CMOVE> instead of the previous
name for the same function, <CMOVE.  The operation makes it
possible to copy a region of memory forward into overlapping memory.  It
accomplishes this by starting with the last byte and working backward.
In the new name, the forwardness of the ``what'' supercedes the
backwardness of the ``how.''

\begin{tip}
Find the most expressive word.
\end{tip}
%Page 164 in first edition.
%!!<SMALLPRINT>
A powerful agent is the right word.  Whenever we come upon one of those intensely
right words in a book or a newspaper the resulting effect is physical
as well as spiritual, and electrically prompt (Mark Twain).

The difference between the right word and the almost-right word is like the
difference between lightning and the lightning bug (Mark Twain).

Suit the action to the word, the word to the action (Shakespeare, Hamlet,
Act III).
%!!</SMALLPRINT>
Henry Laxen, a FORTH consultant and author, suggests that the most
important FORTH development tool is a good thesaurus \cite{laxen}.

Sometimes you'll think of an adequate word for a definition, but it
doesn't feel quite right.  It may be months later before you realize
that you fell short of the mark.  In the Roman numeral example in
\Chap{4}, there's a word that handles the exception case: numbers that
are one-less-than the next symbol's value.  My first choice was
4-0R-9.  That's awkward, but it was much later that I thought of
ALMOST.

Most fig-FORTH systems include the word VLIST, which lists the
names of all the words in the current vocabulary.  After many years someone
realized that a nicer name is WORDS.  Not only does WORDS sound
more pleasant by itself, it also works nicely with vocabulary names.  For
instance:
\begin{Code}
EDITOR WORDS
\end{Code}
or
\begin{Code}
ASSEMBLER WORDS
\end{Code}
On the other hand, Moore\index{M!Moore, Charles} points out that inappropriate names can
become a simple technique for encryption.  If you need to provide security
when you're forced to distribute source, you can make your code very
unreadable by deliberately choosing misleading names.  Of course, maintenance
becomes impossible.

\begin{tip}
Choose names that work in phrases.
\end{tip}
Faced with a definition you don't know what to call, think about how the
word will be used in context.  For instance:
\begin{Code}
SHUTTER OPEN
\end{Code}
OPEN is the appropriate name for a word that sets a
bit in an 1/0 address identified with the name
SHUTTER.
%Page 165 in first edition.
\begin{Code}
3 BUTTON DOES IGNITION
\end{Code}
%!!<INDENT3LEVELS>
DOES is a good choice for a word that vectors the
address of the function IGNITION into a table of
functions, so that IGNITION will be executed when
Button 3 is pushed.
%!!</INDENT3LEVELS>
\begin{Code}
SAY HELLO
\end{Code}
%!!<INDENT3LEVELS>
SAY is the perfect choice for vectoring HELLO into an
execution variable.  (When I first wrote this example
for Starting FORTH, I called it VERSION.  Moore\index{M!Moore, Charles|(}
reviewed the manuscript and suggested SAY, which is
clearly much better.)
%!!</INDENT3LEVELS>
\begin{Code}
I'M HARRY
\end{Code}
%!!<INDENT3LEVELS>
The word I'M seems more natural than LOGON HARRY,
LOGIN HARRY or SESSION HARRY, as often seen.
%!!</INDENT3LEVELS>
%!!<SMALLPRINT>
The choice of I'M is another invention of Moore, who says:

I detest the word LOGON.  There is no such word in English.  I was looking
for a word that said, ``I'm. ...'' It was a natural.  I just stumbled across it.
Even though it's clumsy with that apostrophe, it has that sense of
rightness.

All these little words are the nicest way of getting the ``Aha!'' reaction.  If
you think of the right word, it is obviously the right word.

If you have a wide recall vocabulary, you're in a better position to come up
with the right word.
%!!</SMALLPRINT>
Another of Moore's favorite words is TH, which he uses as an array indexing
word.  For instance, the phrase
\begin{Code}
5 TH
\end{Code}
returns the address of the ``fifth'' element of the array.
\index{M!Moore, Charles|)}

\begin{tip}
Spell names in full.
\end{tip}
I once saw some FORTH code published in a magazine in which the
author seemed hell-bent on purging all vowels from his names, inventing
such eyesores as DSPL-BFR for ``display buffer.'' Other writers seem to
think that three characters magically says it all, coining LEN for
``length.'' Such practices reflect thinking from a bygone age.

FORTH words should be fully spelled out.  Feel proud to type every
letter of INITIALIZE or TERMINAL or BUFFER.  These are the words
you mean.
%Page 166 in first edition.

The worst problem with abbreviating a word is that you forget just
how you abbreviated it.  Was that DSPL or DSPLY?

Another problem is that abbreviations
\index{A!Abbreviations!hinder readability}hinder readability.
Any programming language is hard enough to read without compounding
the difficulty.

Still, there are exceptions.  Here ate a few:
%!!<ENUMERATION>
\begin{enumerate}
\item Words that you use extremely frequently in code. FORTH employs a handful
of commands that get used over and over, but have little or no intrinsic
meaning:
\begin{Code}
:   ;   @   !   .   ,
\end{Code}
But there are so few of them, and they're used so often, they become old
friends.  I would never want to type, on a regular basis,
\begin{Code}
DEFINE  END-DEFINITION  FETCH  STORE  PRINT  COMPILE#
\end{Code}
(Interestingly, most of these symbols don't have English counterparts.  We
use the phrase ``\emph{colon} definition'' because there's no other term; we say
``\emph{comma} a number into the dictionary'' because it's not exactly compiling,
and there's no other term.)
\item Words that a terminal operator might use frequently to control an operation.
These words should be spelled as single letters, as are line editor
commands.
\item Words in which familiar usage implies that they be abbreviated.  FORTH
assembler mnemonics are typically patterned after the manufacturer's suggested
mnemonics, which are abbreviations (such as JMP and MOV).
\end{enumerate}
%!!</ENUMERATION>
Your names should be pronounceable; otherwise you may regret it when
you try to discuss the program with other people.  If the name is symbolic,
invent a pronunciation (e.g., >R is called ``to-r''; R> is called
``r-from'').

\begin{tip}
Favor short words.
\end{tip}
Given the choice between a three-syllable word and a one-syllable word
that means the same thing, choose the shorter.  BRIGHT is a better name
than INTENSE.  ENABLE is a better name than ACTIVATE; GO,
RUN, or ON may be better still.

Shorter names are easier to type.  They save space in the source
screen.  Most important, they make your code crisp and clean.

\begin{tip}
Hyphenated names may be a sign of bad factoring.
\end{tip}
%Page 168 in first edition.

Moore:\index{M!Moore, Charles|(}

\begin{tfquot}
There are diverging programming styles in the FORTH community.  One
uses hyphenated words that express in English what the word is doing.
You string these big long words together and you get something that is
quite readable.

But I immediately suspect that the programmer didn't think out the words
carefully enough, that the hyphen should be broken and the words defined
separately.  That isn't always possible, and it isn't always advantageous.
But I suspect a hyphenated word of mixing two concepts.
\end{tfquot}\index{M!Moore, Charles|)}

Compare the following two strategies for saying the same thing:
\begin{Code}
ENABLE-LEFT-MOTOR        LEFT MOTOR ON
ENABLE-RIGHT-MOTOR       RIGHT MOTOR ON
DISABLE-LEFT-MOTOR       LEFT MOTOR OFF
DISABLE-RIGHT-MOTOR      RIGHT MOTOR OFF
ENABLE-LEFT-SOLENOID     LEFT SOLENOID ON
ENABLE-RIGHT-SOLENOID    RIGHT SOLENOID ON
DISABLE-LEFT-SOLENOID    LEFT SOLENOID OFF
DISABLE-RIGHT-SOLENOID   RIGHT SOLENOID OFF
\end{Code}
The syntax on the left requires eight dictionary entries; the syntax on the
right requires only six-and some of the words are likely to be reused in
other parts of the application.  If you had a MIDDLE motor and solenoid
as well, you'd need only seven words to describe sixteen combinations.

\begin{tip}
Don't bundle numbers into names.
\end{tip}
Watch out for a series of names beginning or ending with numbers, such
as 1CHANNEL, 2CHANNEL, 3CHANNEL, etc.

This bundling of names and numbers may be an indication of bad
factoring.  The crime is similar to hyphenation, except that what should
be factored out is a number, not a word.  A better factoring of the above
would be
\begin{Code}
1 CHANNEL
2 CHANNEL
3 CHANNEL
\end{Code}
In this case, the three words were reduced to one.
Often the bundling of names and numbers indicates fuzzy naming.

%Page 168 in first edition.
In the above case, more descriptive names might indicate the purpose of
the channels, as in
\begin{Code}
VOICE  , TELEMETRY  , GUITAR
\end{Code}
%% Note: AH: the commas be better left out here.

We'll amplify on these ideas in the next chapter on ``Factoring.''

\section{Naming Standards: The Science}

\begin{tip}
Learn and adopt FORTH's naming conventions.
\end{tip}
In the quest for short, yet meaningful names, FORTH programmers
have adopted certain naming conventions.  \Chap{E} includes a list of
the most useful conventions developed over the years.

An example of the power of naming conventions is the use of ``dot''
to mean ``print'' or ``display.'' FORTH itself uses
\begin{Code}
.  D.  U.R
\end{Code}
for displaying various types of numbers in various formats.  The convention
extends to application words as well.  If you have a variable called
DATE, and you want a word that displays the date, use the name
\begin{Code}
.DATE
\end{Code}
A caution: The overuse of prefixes and suffixes makes words uglier and
ultimately less readable.  Don't try to describe everything a word does by
its name alone.  After all, a name is a symbol, not a shorthand for code.
Which is more readable and natural sounding?:
%!!<SMALLPRINT>
\begin{tfquot}
Oedipus complex
\end{tfquot}
%!!</SMALLPRINT>
(which bears no intrinsic meaning), or
%!!<SMALLPRINT>
\begin{tfquot}
subconscious-attachment-to-parent-of-opposite-sex complex
\end{tfquot}
%!!</SMALLPRINT>
Probably the former, even though it assumes you know the play.

\begin{tip}
Use prefixes and suffices to differentiate between like words rather than to
cram details of meaning into the name itself.
\end{tip}
%Page 169 in first edition.

For instance, the phrase
\begin{Code}
... DONE IF CLOSE THEN ...
\end{Code}
is just as readable as
\begin{Code}
... DONE? IF CLOSE THEN ...
\end{Code}
and cleaner as well.  It is therefore preferable, unless we need an additional
word called DONE (as a flag, for instance).
A final tip on naming:\index{W!Words:!choosing names|)}

\begin{tip}
Begin all hex numbers with ``0'' (zero) to avoid potential collisions with
names.
\end{tip}
For example, write 0ADD, not ADD.

By the way, don't expect your FORTH system to necessarily conform
to the above conventions.  The conventions are meant to be used in
new applications.

FORTH was created and refined over many years by people who
used it as a means to an end.  At that time, it was neither reasonable nor
possible to impose naming standards on a tool that was still growing and
evolving.

Had FORTH been designed by committee, we would not love it so.

\section{More Tips for Readability}

Here are some final suggestions to make your code more readable.
(Definitions appear in \Chap{C}.)

One constant that pays for itself in most applications is
BL\index{B!Blank space(BL)} (the ASCII value for ``blank-space'').

The word ASCII is used primarily within colon definitions to free
you from having to know the literal value of an ASCII character.  For instance,
instead of writing:
\begin{Code}
: ( 41 WORD DROP: IMMEDIATE
\end{Code}
where 41 is the ASCII representation for right-parenthesis, you can write
\begin{Code}
: ( ASCII) WORD DROP; IMMEDIATE
\end{Code}
A pair of words that can make dealing with booleans more readable are
TRUE and FALSE.  With these additions you can write phrases such as

%Page 170 in first edition.
\begin{Code}
TRUE 'STAMP? !
\end{Code}
to set a flag or
\begin{Code}
FALSE 'STAMP? !
\end{Code}
to clear it.

(I once used T and F, but the words are needed so rarely I now heed
the injunction against abbreviations.)

As part of your application (not necessarily part of your FORTH
system), you can take this idea a step further and define:
\begin{Code}
: ON   ( a) TRUE SWAP ! ;
: OFF   ( a) FALSE SWAP ! ;
\end{Code}
These words allow you to write:
\begin{Code}
'STAMP? ON
\end{Code}
or
\begin{Code}
'STAMP? OFF
\end{Code}
Other names for these definitions include SET and RESET, although
SET and RESET most commonly use bit masks to manipulate individual
bits.

An often-used word is WITHIN,\index{W!WITHIN} which determines whether a given
value lies within two other values.  The syntax is:
\begin{Code}
n  lo hi WITHIN
\end{Code}
where ``n'' is the value to be tested and ``lo'' and ``hi'' represent the range.
WITHIN returns true if ``n'' is \emph{greater-than} or \emph{equal-to} ``lo'' and
\emph{less-than} ``hi.'' This use of the non-inclusive upper limit parallels the syntax
%% Note: AH : the original text has greater-then not slanted. Considered a typo.
of DO LOOPs.

Moore\index{M!Moore, Charles} recommends the word UNDER+.  It's useful for adding a
value to the number just under the top stack item, instead of to the top
stack item.  It could be implemented in high level as:
\begin{Code}
: UNDER+ (a b c -- a+c b) ROT + SWAP !
\end{Code}

\section{Summary}
Maintainability requires readability.  In this chapter we've enumerated
various ways to make a source listing more readable.  We've assumed a
%Page 171 in first edition.
policy of making our code as self-documenting as possible.  Techniques include
listing organization, spacing and indenting, commenting, name
choices, and special words that enhance clarity.
We've mentioned only briefly auxiliary documentation, which includes
all documentation apart from the listing itself.  We won't discuss
auxiliary documentation further in this volume, but it remains an integral
part of the software development process.

\begin{references}{9}
\bibitem{stevenson81} Gregory Stevenson, ``Documentation Priorities,''
1981 FORML Conference Proceedings, p. 401.
\bibitem{lee81} Joanne Lee, ``Quality Assurance in a FORTH
Environment,'' (Appendix A), 1981 FORML Proceedings, p. 363.
\bibitem{dijkstra82} Edsger W. Dijkstra, Selected Writings on
Computing:\ A Personal Perspective, New York, Springer Verlag, Inc.,
1982.
\bibitem{laxen} Henry Laxen, ``Choosing Names,'' FORTH Dimensions,
vol. 4, no.\ 4, FORTH Interest Group.
\end{references}

%Page 172 in first edition.

%% Thinking Forth
%% Copyright (C) 2004 Leo Brodie
%% Initial transcription by Ed Beroset
%% Translated into Spanish (es_ES) by Francisco Escobedo
%% 
%% Chapter: Appendix C, Other Utilities Described in This Book
\Chapmark{C}
\chapter{Otras~utilidades descritas en~este~libro}

\initial Este ap��ndice est�� aqu�� para ayudarle a definir algunas de
las palabras a la que se hace referencia en este libro y que pueden no
existir en su sistema. Las definiciones se hacen seg��n la norma del
\Forth{}-83.

\section{Del \Chap{4}}

Una definici��n of \forthb{ASCII}\index{A!ASCII} que funcionar�� en
Forth{}-83 es:
\begin{Code}
: ASCII  ( -- c)  \  Compilando:  c  ( -- )
\ Interpretando:   c   ( -- c)
     BL WORD 1+ C@  STATE @
     IF [COMPILE] LITERAL  THEN ; IMMEDIATE
\end{Code}

\section{Del \Chap{5}}

La palabra \forthb{\bs}\index{S!salto, ��rdenes de} puede definirse
como:
\begin{Code}
: \  ( salta el resto de la l��nea)
     >IN @  64 / 1+  64 *  >IN ! ; IMMEDIATE
\end{Code}
Si decide no usar \forthb{EXIT} para concluir una pantalla, puede
definir \forthb{\bs S} como:
\begin{Code}
: \S   1024 >IN ! ;
\end{Code}
\index{F!FH|(}%
La palabra \forthb{FH} puede definirse simplemente como:
\begin{Code}
: FH   \   ( desplazamiento -- desplazamiento-bloque)   "from here" ("desde aqu��")
    BLK @ + ;
\end{Code}
Esta descomposici��n permite usar \forth{FH} de muchas formas, p.e.:
\begin{Code}
: TEST   [ 1 FH ] LITERAL LOAD ;
\end{Code}
or
\begin{Code}
: SEE   [ 2 FH ] LITERAL LIST ;
\end{Code}
Una versi��n ligeramente m��s complicada de \forth{FH} tambi��n permite
editar o cargar una pantalla con una frase como ``\forth{14 FH
  LIST}'', relativa a la pantalla que se acaba de listar
(\forth{SCR}):
\begin{Code}
: FH   \   ( desplazamiento -- desplazamiento-bloque)   "from here" ("desde aqu��")
     BLK @  ?DUP 0= IF  SCR @  THEN  + ;
\end{Code}
\index{F!FH|)}
\forthb{BL}\index{B!Blank space (BL)} es una simple constante:
\begin{Code}
32 CONSTANT BL
\end{Code}
\forthb{TRUE}\index{T!TRUE} y \forthb{FALSE}\index{F!FALSE}
pueden definirse como:
\begin{Code}
0 CONSTANT FALSE
-1 CONSTANT TRUE
\end{Code}
(las palabras de control de \Forth{} como \forth{IF} y \forth{UNTIL}
interpretan cero como ``falso'' y cualquier valor distinto de cero
como ``verdadero''. Antes de \Forth{}-83, el convenio era indicar
``verdadero'' con el valor $1$. Desde \Forth{}-83, sin embargo,
``verdadero'' se indica con hex \code{FFFF}, que es el n��mero con
signo $-1$ (todos los bits a 1).)

\forthb{WITHIN}\index{W!WITHIN} puede definirse en alto nivel as��:
\begin{Code}
: WITHIN  ( n bajo alto+1 -- ?)
     >R  1- OVER <  SWAP R>  < AND ;
\end{Code}
o
\begin{Code}
: WITHIN ( n bajo alto+1 -- ?)
   OVER -  >R - R> U< ;
\end{Code}

\section{Del \Chap{8}}

La manera de hacer \forthb{LEAP}\index{L!LEAP} depender�� de c��mo su
sistema hace los bucles \forthb{DO} \forthb{LOOP}. Si \forthb{DO}
guarda 2 elementos en la pila de retorno (el ��ndice y el l��mite),
\forthb{LEAP} debe descartar ambos adem��s de otro elemento de la pila
de retorno para salir:
\begin{Code}
: LEAP   R> R> 2DROP  R> DROP ;
\end{Code}
Si \forthb{DO} guarda \emph{3} elemento en la pila de retorno, debe
definirse:
\begin{Code}
: LEAP   R> R> 2DROP  R> R> 2DROP ;
\end{Code}


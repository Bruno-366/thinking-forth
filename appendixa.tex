%% Thinking Forth
%% Copyright (C) 2004 Leo Brodie
%% Initial transcription by Ed Beroset
%% 
%% Chapter: Appendix A, Overview of FORTH (For Newcomers)
\appendix{}
\Chapmark{A}
\chapter{Overview
of~FORTH
(For~Newcomers)}
\section{The Dictionary}
FORTH is expressed in words (and numbers) and is separated by spaces:
\begin{verbatim}
    HAND OPEN  ARM LOWER  HAND CLOSE  ARM RAISE 
\end{verbatim}
Such commands may be typed directly from the keyboard, or edited onto 
mass storage then ``{\bf LOAD}''ed.

All words, whether included with the system or user-defined,
exist in the ``dictionary,'' a linked list.  A ``defining word,'' is used to add
new names to the dictionary.  One defining word is {\bf :} (pronounced
``colon''), which is used to define a new word in terms of previously-
defined words.  Here is how one might define a new word called LIFT:
\begin{verbatim}
    : LIFT   HAND OPEN  ARM LOWER  HAND CLOSE  ARM RAISE ;
\end{verbatim}
The {\bf ;} terminates the definition.  The new word LIFT may now be used
instead of the long sequence of words that comprise its definition.

FORTH words can be nested like this indefinitely.  Writing a 
FORTH application consists of building increasingly powerful definitions,
such as this one, in terms of previously defined ones.

Another defining word is {\bf CODE}, which is used in place of colon to
define a command in terms of machine instructions for the native processor.
Words defined with {\bf CODE} are indistinguishable to the user from 
words defined with colon.  {\bf CODE} definitions are needed only for the most
time-critical portions of an applications, if at all.
\subsection{Data Structures}
Still another defining word is {\bf CONSTANT}, which is used like this:
\begin{verbatim}
    17 CONSTANT SEVENTEEN
\end{verbatim}
The new word SEVENTEEN can now be used in place of the actual number 17.

The defining word VARIABLE creates a location for temporary data.  
VARIABLE is used like this:
\begin{verbatim}
    VARIABLE BANANAS
\end{verbatim}
This reserves a location which is identified by the name BANANAS.

Fetching the contents of this location is the job of the word {\bf @}
(pronounced ``fetch'').  For instance,
\begin{verbatim}
    BANANAS @
\end{verbatim}
fetches the contents of the variable BANANAS.  Its counterpart is {\bf !}
(pronounced ``store''), which stores a value into the location, as in:
\begin{verbatim}
    100 BANANAS !
\end{verbatim}
FORTH also provides a word to increment the current value by the given
value; for instance, the phrase
\begin{verbatim}
    2 BANANAS +!
\end{verbatim}
increments the count by two, making it 102.

FORTH provides many other data structure operators, but more
importantly, it provides the tools necessary for the programmer to
create any time of data structure needed for the application.
\subsection{The Stack}
In FORTH, variables and arrays are used for saving values that may be
required by many other routines and/or at unpredictable times.  They are {\em not}
used for the local passing of data between the definitions.  For this, FORTH
employs a much simpler mechanism: the data stack.

When you type a number, it goes on the stack.  When you invoke a 
word which has numeric input, it will take it from the stack.  Thus the
phrase
\begin{verbatim}
    17 SPACES
\end{verbatim}
will display seventeen blanks on the current output device.  ``17'' pushes
the binary value 17 onto the stack; the word {\bf SPACES} consumes it.

A constant also pushes its value onto the stack; thus the phrase:
\begin{verbatim}
    SEVENTEEN SPACES
\end{verbatim}
has the same effect.

The stack operates on a ``last-in, first-out'' (LIFO) basis.  This means
that data can be passed between words in an orderly, modular way, consistent
with the nesting of colon definitions.

For instance, a defintion called GRID might invoke the phrase 17 SPACES.  
This temporary activity on the stack will be transmparent to any
other definition that invokes GRID because the value placed on the stack
is removed before the defintion of GRID ends.  The calling definition 
might have placed some numbers of its own on the stack prior to calling 
GRID.  These will remain on the stack, unharmed, until GRID has been 
executed  and calling definition continues.   % probably should be the calling..
\subsection{Control Structures}
FORTH provides all the control structures needed for structured, GOTO-less
programming.

The syntax of the {\bf IF THEN} construct is as follows:
\begin{verbatim}
    ... ( flag ) IF  KNOCK  THEN  OPEN ...
\end{verbatim}
The ``flag'' is a value on the stack, consumed by IF. A non-zero value
indicates true, zero indicates false.  A true flag causes the code after {\bf IF} (in
this case, the word KNOCK) to be executed.  The word {\bf THEN} marks the
end of the conditional phrase; execution resumes with the word OPEN.  A
false flag causes the code between {\bf IF} and {\bf THEN} to {\em not} be executed.  In
either case, OPEN will be performed.

The word {\bf ELSE} allows an alternat phrase to be executed in the
false case.  In the phrase:
\begin{verbatim}
    ( flag ) IF KNOCK  ELSE  RING  THEN  OPEN ...
\end{verbatim}
the word KNOCK will be performed if the flag is true, otherwise the word
RING will be performed.  Either way, execution will continue starting
with OPEN.

FORTH also provides for indexed loops in the form
\begin{verbatim}
    ( limit) ( index) DO ... LOOP
\end{verbatim}
and indefinite loops in the forms:
\begin{verbatim}
    ... BEGIN  ...  ( flag) UNTIL
\end{verbatim}
and
\begin{verbatim}
    ... BEGIN  ...  ( flag) WHILE ... REPEAT ;
\end{verbatim}
\subsection{For the Whole Story}
For a complete instroduction to the FORTH command set, read {\em Starting
FORTH}, published by Prentice-Hall.

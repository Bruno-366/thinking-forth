%% Thinking Forth
%% Copyright (C) 2004 Leo Brodie
%% Initial transcription by Ed Beroset
%% 
%% Chapter: Appendix A, Overview of \Forth{} (For Newcomers)
\appendix{}
\Chapmark{A}
\chapter{Overview
of~\Forth{}
(For~Newcomers)}
\section{The Dictionary}\index{D!Dictionary:!defined|(}%
\Forth{}\index{F!\Forth{}!overview of|(} is expressed in words (and numbers) and is separated by spaces:
\begin{Code}
HAND OPEN  ARM LOWER  HAND CLOSE  ARM RAISE 
\end{Code}
Such commands may be typed directly from the keyboard, or edited onto 
mass storage then ``\forthb{LOAD}''ed.

All words, whether included with the system or user-defined,
exist in the ``dictionary,'' a linked list.  A ``defining word,'' is used to add
new names to the dictionary.  One defining word is \forthb{:} (pronounced
``colon''), which is used to define a new word in terms of previously-
defined words.  Here is how one might define a new word called \forth{LIFT}:
\begin{Code}
: LIFT   HAND OPEN  ARM LOWER  HAND CLOSE  ARM RAISE ;
\end{Code}
The \forthb{;} terminates the definition.  The new word \forth{LIFT} may now be used
instead of the long sequence of words that comprise its definition.

\Forth{} words can be nested like this indefinitely.  Writing a 
\Forth{} application consists of building increasingly powerful definitions,
such as this one, in terms of previously defined ones.

Another defining word is \forthb{CODE},\index{C!CODE} which is used in place of colon to
define a command in terms of machine instructions for the native processor.
Words defined with \forthb{CODE} are indistinguishable to the user from 
words defined with colon.  \forthb{CODE} definitions are needed only for the most
time-critical portions of an applications, if at all.
\section{Data Structures}%
\index{D!Data structures:!operators}%
Still another defining word is \forthb{CONSTANT},\index{C!CONSTANT} which
is used like this:
\begin{Code}
17 CONSTANT SEVENTEEN
\end{Code}
The new word \forth{SEVENTEEN} can now be used in place of the actual number 17.

The defining word \forth{VARIABLE}\index{V!VARIABLE} creates a location
for temporary data. \forth{VARIABLE} is used like this:
\begin{Code}
VARIABLE BANANAS
\end{Code}
This reserves a location which is identified by the name \forth{BANANAS}.

Fetching the contents of this location is the job of the word \forthb{@}
(pronounced ``fetch'').  For instance,
\begin{Code}
BANANAS @
\end{Code}
fetches the contents of the variable \forth{BANANAS}.  Its counterpart is \forthb{!}
(pronounced ``store''), which stores a value into the location, as in:
\begin{Code}
100 BANANAS !
\end{Code}
\Forth{} also provides a word to increment the current value by the given
value; for instance, the phrase
\begin{Code}
2 BANANAS +!
\end{Code}
increments the count by two, making it 102.

\Forth{} provides many other
data structure operators\index{D!Data structures:!operators}, but more
importantly, it provides the tools necessary for the programmer to
create any time of data structure needed for the application.%
\index{D!Dictionary:!defined|)}%

\section{The Stack}
In \Forth{}, variables and arrays are used for saving values that may be
required by many other routines and/or at unpredictable times.  They are {\em not}
used for the local passing of data between the definitions.  For this, \Forth{}
employs a much simpler mechanism: the data stack.

When you type a number, it goes on the stack.  When you invoke a 
word which has numeric input, it will take it from the stack.  Thus the
phrase
\begin{Code}
17 SPACES
\end{Code}
will display seventeen blanks on the current output device.  ``17'' pushes
the binary value 17 onto the stack; the word \forthb{SPACES} consumes it.

A constant also pushes its value onto the stack; thus the phrase:
\begin{Code}
SEVENTEEN SPACES
\end{Code}
has the same effect.

The stack operates on a ``last-in, first-out'' (LIFO) basis.  This means
that data can be passed between words in an orderly, modular way, consistent
with the nesting of colon definitions.

For instance, a defintion called \forth{GRID} might invoke the phrase
\forth{17 SPACES}.  This temporary activity on the stack will be
transmparent to any other definition that invokes \forth{GRID} because
the value placed on the stack is removed before the defintion of
\forth{GRID} ends.  The calling definition might have placed some
numbers of its own on the stack prior to calling \forth{GRID}.  These
will remain on the stack, unharmed, until \forth{GRID} has been
executed and the calling definition continues.
% original ``and calling definition continues.

\section{Control Structures}
\Forth{} provides all the control structures%
\index{C!Control structures:!defined}%
needed for structured, GOTO-less programming.

The syntax of the \forthb{IF THEN} construct is as follows:
\begin{Code}
... ( flag ) IF  KNOCK  THEN  OPEN ...
\end{Code}
The ``flag''\index{F!Flags} is a value on the stack, consumed by IF. A non-zero value
indicates true, zero indicates false.  A true flag causes the code after \forthb{IF} (in
this case, the word \forth{KNOCK}) to be executed.  The word \forthb{THEN} marks the
end of the conditional phrase; execution resumes with the word \forth{OPEN}.  A
false flag causes the code between \forthb{IF} and \forthb{THEN} to {\em not} be executed.  In
either case, \forth{OPEN} will be performed.

The word \forthb{ELSE}\index{E!ELSE} allows an alternate phrase to be executed
in the false case. In the phrase:
\begin{Code}
( flag ) IF KNOCK  ELSE  RING  THEN  OPEN ...
\end{Code}
the word \forth{KNOCK} will be performed if the flag is true, otherwise the word
\forth{RING} will be performed.  Either way, execution will continue starting
with \forth{OPEN}.

\Forth{} also provides for indexed loops\index{L!Loops} in the form
\begin{Code}
( limit) ( index) DO ... LOOP
\end{Code}
and indefinite loops in the forms:
\begin{Code}
... BEGIN  ...  ( flag) UNTIL
\end{Code}
and
\begin{Code}
... BEGIN  ...  ( flag) WHILE ... REPEAT ;
\end{Code}
\section{For the Whole Story}
For a complete instroduction to the \Forth{} command set, read {\em Starting
\Forth{}}, published by Prentice-Hall.%
\index{F!\Forth{}!overview of|)}%
